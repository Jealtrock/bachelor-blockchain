\documentclass[12pt, a4paper, twoside, openright]{report}
\usepackage{amsmath}
\usepackage{amsfonts}
\usepackage{MnSymbol}
\usepackage[ansinew]{inputenc}
\usepackage[T1]{fontenc}
\usepackage[ngerman]{babel}
\usepackage{graphicx}
\usepackage{fancyhdr}
\usepackage{array}
\usepackage{bibgerm}
\usepackage{listings}
\usepackage{color}
\usepackage{setspace} 						% zeilenabstand
\usepackage{parskip}							% absatzabstand
\usepackage{titling}
\usepackage{lmodern}							% pdfs werden damit sch�rfer

% 1,5 zeilenabstand
\onehalfspacing

% pfad, wo nach bilder gesucht werden soll
\graphicspath{{images/}}

% neuer spaltentyp der wie p spalten in gr��e aufteilt aber gleichzeitig zentriert
\newcolumntype{P}[1]{>{\centering\arraybackslash}p{#1}}

%%%%%%%%%%%%%%%%%%%%%%%%%%%%%%%%%%%%%%%%%%%%%%%%%%%%%%%
%%%%%%%%%%%%%%%% fancy header and footer %%%%%%%%%%%%%%
%%%%%%%%%%%%%%%%%%%%%%%%%%%%%%%%%%%%%%%%%%%%%%%%%%%%%%%

\fancyfoot{}
\fancyfoot[RO, LE]{\thepage}
\fancyhead{}
\fancyhead[RO, LE]{\rightmark}
\fancyhead[LO, RE]{\ifnum\thechapter > 0 \leftmark \fi}
\fancypagestyle{plain}{%
\fancyhf{} % clear all header and footer fields
\fancyfoot{}
\fancyhead{}
\fancyfoot[RO, LE]{\thepage}
\renewcommand{\headrulewidth}{0pt}}
	
%%%%%%%%%%%%%%%%%%%%%%%%%%%%%%%%%%%%%%%%%%%%%%%%%%%%%%%
%%%%%%%%%%%%%%%% define colors %%%%%%%%%%%%%%%%%%%%%%%%
%%%%%%%%%%%%%%%%%%%%%%%%%%%%%%%%%%%%%%%%%%%%%%%%%%%%%%%

\definecolor{darkgreen}{rgb}{0,0.6,0}
\definecolor{darkgray}{rgb}{0.1,0.1,0.1}
\definecolor{gray}{rgb}{0.5,0.5,0.5}
\definecolor{purple}{rgb}{0.58,0,0.82}

%%%%%%%%%%%%%%%%%%%%%%%%%%%%%%%%%%%%%%%%%%%%%%%%%%%%%%%
%%%%%%%%%%%%%%%% listings einstellungen %%%%%%%%%%%%%%%
%%%%%%%%%%%%%%%%%%%%%%%%%%%%%%%%%%%%%%%%%%%%%%%%%%%%%%%

\lstset{frame=tb,
  language=Python,
  aboveskip=3mm,
  belowskip=3mm,
  showstringspaces=false,
  columns=flexible,
  basicstyle={\small\ttfamily},
  numbers=none,
  numberstyle=\tiny\color{gray},
  keywordstyle=\color{blue},
  commentstyle=\color{darkgreen},
  stringstyle=\color{purple},
  breaklines=true,
  breakatwhitespace=true,
  tabsize=4
}

% JavaScript erh�lt ein Markup
\lstdefinelanguage{JavaScript}{
keywords={typeof, new, true, false, catch, function, return, null, catch, switch, var, if, in, while, do, else, case, break},
keywordstyle=\color{blue}\bfseries,
ndkeywords={class, export, boolean, throw, implements, import, this},
ndkeywordstyle=\color{darkgray}\bfseries,
identifierstyle=\color{black},
sensitive=false,
comment=[l]{//},
morecomment=[s]{/*}{*/},
commentstyle=\color{gray}\ttfamily,
stringstyle=\color{purple}\ttfamily,
morestring=[b]',
morestring=[b]"
}

%%%%%%%%%%%%%%%%%%%%%%%%%%%%%%%%%%%%%%%%%%%%%%%%%%%%%%%
%%%% damit leere seiten, die vor chapters entstehen, %%
%%%% unbeschriftet sind %%%%%%%%%%%%%%%%%%%%%%%%%%%%%%%
%%%%%%%%%%%%%%%%%%%%%%%%%%%%%%%%%%%%%%%%%%%%%%%%%%%%%%%

\makeatletter
\def\cleardoublepage{\clearpage\if@twoside \ifodd\c@page\else
\hbox{}
\thispagestyle{empty}
\newpage
\if@twocolumn\hbox{}\newpage\fi\fi\fi}
\makeatother

%%%%%%%%%%%%%%%%%%%%%%%%%%%%%%%%%%%%%%%%%%%%%%%%%%%%%%%
%%%%%%%%%%%%%%%%%%%% sonstiges %%%%%%%%%%%%%%%%%%%%%%%%
%%%%%%%%%%%%%%%%%%%%%%%%%%%%%%%%%%%%%%%%%%%%%%%%%%%%%%%

\title{Untersuchung der Distributed Ledger Technologien auf Zukunftstauglichkeit}
\author{Janusz Spatz and Jens Altrock}
\date{\today}



\begin{document}

\begin{titlepage}
	\begin{center}
		\includegraphics[scale=0.85]{hsblogo.png}
		
		\hrulefill
		
		\Large
		\textbf{Bachelor-Thesis}\\
		
		\vspace{0.5cm}
		\normalsize
		zur Erlangung des akademischen Grades\\
		Bachelor of Science (B.Sc.)\\
		
		\vspace{1cm}
		
		\LARGE
		\textbf{\thetitle}
		
		\hrulefill
		
		\vspace{0.5cm}
		\normalsize
		
		\begin{tabular}{ll}
		Fakult�t 4 & Elektrotechnik und Informatik\\
		Erstpr�fer: & Prof. Dr. Lars Braubach\\
		Zweitpr�fer: & Prof. Dr. Martin Hering-Bertram\\
		Abgabetermin: & 15.10.2018
		\end{tabular}
		
		\vspace{1,5cm}
		
		vorgelegt von
		
		\begin{tabular}{cc}
		Jens Altrock & Janusz Spatz\\
		399144 & 399341
		\end{tabular}
		
	\end{center}
\end{titlepage}
\cleardoublepage
\pagenumbering{gobble}
\cleardoublepage
\pagestyle{fancy}
\pagenumbering{roman}

\tableofcontents
\listoffigures
\listoftables

\cleardoublepage
\pagenumbering{arabic}

\chapter{Einleitung}
Seit 2017 besteht ein starker Wachstum des Interesses an Blockchain, wobei Bitcoin als prominenteste Implementation angesehen wird.(QUELLE) Am Sonntag, den 17. Dezember 2017 durchbricht Bitcoin erstmals fast die \$20.000 Marke (\$19,783.21)~\cite{ja:bitcoin20000}(FOOTNOTE). Doch nicht nur Bitcoin hat von dem starken Interesse profitiert, sondern auch viele weitere Kryptow�hrungen sowie andere Blockchain-Implementierungen. Damit war die Technologie in das Bewusstsein der breiten Masse gerutscht und zugleich h�ufen sich Fragen im Bezug auf Zukunftssicherheit und Tauglichkeit der jeweiligen Technologien. Jene Fragen sollen in dieser Bachelorarbeit er�rtert werden.

\section{Kooperationen mit Blockchain/Tangle}
Immer mehr kommerzielle Firmen entdecken das Potenzial von Kryptow�hrungen und weiteren Blockchain-Technologien.(QUELLE) Die Kooperation zwischen dem Automobilhersteller VW und diversen Kryptow�hrungen ist ein aktuelles Beispiel in dem eine Firma Potential in dieser Technologie sieht.
In einer am 8. August, 2018 ver�ffentlichten Kurznachricht auf der Social-Media-Plattform Twitter schreibt die Volkswagen Group:

\begin{quote}
"`Bringing \#blockchain systems to the road: We�re working full steam ahead on making super-safe \#cryptosystems available to our customers. For filling the tank, unlocking your car � and all kinds of other possibilities: \#bitcoin \#ethereum \#iota"'

- Volkswagen Group auf Twitter\footnote{https://twitter.com/VWGroup/status/1027205629436407810}
\end{quote}

In einem Blockeintrag der Volkswagen Group geht hervor, dass sie anhand von Blockchains, die Zuverl�ssigkeit und Sicherheit von Autos erh�hen wollen. Zum Beispiel indem Kilometerz�hler vor Manipulationen gesch�tzt und Hackerangriffe auf selbstfahrende Autos vorgebeugt werden.[2](QUELLE)

Auch weitere Firmen wie Bosch geben in einer Pressemitteilung Ende 2017 bekannt, dass auch sie in IOTA investieren und eng mit ihnen an verschiedenen Fronten zusammenarbeiten werden. [3](QUELLE) Sowie auch Microsoft hat sich die Kryptow�hrungen zu Nutzen gemacht. Die Nutzer k�nnen anhand von Bitcoin k�ufe im Microsoft Store t�tigen [4](QUELLE).

Es ist davon auszugehen, dass viele weitere namhafte Firmen dem Beispiel folgen werden und auch Blockchain/Tangle Technologien in ihren jeweiligen Anwendungsbereichen einbinden werden. So kristallisiert sich die M�glichkeit heraus, dass Blockchains oder Tangles im zuk�nftigen Alltag eine erhebliche Rolle spielen werden, was zum Beispiel Transaktionen und Sicherheit betrifft.

\section{Sicherheit}
Kryptow�hrungen und die genutzte Blockchain steht in dem Punkt Sicherheit 

\section{Kryptographie}
Kryptographie ist ein wichtiger Bestandteil von Distributed Ledger Architekturen. Um die Integrit�t einer Nachricht zu gew�hrleisten, wird eine Kombination aus privaten und �ffentlichen Schl�sseln ben�tigt, die dank Kryptographie erstellt wurden. Die Nachricht wird mit dem privaten Schl�ssel verrechnet und das Produkt ist eine Signatur. Die Nachricht kann dank des �ffentlichen Schl�ssels und der Signatur validiert werden.

\section{Zielsetzung}
Grundlegend soll die Frage gekl�rt werden, wie sich ein, auf Blockchain basierendes, Konzept f�r die Zukunft durchsetzen kann und mit welchen realistischen Aussichten zu rechnen ist. Hierbei ist es notwendig, vorhandene Implementierungen nach Einsatzm�glichkeit und Art der Technologie zu kategorisieren und zu beschreiben.

Um einen genauen Ausblick f�r die Zukunft geben zu k�nnen, muss die Entwicklung, Sicherheit und Usability von den zuvor ausgew�hlten Technologien ausf�hrlich und kritisch untersucht werden. Unterst�tzt wird das Ergebnis der Thesis von einem selbst erstellten Programm, welches mit einer Kryptow�hrung arbeitet. Dieses Programm soll eine m�gliche zuk�nftige Nutzung von Blockchain aufzeigen. Sie soll eine Dienstleistung annehmen k�nnen, und anschlie�end autonom den Dienst mittels einer Kryptow�hrung bezahlen k�nnen.

Die Analyse der Technologien soll einen ausf�hrlichen Entwicklungsstand vermitteln. Dank dieser sollen fundierte Aussagen �ber die Zukunftstauglich

\chapter{Grundlagen}
\section{Raspberry Pi}
Bei einem Raspberry Pi handelt es sich um einen Einplatinencomputer im Kreditkarten-Format. Das Ger�t besitzt unter anderem USB-, Ethernet- und AUX-Eing�nge, sowie einen Steckplatz f�r eine Micro-SD-Karte, die als Speicherort des Betriebssystems und der Daten genutzt wird. Bildinformationen werden an den HDMI-Ausgang gesendet. Als Stromquelle dient der Micro-B-Eingang der mit einer Eingangsspannung von 5 Volt betrieben werden kann. [11]
Urspr�nglich diente die Entwicklung des Raspberry Pi�s daf�r, heranwachsende Leute in Gro�britannien dem Programmieren nahezubringen. [12] Inzwischen ist der Raspberry Pi in verschiedenen Modellen mehr als 17 Millionen mal (Stand: ende 2017) verkauft worden. [13] Damit geht der Raspberry Pi weit �ber den anf�nglichen Zielen hinaus und wird auch von erfahrenen Programmierern f�r vielf�ltige Projekte genutzt.
Der Einplatinencomputer wird mit einem auf Linux basierendem Betriebssystem genutzt. Aufgrund der Architektur des CPUs sind Betriebssysteme wie Windows nicht auf dem Raspberry Pi verf�gbar. Das offizielle Betriebssystem ist das auf Debian (Linux) basierende Raspbian. [17] Es beinhaltet alle Programme die n�tig sind um mit dem Programmieren des Raspberry Pis zu beginnen.
Durch die Faktoren von einer umfangreichen und integrierten Programmieroberfl�che, der simplen Installation und die geringen Kosten machen den Raspberry Pi zu einer idealen Umgebung um prototypische Anwendungen umzusetzen.

\section{GPIO}
Zahlreiche GPIO-Pins befinden sich auf dem Raspberry Pi um eine direkte Kommunikation mit externen Schaltkreisen zu erlauben. GPIO ist eine Abk�rzung f�r General Purpose Input Output, �bersetzt bedeutet dies Ein- und Ausgabe f�r allgemeine Einsatzm�glichkeiten. Daf�r stehen auf dem Raspberry Pi bis zu 40 Pins zur Verf�gung. Die einzelnen Pins k�nnen mittels Software und Hardware an- oder ausgeschaltet werden. Dar�ber hinaus kann ein Wert auch eingelesen oder gesetzt werden. Doch nicht jeder Pin steht f�r GPIO-Zwecke zur Verf�gung. Einige Pins sind f�r andere Funktionen vorbehalten (z.B. Daten�bertragung, Strompotenzial von +5V, +3.3V und 0V). Aus diesem Grund empfiehlt es sich, das Datenblatt �ber die Verteilung der Pins anzugucken.
Die GPIO-Pins erm�glichen es, angeschlossene Schaltkreise zu steuern oder einzulesen. Sie sind nach belieben programmierbar. Daher finden Raspberry Pis oft Verwendung in Projekten wie Heimautomatisierung, Drohnen und Sicherheitssystemen. [14][15][16]

\section{Steckbrett (Breadboard)}
Ein Steckbrett wird genutzt um l�tfrei Komponenten mit den Pins des Raspberry Pis zu verbinden. Mittels Kabeln werden die Pins mit einer Reihe des Bretts verbunden, wodurch an der Reihe eine Spannung angelegt wird. Nun kann an dieser Reihe eine Komponente angesteckt werden. Damit der Schaltkreis geschlossen wird, muss das andere Ende mit einem Pin verbunden werden, der geerdet ist (ein GRND-Pin).
Da es damit m�glich ist, schnell neue Komponenten anzuschlie�en und den Schaltkreis nach belieben zu ver�ndern ist es gut geeignet um prototypisch Schaltkreise zu erstellen.

\section{Hash Tree (Merkle Tree)}
Im Jahr 1979 patentierte Ralph Merkle das Prinzip des Hash Trees. Dieses ist sp�ter auch bekannt als �Merkle Tree�.[5] Der Nutzen des Hash Trees wird bei der effizienten Verifizierung von Daten deutlich. Anf�nglich werden Dateien, in der Regel Paare aus zwei Dateien, miteinander gehasht. Der daraus resultierende Kind-Hash wird mit dem Kind-Hash anderer Dateien zu einem gemeinsam Hash verrechnet. So entsteht nach und nach ein Baum aus diversen Hashes die schlie�lich einen Top-Hash ergeben. Mithilfe dieses Top-Hashes ist es m�glich die Unversehrtheit jeder anf�nglichen Datei zu gew�hrleisten. 

\section{Proof-of-Work (Hashcash)}
Das Proof-of-Work Konzept wurde anf�nglich gegen denial-of-service Attacken oder f�r die Abwehr von Spam beim Email Verkehr gedacht. Der Erfinder Adam Back nannte dieses System auch �Hashcash�.[6] Damit eine Person, die dieses Konzept nutzt, beispielsweise eine Email versenden kann, muss sie vorher einen kleinen Aufwand betreiben. Dies k�nnte eine einfache mathematische Formel sein. Bei erfolgreicher Berechnung ist die Email freigegeben. Da eine Vielzahl dieser kleinen Aufw�nde sich zu einem gro�em Rechenaufwand aufstaut und da die Komplexit�t der Aufw�nde mit jeder erfolgreichen Abarbeitung steigt, wird in der Schlussfolgerung das schnelle Versenden vieler aufeinander folgender Emails verhindert.

\section{Internet of Things (IoT)}
Das Internet of Things (deutsch: Internet der Dinge) beschreibt die Verkn�pfung von Ger�ten und Sensoren die vorher eigenst�ndig und isoliert waren mit dem Internet. Damit ist es ihnen m�glich Daten untereinander auszutauschen indem sie ihre eigenen Daten �bertragen und externe Daten anfordern. So lassen sich die Ger�te �ber das Internet gleichzeitig von Menschen und auch anderen Maschinen fernsteuern. Mit diesem Begriff wird die Idee eingeleitet, dass die Nutzer des Internets nicht ausschlie�lich Menschen sind, sondern zunehmend auch �Dinge� (Things) �ber diesem Kommunizieren.
Im privaten Gebrauch werden als Beispiele allt�gliche Ger�te aufgef�hrt, die durch die Vernetzung im Internet mit anderen Ger�ten das Leben des Nutzers komfortabler machen oder auf sonstige Weise positiv beeinflussen. Ein Beispiel w�re die Hausautomation mit Sicherheitskameras, die, wenn etwas verd�chtiges ermittelt wird, sofort den Hauseigent�mer �ber das Internet warnen und das Videomaterial live auf das Smartphone �bertragen wird. [18]
Das Internet der Dinge l�sst sich auch im industriellen Bereich anwenden. So lassen sich Herstellungsprozesse durch die Vernetzung von Sensoren, Anlagen und Maschinen so automatisieren, dass der Prozess effizienter wird und menschliche Hilfe immer obsoleter wird. [18]

\section{Industrie 4.0}
In diesem Zusammenhang f�llt oft der Begriff Industrie 4.0. [19] Damit ist eine vierte industrielle Revolution gemeint. Kurz beschrieben geht die erste industrielle Revolution mit der Mechanisierung einher, die zweite mit der Massenfertigung von Produkten mittels Flie�b�ndern und Fabriken, und die dritte mit der Automatisierung von Arbeitsschritten mittels Maschinen. [20] Insofern handelt es sich bei Industrie 4.0 um die Vernetzung der automatisierten Arbeitsschritte, sodass sie autonom miteinander kommunizieren k�nnen und sich dem n�chsten Arbeitsschritt anpassen k�nnen.

\section{Trits, Trytes \& Tryte-Alphabet}
Das Tern�re System ist eine Alternative zu dem Bin�ren System, mit dem Unterschied dass das Tern�re System die Basis 3 hat. Ein Trit kann drei verschiedene Werte annehmen. Es ist die kleinste Einheit in dem Tern�ren System. Die Werte belaufen sich auf -1, 0 oder 1. Drei Trits ergeben ein Tryte. Ein Tryte kann $3^3 = 27$ Zust�nde annehmen. Um die Lesbarkeit einer l�ngeren tern�ren Zahl zu verbessern wurde von der IOTA-Foundation das sogenannte "`Tryte-Alphabet"' erstellt. Es besteht aus den 26 Zeichen des Alphabetes in Gro�buchstaben (A-Z) und der Zahl $9$. Jedes Zeichen repr�sentiert den Wert eines Tryte.

\begin{table}[h]
\centering
	\begin{tabular}{c|c}
		\begin{tabular}{P{1.8cm}|P{1.8cm}|P{1.8cm}}
			Tryte & Dezimal & Buchstabe \\
			\hline
			0, 0, 0 & 0 & 9 \\
			\hline
			1, 0, 0 & 1 & A \\
			\hline
			-1, 1, 0 & 2 & B \\
			\hline
			0, 1, 0 & 3 & C \\
			\hline
			1, 1, 0 & 4 & D \\
			\hline
			-1, -1, 1 & 5 & E \\
			\hline
			0, -1, 1 & 6 & F \\
			\hline
			1, -1, 1 & 7 & G \\
			\hline
			-1, 0, 1 & 8 & H \\
			\hline
			0, 0, 1 & 9 & I \\
			\hline
			1, 0, 1 & 10 & J \\
			\hline
			-1, 1, 1 & 11 & K \\
			\hline
			0, 1, 1 & 12 & L \\
			\hline
			1, 1, 1 & 13 & M \\
		\end{tabular}&
		\begin{tabular}{P{1.8cm}|P{1.8cm}|P{1.8cm}}
			Tryte & Dezimal & Buchstabe \\
			\hline
				&	&	\\
			\hline
			-1, -1, -1 & -13 & N \\
			\hline
			0, -1, -1 & -12 & O \\
			\hline
			1, -1, -1 & -11 & P \\
			\hline
			-1, 0, -1 & -10 & Q \\
			\hline
			0, 0, -1 & -9 & R \\
			\hline
			1, 0, -1 & -8 & S \\
			\hline
			-1, 1, -1 & -7 & T \\
			\hline
			0, 1, -1 & -6 & U \\
			\hline
			1, 1, -1 & -5 & V \\
			\hline
			-1, -1, 0 & -4 & W \\
			\hline
			0, -1, 0 & -3 & X \\
			\hline
			1, -1, 0 & -2 & Y \\
			\hline
			-1, 0, 0 & -1 & Z \\
		\end{tabular}
	\end{tabular}
\caption{Das Tryte Alphabet}
\label{tab:comparison}
\end{table}

\section{Distributed Ledger}
Eine Distributed Ledger (w�rtlich "`verteiltes Kontobuch"') Technologie (kurz DLT) kann sich wie ein �ffentlich einsehbares Kontobuch vorgestellt werden. Bei handels�blichen zentralen Transaktionen kann der Zahlende sich seines Handels zwar bewusst sein, jedoch nicht �ber den Status des Eingangs bei dem Zahlungsempf�ngers. Des weiteren hat der Zahlungsempf�nger keine Einsicht in die Verf�gbarkeit des Geldes oder des zu transferierenden Gegenstandes des Zahlenden. Der Handelsgegenstand kann in einer weiteren Transaktion bereits reserviert sein. Bei zentralisierten Handelsauftr�gen garantiert eine dritte Partei die Richtigkeit des Handels. Dabei wird das Vertrauen jedoch in diese vorausgesetzt. Eine DLT l�st dieses Problem, indem die Richtigkeit eines Handels nicht durch zentrale Parteien versichert wird, sondern durch systemimmanente Prozesse. Dar�ber hinaus sind die get�tigten Handelsauftr�ge anonymisiert �ffentlich einsehbar und nicht auf die Handelsparteien zur�ckzuf�hren. Die Richtigkeit basiert nicht weiter auf Vertrauen.

Eine DLT zeichnet den gesamten Handelsverlauf auf. Jeder Handel kann auf den Ursprung zur�ckverfolgt werden. Dem Handelspartner ist es ein leichtes, den Verlauf eines Guts oder Geldes nachzuweisen, indem er die Historie dessen zur�ckverfolgt.\footnote{Daten bezogen von: https://www.bafin.de/SharedDocs/Veroeffentlichungen/\\DE/Fachartikel/2016/fa\_bj\_1602\_blockchain.html}
\chapter{Analyse der DLT}
In der folgenden Analyse werden zwei Distributed Ledger Architekturen auf ihre Funktion, Sicherheit und Nutzen untersucht. Die gewonnen Informationen werden zum Ende dieses Kapitels gegen�bergestellt, um eine fundierte Aussage �ber die Zukunftstauglichkeit zu treffen. Bei den zwei Technologien handelt es sich zum einen um die popul�re Kryptow�hrung Bitcoin, welche erstmalig das Prinzip der Blockchain umsetzt und neue M�glichkeiten der dezentralen Datenverarbeitung realisiert. Als weitere Distributed Ledger Technologie wird IOTA herangezogen. IOTA leitet sich von "`Internet of Things (IoT)"' ab und stellt eine L�sung f�r das Skalierungsproblem von gro�en Datenmengen bereit. Beide Technologien basieren auf dem Konzept dezentral die Richtigkeit von Daten zu garantieren, indem ein Konsens gefunden wird. Die jeweilige Realisierung dieser Konzepte unterscheidet sich jedoch fundamental. Dank ihrer Unterschiede sind die beiden Technologien besonders geeignet um eine begr�ndete Aussage �ber die allgemeine Zukunftstauglichkeit der Distributed Ledger Architektur zu treffen.

\chapter{Funktion}
Die Funktion von Distributed Ledger Architekturen wird �ber verschiedene Kryptographien und Techniken erm�glicht. Das kommende Kapitel stellt die Funktion dieser Konzepte anhand der jeweiligen Umsetzung in Bitcoin und IOTA detailliert dar.

\section{Bitcoin}
Bitcoin ist ein Distributed Ledger Protokoll das erstmalig Geldtransfer �ber eine dezentrale, anonyme �bereinstimmung der Richtigkeit verf�gt. Diesen Konsens erreicht es �ber die implementierte Blockchain, die wie ein global einsichtiges Kassenbuch funktioniert. Die Blockchain kann auf verschiedene Art und Weise genutzt und entwickelt werden. Sie erm�glicht es �ber das Internet Werte auszutauschen ohne von einem Mittelsmann abh�ngig zu sein~\cite{je:whatIsBitcoin}. Durch kryptographische Techniken sind Daten die in die Blockchain gelangen nur durch sehr gro�en Aufwand manipulierbar, was sie f�r das Speichern von sensiblen Daten attraktiv macht. Im Folgenden wird die grundlegende Funktion der Blockchain sowie dessen Einsatz bei Bitcoin sachlich untersucht.

\subsection{Allgemeine Funktion}
F�r die gegenw�rtige Funktion von Bitcoin und dessen Blockchain sind zwei elementare Techniken notwendig. Zum Einen ist das die Public-Key-Kryptographie zum anderen die kryptographischen Hashfunktionen.

Bei der Public-Key-Kryptographie resp. digitalen Signatur erstellt der Sender einer Nachricht ein Schl�sselpaar bestehend aus einem privaten sowie einem �ffentlichen Schl�ssel. Die beiden Schl�ssel haben zwei erg�nzende Funktionen. Der Autor der Nachricht verwendet den privaten Schl�ssel um diesen mit seinen Informationen zu verbinden und dadurch zu signieren. Die signierten Daten werden gemeinsam mit dem �ffentlichen Schl�ssel an den Empf�nger weitergegeben. Dank des �ffentlichen Schl�ssel ist es dem Empf�nger m�glich die Daten zu authentifizieren. Des Weiteren ist durch die Verkn�pfung der Daten mit dem privaten Schl�ssel die inhaltliche Integrit�t vor Manipulation gesch�tzt.

Durch kryptographische Hashfunktionen entstehen aus Zeichenketten mit variabler L�nge Zeichenketten fester L�nge. Diese Zeichenketten sind �deterministisch�. Die gleichen Eingangsdaten werden nach der Hash-Berechnung immer den gleichen Hash verursachen. Ver�nderte Eingangsdaten f�hren zu einem stark abweichenden Hash.

Au�erdem gibt es drei weitere nennenswerte Eigenschaften von Hashfunktionen. Zum Einen ist es nahezu unm�glich durch den Hash die anf�nglichen Daten wiederherzustellen. Des Weiteren ist es nahezu unm�glich mit abweichenden Eingangsdaten denselben Hash zu generieren. Zuletzt ist es nahezu unm�glich zwei verschiedene Eingangsdaten zu finden aus denen sich derselbe Hashwert ergibt.~\cite{je:whatIsHash}

\subsection{Privater \& �ffentlicher Schl�ssel}\label{ECC}
Der private Schl�ssel in Bitcoin kennzeichnet den Besitz der �bertragenden BTCs (Geldeinheit). Sollte der Schl�ssel verloren gehen, ist es nicht m�glich die BTCs, die dem privaten Schl�ssel zugeordnet sind, wiederherzustellen. Der �ffentliche Schl�ssel wird f�r die Verifikation einer Transaktion ben�tigt, da er durch den privaten Schl�ssel generiert wurde. Man kann durch den �ffentlichen Schl�ssel, nach heutigem Stand, nicht den privaten Schl�ssel erraten. Dieses Ziel wird durch die folgende Kryptographie erreicht:

\paragraph{Endliche Felder}
Ein endliches Feld ist eine Gruppe aus Zahlen, das, wie der Name suggeriert, endlich ist. Ein endliches Feld, das mit Zahlen des Modulo P gebildet wird, bei dem P eine Primzahl ist, f�hrt zu n�tzlichen Funktionen in der Kryptographie.

\paragraph{Elliptische Kurven}
In der Mathematik sind elliptische Kurven aufgrund ihrer Eigenschaft, mathematische Gruppen zu sein, interessant. Eine elliptische Kurve wird nach der folgenden Formel gebildet:

\begin{equation*}
	\{(x,y) \in \mathbb{R}^2 \mid y^2 = x^3 + ax + b,\, 4a^3 + 27b^2 \neq 0 \} \cup \{ 0 \}
\end{equation*}

(PICTURE)
F�r jeden Punkt der elliptischen Kurve gelten die Gesetze der Kommutativit�t, Assoziativit�t und Distributivit�t. Dar�ber hinaus gelten weitere, f�r elliptische Kurven spezifische, Regeln:

\begin{description}
	\item[-] Die Inverse -P eines Punktes P kann durch das Spiegeln des Punktes P an der x-Achse bestimmt werden.
	\item[-] Wenn zwei Punkte (P, Q) einer elliptischen Kurve bekannt sind, kann immer auch ein dritter Punkt (R) bestimmt werden. Die Addition der drei Punkte ergibt in jedem Fall $ P + Q + R = 0 $. Insofern w�re die Rechnung der Punkte zur Bestimmung von Punkt R: $ P + Q = -R $. Punkt -R ist die Inverse von Punk R gespiegelt an der x-Achse.
\end{description}

Eine Formel die diese beiden Konzepte, der elliptischen Kurven und der endlichen Felder, vereint sieht wie folgt aus:

\begin{equation*}
	\{(x,y) \in \mathbb{R}^2 \mid y^2 \equiv x^3 + ax + b\, (mod\, p),\, 4a^3 + 27b^2 \not\equiv 0\, (mod\, p)\} \cup \{ 0 \}
\end{equation*}
Gruppe aus den Reellen Zahlen der elliptischen Kurven des modulo P \cite{je:ellipticCurve}

Dank des Moduls p befindet sich die Gleichung in einem endlichen Feld. Eine elliptische Kurve die sich in einem endlichen Feld befindet ist weiterhin in eine mathematische Gruppe. Alle Formeln k�nnen wie zuvor beschrieben angewandt werden.

Tritt der Fall ein, dass Punkt P sich wie eine Tangente zu der elliptischen Kurve verh�lt, wird das sogenannte Prinzip der Punktdopplung genutzt: 
\begin{equation*}
	P + Q = -R;\: Q = P \rightarrow 2P = -R
\end{equation*}

Es ist m�glich den Punkt P um eine beliebige Anzahl x zu skalieren um einen Punkt R zu erreichen.

\begin{equation*}
	xP = R
\end{equation*}

Da sich Punkt P in einem endlichen Feld befindet, wird durch ein, abh�ngig des Punktes P, gew�hlter bestimmter Skalar x daf�r Sorgen, dass das Produkt auf die Ausgangsposition P verweist. Dieses Prinzip l�sst sich gut durch den Zeiger einer Uhr verdeutlichen. Die Ausgangsuhrzeit ist 3 Uhr. Der Zeiger wandert in drei Stunden Schritten voran. Nach vier Schritten zeigt der Stunden Zeiger erneut auf die Ausgangsuhrzeit 3 Uhr.

Es entsteht eine Untergruppe die durch den Punkt P definiert ist. Die Ordnung n dieser Gruppe P wird so gew�hlt dass nP = 0 ist. In dem Beispiel der Uhrzeit entspricht die Ordnung n = 4 denn $4(3)\: mod(12) = 0$.

Bitcoin nutzt folgende Werte f�r die elliptische Kurven Kryptographie:

\begin{description}
	\item[1.] Der prim�re Wert des Moduls zum Bestimmen des endlichen Raumes entspricht: $2^{256} - 2^{32} - 2^9 - 2^8 - 2^7 - 2^6 - 2^4 - 1 \rightarrow$ FFFFFFFF FFFFFFFF FFFFFFFF FFFFFFFF FFFFFFFF FFFFFFFF FFFFFFFE FFFFFC2F (hex).
	\item[2.] Die elliptische Kurve wird mit den Parametern $a=0$ und $b=7$ gebildet.
	\item[3.] Die Basis der Untergruppe P betr�gt in hexadezimal: 04 79BE667E F9DCBBAC 55A06295 CE870B07 029BFCDB 2DCE28D9 59F2815B 16F81798 483ADA77 26A3C465 5DA4FBFC 0E1108A8 FD17B448 A6855419 9C47D08F FB10D4B8 (hex).
	\item[4.] Die Ordnung n der Untergruppe P betr�gt in hexadezimal: FFFFFFFF FFFFFFFF FFFFFFFF FFFFFFFE BAAEDCE6 AF48A03B BFD25E8C D0364141 (hex).
\end{description}

Der private Schl�ssel ist eine Zahl zwischen 1 und der angegebenen Ordnung. Um den �ffentlichen Schl�ssel zu errechnen, wird der private Schl�ssel mit dem Basispunkt der Untergruppe P multipliziert. Das Ergebnis $mod$ des prim�ren Wertes des Moduls ergibt den �ffentlichen Schl�ssel. Dieser entspricht einem Wert der durch nahezu unendlich M�glichkeiten erreicht werden kann. Nur der Besitzer des privaten Schl�ssels kann diesen Wert gezielt reproduzieren.~\cite{je:ellipticCurve}

\subsection{Transaktionen}
Um eine Transaktion im Bitcoin Netzwerk zu veranschaulichen wird ein fiktives Beispiel herangezogen. Alice m�chte an Bob zwei Bitcoins (BTC) versenden. Es existieren jedoch keine Konten oder Kontost�nde, sondern ausschlie�lich die �ffentliche Liste aller jemals get�tigten Transaktionen, also die Bitcoin - Blockchain selbst, sowie die Hashwerte der einzelnen �ffentlichen Schl�ssel denen die existierenden Transaktionen zugeordnet werden. Aufgrund der vergangenen Transaktionen kann der Wertbestand an BTCs des zugeh�rigen privaten Schl�ssels ermittelt werden. Mithilfe einer digitalen Software (Wallet) kann dieser Wertbestand des privaten Schl�ssels eingesehen und neue Transaktionen eingereicht werden. Um eine Transaktion zu veranlassen werden mithilfe des privaten Schl�ssels die zur�ckliegenden eingegangenen Transaktionen signiert und zusammen mit der ausgehenden Transaktion an das Bitcoin Netzwerk versandt. Dies entspricht in dem anf�nglichen Beispiel einer Auflistung aller Transaktionen in denen Alice involviert war, sowie Alice� Intention zwei BTCs an den �ffentlichen Schl�ssel von Bob zu senden. 

Diese Nachricht gelangt an einen Netzknoten der �ber verschiedene Ressourcen verf�gt. Neben einer aktuellen und vollst�ndigen Kopie der Blockchain, einen Cachespeicher ("`Unspent Transaction Output"' UTXO), der Transaktionswerte der Blockchain enth�lt die noch nicht f�r neue Transaktionen verwendet wurden, existiert noch eine Datenbank mit unbest�tigten Transaktionen. In diesem Netzknoten wird die Legitimit�t von Alice� Transaktion �berpr�ft, indem zum einen ihre Liquidit�t aufgrund ihrer bisherigen Transaktionen, sowie gleichzeitig ihre Signatur anhand des �ffentlichen Schl�ssels, best�tigt wird. Mithilfe der UTXO wird au�erdem gepr�ft, ob die genutzten BTCs nicht bereits anderweitig reserviert wurden. Wenn dies fehlerfrei geschieht, wird die unbest�tigte Transaktion in die daf�r eingerichtete Datenbank aufgenommen. Der Netzknoten meldet diese eingereichte Transaktion an m�glichst viele weitere Netzknoten weiter, die wiederum die Daten nach dem erkl�rten Schema �berpr�fen und anschlie�end in die Datenbank der unbest�tigten Transaktionen aufnehmen.~\cite{je:blockchainBasics}

\subsection{Mining}
Das Bitcoin-Netzwerk verf�gt �ber zwei Arten von Netzwerkknoten. Einer ist ausschlie�lich f�r das zuvor genannte Verifizieren und Speichern der eingehenden Transaktionen sowie dessen Weiterleitung an weitere Netzknoten zust�ndig. Der zweite ist dar�ber hinaus bef�higt neue Bl�cke in der Blockchain zu speichern. Dieser sogenannte �Mining-Netzknoten� speichert zun�chst unbest�tigte Transaktionen in einem Block zusammen. Dieser Block besitzt zus�tzlich einen Blockheader, der f�r die kommende Abspeicherung in der Blockchain essentiell ist.

Es ergeben sich ab diesem Zeitpunkt einige Problematiken. Aufgrund von Verz�gerungen im Netzwerk oder m�glichen Ausf�llen sind einige Netzknoten aktueller. Dies f�hrt dazu, dass verschiedene Transaktionen, welche sich voneinander unterscheiden, in den einzelnen Netzknoten existieren. Au�erdem k�nnen b�sartige Absender gef�lschte Transaktionen weiterleiten, welche dieselben Wert-Ressourcen aufweisen. Bitcoin verwendet f�r diese Problematiken die Proof-Of-Work (PoW) Technologie. In der Regel wird diese Technik f�r das Verhindern von missbr�uchlicher Benutzung von Diensten eingesetzt. Damit ein Dienst genutzt werden kann muss bei dieser Technik ein gewisser Aufwand erbracht werden. Im Falle der Bitcoin-Blockchain muss gem�� dieses Schemas eine rechenintensive Aufgabe gel�st werden. Der Block-Header ergibt einen gewissen Hashwert. Dieser muss solange manipuliert werden bis das Ergebnis unterhalb eines gewissen Zielwertes liegt. Diese partielle Hashinversion beruht auf dem Hashcash-Prinzip von Adam Back (siehe Grundlagen Kapitel \ref{basic_HashCash} auf Seite \pageref{basic_HashCash} ). Der Hash entspringt dem Block-Header und wird aus einer Referenz zum vorherigen Block, einem Zeitstempel, dem Zielwert des Hash-R�tsels, dem Top Hash eines Merkle-Trees, welcher alle Transaktionen durch aufeinander aufbauende Hashes zusammenfasst, sowie einer variablen Zeichenfolge (Nonce) errechnet. Die Nonce wird so oft ge�ndert bis der Hashwert unterhalb des Zielwertes liegt. In anderen Worten muss dieser mit einer gewissen Anzahl von Nullen beginnen. Der erste Mining-Netzknoten der dieses R�tsel l�st versendet seinen Block an das Netzwerk. Dort wird der Block ebenfalls auf Validit�t gepr�ft und bei Erfolg in die eigene Blockchain integriert.

In 1,69\% der F�lle, finden zwei Netzknoten zu nahezu der gleichen Zeit eine L�sung des Hash-R�tsels und versenden ihre jeweiligen Bl�cke. In diesem Fall erhalten unabh�ngige Netzknoten verschieden Versionen der Blockchain. Infolgedessen teilen sich die Netzknoten auf. Dieses Ph�nomen ist als Gabelung bekannt. Die jeweiligen Netzknoten arbeiten weiter auf Grundlage ihrer bekannten Blockchain bis zu dem Moment bei dem sie �ber eine l�ngere Blockchain Version informiert werden. Tritt dieser Fall ein, wird die bekannte Blockchain durch die l�ngere ausgetauscht. Sollte die l�ngere Version einige Transaktionen der zuvor bekannten Blockchain nicht enthalten, werden diese wieder in die Datenbank der nicht best�tigten Transaktionen aufgenommen. Momentan wird ca. alle zehn Minuten ein neuer Block in die Blockchain aufgenommen. Damit sich diese Zeitspanne weiter in einem angemessenen Rahmen befindet, wird stetig die Komplexit�t des Hash-R�tsels angepasst. F�r das Finden der L�sung eines Hash-R�tsels und das erfolgreiche Integrieren eines neuen Blockes in der Blockchain, erh�lt der Mining-Netzknoten eine gewisse Menge an BTCs, wodurch neue BTCs erzeugt werden.~\cite{je:blockchainBasics}

\section{IOTA}
IOTA ist ein Distributed Ledger Protokoll, welches eine grenzenlose Skalierbarkeit des eigenen Netzwerkes zul�sst. Dies wird durch die Eigenschaft erm�glicht, das der Benutzer und der Miner nicht l�nger voneinander getrennt sind. In IOTA wird das Prinzip der durch Bitcoin bekannten Blockchain umgedacht. Transaktionen werden nicht l�nger in einem Block gespeichert, sondern �ber ein Netz aus Transaktionen verteilt. In diesem Netz best�tigen sich die jeweiligen Transaktionen gegenseitig ihre Richtigkeit. Dieses Netz wird Tangle genannt und kann sich wie ein "`Directed Acyclic Graph"' vorgestellt werden. Die folgende Untersuchung liefert einen sachlichen Einblick in die Funktion IOTAs und der implementierten Tangle.

\subsection{IOTA Seed}
Der IOTA Seed ist essentiell wichtig f�r die Funktion von IOTA. Er fungiert als "`Kontonummer"' und verweist auf die Menge an IOTAs die dem Konto zugeordnet werden (das IOTA-Wallet). Wenn der Besitzer des Seeds eine Transaktion durch die IOTA-Tangle veranlassen m�chte, wird mit Hilfe von kryptographischen Hashfunktionen eine Adresse aus dem Seed erstellt. Diese Adresse kann nun genutzt werden um IOTAs zu empfangen oder, sollte sie bereits einen Wert besitzen, zu versenden. Adressen k�nnen nur durch den Seed generiert werden und es ist (technisch) unm�glich von der Adresse den Seed wiederherzustellen. Ein Seed besteht aus 81 Trytes die dank des von der IOTA-Foundation zur Verf�gung gestellten "`Tryte-Alphabet"' durch die 26 Gro�buchstaben A-Z des Alphabets und der Zahl 9 dargestellt werden. \footnote{Diese Vorgaben sind von der IOTA Foundation ver�ffentlicht: https://iota.readme.io/docs/seeds-private-keys-and-accounts}

\subsection{Tangle}\label{grundlagen:tangle}
Bei Tangle handelt es sich um ein Distributed Ledger Software Protokoll, welches sich grundlegend von Blockchain unterscheidet. Die Tangle-Technologie wird von den Entwicklern als n�chste evolution�re Weiterentwicklung der Blockchain beschrieben ~\cite{je:theTangle}. Die Technologie nimmt sich den Schw�chen der Blockchain an und integriert L�sungen f�r diese. Konzeptionell bedeutet dies, dass die heterogene Unterscheidung von Minern und Usern in Blockchains homogenisiert werden soll, indem Transaktionen von Nutzern verifiziert werden sollen, die selber eine Transaktion veranlassen m�chten. So l�st sich das Problem der Skalierung und zugleich fallen die Transaktionsgeb�hren weg. Letzteres ist m�glich, da es zwingend notwendig ist, andere Transaktionen zu verifizieren, falls eine eigene Transaktion aufgegeben werden soll. So "`zahlt"' der Nutzer die Geb�hren mit Rechenleistung.

Tangle wurde von der IOTA-Foundation entwickelt und kommt in der gleichnamigen Kryptow�hrung IOTA zum Einsatz.

Anders als bei Blockchain ist es nicht m�glich, neue Tokens zu "`minen"' (bedeutet: herzustellen). Es besteht seit der Entwicklung ein fester Betrag an verf�gbaren Tokens von genau 2.779.530.283 MIOTA (1 MIOTA = 1.000.000 IOTA) \footnote{Daten bezogen von: https://coinmarketcap.com/currencies/iota/}. Die Tangle richtet sich nach einem Mathematischen Konzept, dem Directed Acyclic Graph. Mit diesem Konzept speichert IOTA die Transaktionen.
Eine Transaktion muss von einem Node aufgegeben werden. Sobald eine Transaktion nicht verifiziert werden kann, kann die urspr�ngliche Transaktion nicht ausgehen.~\cite{je:theTangle}

\subsection{Directed Acyclic Graph}
Die IOTA-Tangle ist aufgebaut in Form eines "`Directed Acyclic Graph"' (DAG) was in diesem Kontext bedeutet, dass sie sich in eine Richtung bewegt und niemals kreisf�rmig ist. Damit eine neue nicht verifizierte Transaktion in die Tangle aufgenommen werden kann, ist neben dem Proof-of-Work eine Verifikation von zwei ebenfalls nicht verifizierten Transaktionen notwendig. Diese beiden Transaktionen werden in dem Transaktions-Bundle abgespeichert und sind dementsprechend referenziert. Da nur nicht verifizierte Transaktionen in diesem Prinzip verwendet werden, entspricht der entstehende Graph eines DAG. \footnote{Daten bezogen von: https://www.forbes.com/sites/shermanlee/2018/01/22/explaining-directed-acylic-graph-dag-the-real-blockchain-3-0}

\subsection{Transaktion}
Eine Transaktion in IOTA besteht aus mehreren Hashes und Werten, die jeweils untereinander eine verschiedene Aufgabe erf�llen. In den folgenden Abschnitten werden diese detailliert aufgef�hrt.

\paragraph{signatureMessageFragment}
Sollte die Transaktion eine ausgehende Zahlung sein, wird dieses Feld ben�tigt um die Signatur des privaten Schl�ssels zu enthalten. Die L�nge der Signatur ist davon abh�ngig, mit welcher Sicherheitsstufe diese Transaktion gehasht wird. Sollte die Sicherheitsstufe 2 oder 3 sein, wird eine weitere wertlose Transaktion ben�tigt. Diese speichert in dem noch freien Signature Message Fragment den Rest der Signatur der vorangehenden ausgehenden Zahlung.

Sollte jedoch die Signatur des privaten Schl�ssels nicht erforderlich sein, verbleibt dieses Feld leer und kann f�r das �bertragen einer Nachricht genutzt werden.
Diesem Feld werden 2187 Trytes reserviert.

\paragraph{hash}
In diesem Feld wird der "`transaction Hash"' nach dem Finden der "`nonce"' und dem Proof-of-Work abgespeichert. Die L�nge betr�gt 81 Trytes.

\paragraph{address}
Die Adresse der Transaktion wird erneut f�r verschiedene Aufgaben verwendet, die sich aus der Art der Transaktionen definieren. Sollte eine Zahlung durchgef�hrt werden, wird in diesem Feld die Adresse des Empf�ngers angegeben. Handelt es sich jedoch um eine urspr�ngliche Geldeingangs-Adresse so ist hier eine Adresse aufgef�hrt die aus einem private Key des Besitzer's Seed erstellt wurde.
F�r dieses Feld sind 81 Trytes reserviert.

\paragraph{value}
Der Wert einer Transaktion definiert dessen Art. Sollte dieser Wert positiv sein, handelt es sich um eine zahlende Transaktion (Output). In dem Feld "`Address"' wird sich in diesem Fall die Adresse des Empf�ngers befinden und das Feld "`Signature Message Fragment"' ist entweder leer oder beinhaltet eine benutzerspezifische Nachricht.

Sollte der Wert negativ sein handelt es sich bei der Transaktion um eine empfangene Transaktion (Input). Das Feld "`Address"' enth�lt entsprechend eine Adresse die dank des Seeds und einem daraus entstandenen private Key generiert wurde. Eine eingehende Transaktion enth�lt einen negativen Wert, um den positiven Wert einer ausgehende Zahlung im Bundle zu rechtfertigen. Zwangsl�ufig wurde diese Adresse zuvor von einer fremden IOTA-Transaktion als Empf�nger Adresse genutzt.

Um den Besitz des entsprechenden privaten Schl�ssels zu beweisen, wird in dem "`Signature Message Fragment"' die Signatur durch den privaten Schl�ssel abgespeichert. Sollte die Sicherheitsstufe gr��er als 1 sein, werden weitere Transaktionen ben�tigt, um die gesamte Signatur zu speichern.

Tr�gt dieses Feld keinen Wert und entspricht 0, handelt es sich bei der Transaktion entweder um das Versenden einer einfachen Nachricht an eine Empf�ngeradresse oder wird genutzt um die Signatur einer vorangehenden Transaktion zu speichern.

F�r dieses Feld wurden 27 Trytes reserviert.

\paragraph{obsoleteTag}
Der obsolete Tag enth�lt einen vom Benutzer definierten Tag. Dieser k�nnte in zuk�nftigen IOTA - Iterationen entfernt werden.
Die reservierte L�nge dieses Feldes betr�gt 27 Trytes.

\paragraph{timestamp}
Der Zeitpunkt ist nicht verpflichtend und kann ausgelassen werden.
Ihm werden 9 Trytes reserviert.

\paragraph{currentIndex}
Dieses Feld zeigt die momentane Position im Bundle.
Ihm werden ebenfalls 9 Trytes reserviert.

\paragraph{lastIndex}
Dieses Feld f�hrt den Index der letzten Transaktion des umfassenden Bundles auf und liefert dementsprechend die L�nge dessen.
Reserviert werden erneut 9 Trytes.

\paragraph{bundle}
Dieses Feld enth�lt den Hash des umfassenden Bundles. Er wird genutzt um alle Transaktionen dieses Bundles zu gruppieren. Jede Transaktion in einem Bundle enth�lt in diesem Feld denselben entsprechenden Bundle Hash. Das Feld umfasst 81 Trytes L�nge.

\paragraph{branch- \& trunkTransaction}
Diese beiden Felder sind jeweils 81 Trytes lang und enthalten zwei zuf�llig gew�hlte, nicht verifizierte Transaktionen aus der Tangle. Nicht verifizierte Transaktionen im Tangle werden "'Tips"' genannt. Jede Transaktion ist verpflichtet, zwei zuf�llig gew�hlte Tips in der Tangle zu verifizieren, um selbst als Tip aufgenommen zu werden.

\paragraph{tag}
Dieser Tag wird vom Benutzer gew�hlt und kann frei vergeben werden. Er kann die Suche einer Transaktion im Tangle unterst�tzen. Seine L�nge in der Transaktion betr�gt 27 Trytes.

\paragraph{attachmentTimestamp}
Dieses neun Trytes gro�e Feld beinhaltet den Zeitpunkt direkt nachdem der Proof-of-Work durchgef�hrt wurde. 

\paragraph{attachmentTimestampLowerBound \& attachmentTimestampUpperBound}
Diese beiden Felder zeigen ein Intervall auf in der die Aufnahme in die Tangle geschah. Sie sind jeweils 9 Trytes gro�.

\paragraph{nonce}
Die nonce ist ein besonders wichtiger Teil einer Transaktion, denn sie wird ben�tigt um den Proof-of-Work durchzuf�hren. Die L�nge dieses Feldes betr�gt 27 Trytes.

Rechnet man die L�ngen aller Felder zusammen, so erh�lt man die gesamte L�nge von 2673 Trytes. Dies ist die von der IOTA-Foundation vorgegebene L�nge die eine Transaktion haben soll. Die Art einer Transaktion bestimmt sich durch den Wert. Sollte eine Transaktion einen negativen Wert haben, so handelt es sich dabei um eine "`Input"' Transaktion. Bei einem positiven Wert, wird Balance an eine "`Output"' Adresse geschickt. Viele Felder tragen eine wichtige Aufgabe f�r die Aufnahme in der Tangle. So wird das Feld "`signatureMessageFragment"` entweder f�r benutzerspezifische Nachrichten genutzt oder, falls ben�tigt, f�r die Signatur.\footnote{Diese Vorgaben sind von der IOTA Foundation ver�ffentlicht: https://iota.readme.io/docs/the-anatomy-of-a-transaction}

\subsection{Proof-of-Work}
Um Transaktionen in die Tangle zu platzieren, sind keine Geb�hren notwendig. Das ist aufgrund zwei essentieller Eigenschaften m�glich. Zum Einen muss jede Transaktion zwei weitere nicht verifizierte Transaktionen best�tigen, des Weiteren wird mit Rechenleistung in Form eines Proof-of-Work die Transaktion beglaubigt. Im Detail l�uft dieses Verfahren wie folgt ab:
Nachdem alle Felder der Transaktion, bis auf die "`nonce"' und den Hash f�r  "`bundle"', einen Wert erhalten haben, wird mit Hilfe eines Algorithmus die nonce gesucht. Diese ist 27 Trytes lang. Sobald die nonce gefunden wurde, wird sie f�r die letzten 27 Trytes der insgesamt 2673 Trytes der Transaktion verwendet. Diese Trytes werden durch den Curl Hash Algorithmus zu einem 81 Trytes langen Hash umgewandelt. Der Hash wird im Anschluss in Trits umgerechnet. Am Ende der resultierenden 243 Trits soll eine Folge aus Nullen stehen. Die l�nge der Folge wird von der IOTA-Foundation vorgegeben. So m�ssen die letzten Trits der Trytes, eines Transaktions Hashes, momentan mindestens 14 mal Null in Folge f�r die Aufnahme in der Tangle und Neun mal Null in Folge f�r die Aufnahme in dem Testnetzwerk betragen. Diese Vorgabe wird "`Minimum Weight Magnitude (MWM)"' genannt. Der Vorgang wird sooft wiederholt bis das vorgegebene MWM erreicht ist. Grunds�tzlich kann die Aussage getroffen werden, dass mit einer steigenden MWM auch die Zeit, die f�r das Finden der passenden nonce ben�tigt wird, exponentiell ansteigt.

\subsection{Bundle}
IOTA ist nach einem Account - Schema aufgebaut, was genauer bedeutet, dass Adressen ben�tigt werden auf die Balance (der Fachausdruck des Geldwertes) eingegangen ist, um Balance weiter zu senden. Ein Bundle in IOTA umfasst in der Regel vier Transaktionen. Die erste Transaktion enth�lt die Adresse des Empf�ngers. Diese Transaktion wird mit einer positiven Balance versehen. In dem Bundle ist dies eine sogenannte "`Output"' - Transaktion. Daraufhin folgen in der Regel, abh�ngig von der gew�hlten Sicherheitsstufe, zwei Transaktionen. Jede dieser Transaktionen basiert auf der gleichen Adresse, auf der zuvor Balance eingegangen ist. In anderen Worten muss diese Adresse zuvor Teil einer "`Output"' - Transaktion gewesen sein. Es werden mehrere Transaktionen ben�tigt, da die Signatur des privaten Schl�ssels mit versandt wird. Diese Signatur vergr��ert sich, abh�ngig von der gew�hlten Sicherheitsstufe. Eine Transaktion, die sich einer Adresse bedient, die zuvor Balance empfing, nennt sich "`Input"' - Transaktion. Sollte die Balance der Input Transaktion nicht ausreichen um den Betrag der Output Transaktion zu decken, m�ssen weitere "`Input"' - Transaktionen angef�gt werden, die jeweils durch ihren privaten Schl�ssel signiert werden. Sollte die Balance der "`Input"' - Adressen den Wert der "`Output"' - Transaktion �berschreiten, wird der Restwert an eine "`Output"' - Adresse des Besitzers �bertragen. Solange der summierte Wert der "`Input"'-Adressen nicht �berschritten wird, k�nnen beliebig viele "`Output"'-Adressen einem Bundle angehangen werden.

Da Bundles atomar sind, werden entweder alle oder keine Transaktion verifiziert.

\subsection{Signatur}\label{OneTimeSignature}
Bei IOTA werden ausgehende Zahlungen (Output) mit zuvor eingegangenen Zahlungen (Input) durchgef�hrt. Um den Besitz des notwendigen Inputs nachzuweisen, verwendet IOTA die "`Winternitz One Time Signature"'. Diese Art der Signatur ver�ffentlicht einen Teil des privaten Schl�ssels und wird aus diesem Grund f�r den einmaligen Gebrauch von Signaturen verwendet. Um die Funktionsweise einer "`One Time Signature"' (kurz OTS) zu verdeutlichen, wird die �hnliche Lamport OTS in einem fiktiven Beispiel herangezogen.

Ein privater Schl�ssel besteht aus zwei gleichlangen Zahlenfolgen "`k1"' \& "`k2"'. Die L�nge des jeweiligen Schl�ssel Teils stimmt mit der L�nge der zu signierenden Nachricht �berein. In diesem Beispiel betr�gt die Schl�ssell�nge 512 (2x 256). Die Nachricht kann eine beliebige Zeichenkette sein, die bspw. durch den SHA-256 Hash Algorithmus zu einem Hash "`H"' mit der L�nge von insgesamt 256 Bits umgewandelt wurde. Der Wert eines Bits kann ausschlie�lich Null oder Eins betragen. In einer Schleife wird jeder Wert des Hashes durchlaufen. Betr�gt der Wert H(n) des Hashes an der Stelle n => H(n) = 0, so wird der Wert des ersten Schl�ssel Teils an selbiger Stelle f�r die Signatur Sig(n) = k1(n)  verwendet. Sollte jedoch der Wert H(n) des Hashes an der Stelle n => H(n) = 1 betragen, wird der Wert des zweiten Schl�ssel Teils in der Signatur Sig(n) = k2(n) verwendet. Nach dem erfolgreichen Durchlaufen des Hashes "`H"' wurde eine Signatur "`Sig"' mit einer L�nge von 256 angefertigt. Die Signatur enth�lt 50\% der Werte des privaten Schl�ssels.

Damit der Empf�nger die Signatur �berpr�fen kann, ben�tigt er den �ffentlichen Schl�ssel. Dieser ist bspw. ein, durch den SHA-256 Hash Algorithmus angefertigter, Hash "`pub"' des privaten Schl�ssels.
Der Empf�nger erh�lt die Nachricht, die Signatur �Sig� und den �ffentlichen Schl�ssel "`pub"'. Er wiederholt den Vorgang, den der Versender der Nachricht f�r die Signatur angewandt hat, mit dem �ffentlichen, mitgelieferten Schl�sselpaar "`pub1"' \& "`pub2"' anstatt des privaten Schl�sselpaars "`k1"' \& "`k2"'. Als Resultat erh�lt er eine gehashte Variante "`HSig"'. Sollte jeder Wert der "`HSig(n)"' identisch mit jedem Hash-Wert der Signatur "`SHA-256( Sig(n) )"' sein, ist die Integrit�t der Nachricht gew�hrleistet.~\cite{je:lamportSignature}

Sollte bei den kommenden Transaktionen erneut der gleiche private Schl�ssel zum Signieren herangezogen werden, wird bei jeder Signatur ein weiterer Teil des privaten Schl�ssels aufgedeckt. Ein b�swilliger Beobachter der Tangle k�nnte die Transaktionen abfangen und mittels Bruteforce-Attacken den gesamten privaten Schl�ssel aufdecken. Der Seed w�re weiterhin unbekannt, sollte der Adresse jedoch ein Wert hinterlegt sein, w�re der Angreifer in der Lage dar�ber zu verf�gen.

\section{Zusammenfassung}
Das Kapitel Funktion umfasst die detaillierte Funktionalit�t von den Kryptow�hrungen Bitcoin und IOTA. Die Kryptographie der privaten und �ffentlichen Schl�ssel der jeweiligen Technologien unterscheidet sich insofern, dass Bitcoin ausschlie�lich einen privaten Schl�ssel und den entsprechenden �ffentlichen Schl�ssel f�r das Erstellen und Bearbeiten einer Transaktion nutzt. IOTA hingegen generiert aus dem Seed mehrere private Schl�ssel dessen �ffentliche Schl�ssel als Adressen gelten. Bitcoin speichert Transaktionen in Bl�cken in der Blockchain, die durch "`minen"' circa alle 10 Minuten um einen weiteren Block verl�ngert wird und �ffentlich einsehbar ist. IOTA verwendet eine Tangle die sich nach dem DAG (Directed Acyclic Graph) in eine Richtung fortbewegt. Transaktionen referenzieren und validieren sich untereinander, wodurch kein 'Mining' notwendig ist. Weiterhin ist keine Grenze hinsichtlich der Skalierbarkeit der Datenmenge gesetzt.
\chapter{Sicherheit}
F�r die Beantwortung der zugrundeliegenden Fragestellung, ob die ausgew�hlten Distributed Ledger Technologien zukunftsf�hig sind, m�ssen die Sicherheitsmechanismen untersucht werden, durch die diese Technologien vor Angriffen gesch�tzt sind. Die gew�hlten Technologien werden dar�ber hinaus auch darauf gepr�ft, wie gut die Zahlungsmittel in der jeweiligen virtuellen Geldb�rse (Wallet) vor Diebstahl gesch�tzt sind. Dies beinhaltet auch zum einen, mit welchen Verschl�sselungsalgorithmen gearbeitet wird und zum anderen wie sicher sie sind. Weiterhin soll ergr�ndet werden, welche Ausma�e fehlerhafte Nutzung der Technologie auf die Sicherheit hat.

\section{Bitcoin}
Bitcoin setzt bei seiner Implementierung auf die Blockchain-Technologie, diese enth�lt intrinsische Sicherheitsrisiken, die bei der Umsetzung dieser Kryptow�hrung adressiert werden mussten. Im folgenden werden diese Risiken und das entsprechende Sicherheitsmerkmal auf Zukunftssicherheit gepr�ft.

\subsection{Konto Sicherheit}
Bitcoins geh�ren zu einer digitalen Geldb�rse. Diese Geldb�rse ist vereinfacht betrachtet ein digitaler Schl�ssel zu dem nur der Eigent�mer Zugang haben sollte. Es ist nur mit diesem Schl�ssel m�glich, zu beweisen welche und wie viele Bitcoins einer bestimmten Person zugeordnet sind. Die Generierung der Schl�ssel basiert auf ein ECC (Elliptic Curve Cryptography) Verschl�sselungsverfahren (siehe zu ECC \ref{ECC} auf Seite \pageref{ECC}). 

Aus diesem privatem Schl�ssel kann der �ffentliche Schl�ssel nachvollzogen werden, die eine Adresse darstellt, an die Bitcoins gesendet oder, von ihr ausgehend, Bitcoins versendet werden. Der �ffentliche Schl�ssel wird nochmals mit SHA256 und dann mit RIPEMD-160 gehasht. Die resultierende Zeichenfolge nennt sich eine P2PKH-Adresse (Pay To Public Key Hash). Sie besteht aus 27-34 alphanumerischer Zeichen mit Ausnahme von einigen Buchstaben/Zahlen f�r eine bessere Lesbarkeit. Dies geschieht als Absicherung f�r den Fall, dass es durch einen neuentdeckten Algorithmus  m�glich sein sollte, das ECDSA-Verschl�sselungsverfahren zu invertieren und den privaten Schl�ssel aus dem �ffentlichen Schl�ssel zu generieren.

Ein privater Schl�ssel bei Bitcoin besteht aus 256 Bits. Daraus erschlie�en sich $ 2^{256} \approx 1,16 \times 10^{77} $ Kombinationsm�glichkeiten f�r einen privaten Schl�ssel. Die Wahrscheinlichkeit, dass eine andere Instanz denselben Schl�ssel generiert, stellt somit kein Sicherheitsrisiko dar.

\subsection{Anonymit�t}
Durch die vollst�ndige, �ffentlich zug�ngliche Historie \emph{aller} Bitcointransaktionen ist das Netzwerk transparent. Es ist m�glich Bitcoins zur�ckzuverfolgen, um nachvollziehen zu k�nnen, durch welche Adressen diese gelaufen sind. 

Aufgrund dessen liegt der Aspekt der Anonymit�t nicht in der Verh�llung der Transaktionen, sondern in der Verschleierung der Identit�t. Private und �ffentliche Schl�ssel-Paare sind von der Identit�t des Nutzers entkoppelt, was wiederum bedeutet, dass der Schutz der Identit�t bei der Verantwortung des Nutzers liegt. Der Nutzer kann durch neue Schl�sselpaare versuchen seine Person weiter zu verschleiern und eine Zur�ckverfolgung zu erschweren. Aber auch ein sparsames Herausgeben von Informationen sollte angef�hrt werden als ein "`Best-Practise"'-Umgang mit Kryptow�hrungen allgemein.

\subsection{Validit�t einer Transaktion}
Eine Transaktion wird von allen Nodes verifiziert an die die Transaktion geschickt wird. Diese �berpr�fen die Transaktion auf diverse Kriterien wie unter anderem auf Syntax und ob die Inputs mit den Outputs �bereinstimmen. Nachdem die Transaktion als valide eingestuft wurde, wird sie in die Liste hinzugef�gt, die sich in dem - zu minenden - Block befindet.

Wenn der passende Hash zu dem Block gefunden wurde, wird dieser Block von der Node in das Netzwerk ausgesendet. Andere Nodes verifizieren diesen Block auf Korrektheit der Syntax und der enthaltenen Transaktionen. Nur wenn diese korrekt sind, dann wird an der Erzeugung des n�chsten Blocks auf Basis des vorherigen Blocks gearbeitet.

Es ist also f�r einen Angreifer nicht m�glich ung�ltige Transaktionen in das Netzwerk zu ver�ffentlichen, da sie nicht von ehrlichen Nodes angenommen werden w�rden. Dasselbe gilt auch f�r ung�ltige Bl�cke. Das wird garantiert durch die vollst�ndige Historie aller Bitcoins seit dem ersten Block, die jede Full-Node gespeichert hat. Es w�re realistisch nicht wahrscheinlich, dass ein Angreifer langfristig eine l�ngere Kette als die Hauptblockchain erzeugen kann, wenn er nicht 50\% der Gesamtleistung des Netzwerkes besitzt (mehr hierzu bei \ref{bitcoin:angriffsszenarios} auf Seite \pageref{bitcoin:angriffsszenarios}). Das ist n�mlich n�tig um als korrekte Kette angesehen zu werden, es gilt:\\Die Kette, die am meisten Rechenleistung b�rgt (l�ngste Kette) ist die Hauptkette.

Jedoch sollte angemerkt werden, dass durch die Wahrscheinlichkeit, dass ein Angreifer kurzfristig mehr Bl�cke erzeugen k�nnte als die wahre Hauptkette, eine Transaktion erst nach ca. sechs weiteren Bl�cken als best�tigt angesehen wird.(QUELLE)

\subsection{Angriffsszenarien}\label{bitcoin:angriffsszenarios}
Im Folgenden werden diverse Angriffsszenarien auf das Bitcoin-Netzwerk beschrieben. Es wird erl�utert mit welchen Mechanismen sich die Blockchain-Technologie vor diesen Angriffen sch�tzt.

\subsubsection{51\%-Attacke}
In Satoshi Nakamotos Bitcoin-Whitepaper wird erkl�rt, wie ein m�glicher Angriff durchzuf�hren w�re. Er beschreibt ein Szenario, welches als 51\%-Attacke bekannt ist. Grundlegend hiermit gemeint, das ein Angreifer mehr als 50\% der Hash-Rate (Gesamteistung) des Netzwerkes besitzt und diese Leistung nutzt um schneller eigene Bl�cke in das Netzwerk zu publizieren.

Dies wird n�tzlich, um Transaktionen die man get�tigt hat ung�ltig zu machen indem man eine weitere Transaktion von der Adresse an eine andere, eigene Adresse erstellt. Zum Beispiel kann damit ein Verk�ufer get�uscht werden, sodass die Person denkt, man h�tte die Bitcoins gezahlt. Im Hintergrund jedoch erstellt der Angreifer seine eigene Blockchain mit der, zu diesem Zweck, erstellten Transaktion an die eigene Adresse, womit zugleich die anf�ngliche Transaktion nicht mehr in einen Block aufgenommen werden kann, da die Bitcoins nicht mehr an der hinterlegten Adresse sind. Es m�ssen solange Blocks erzeugt werden, bis die falsche Blockchain l�nger ist als die ehrliche.

Bei einem erfolgreichen Angriff w�rde die eigene Wallet nicht mit der Zahlung an den Verk�ufer belastet werden. Ein weiterer Nebeneffekt ist, dass wom�glich viele Transaktionen in der Hauptkette nicht mehr validiert sind und erneut aufgenommen werden m�ssen.

Der Angriff hat jedoch Grenzen. Es k�nnen keine ung�ltigen Bl�cke vom Angreifer erstellt werden, da diese von anderen ehrlichen Nodes nicht angenommen werden w�rden. Der Konsens liegt weiterhin bei der Hauptkette.

Dieses Szenario ist auch denkbar bei einer Hash-Rate von unter 50\%, wird jedoch immer unwahrscheinlicher je mehr Bl�cke der Angreifer zur�ckliegt.
\begin{figure}[h]%
\includegraphics[scale=0.75]{26Prozent.PNG}%
\caption{Veranschaulichung zeigt wie unwahrscheinlich es wird, je weiter man zur�ckliegt (Quelle https://www.btc-echo.de/tutorial/bitcoin-51-attacke/)}%
\label{fig:26prozent}%
\end{figure}

Langfristig kann die Hauptkette nur ersetzt werden, wenn �ber 50\% der Hash-Rate dem Angreifer zur Verf�gung stehen.
\begin{figure}[h]%
\includegraphics[scale=0.75]{51Prozent.PNG}%
\caption{Zeigt wie hoch die Wahrscheinlichkeit ist, sechs Bl�cke hintereinander zu generieren (Quelle https://www.btc-echo.de/tutorial/bitcoin-51-attacke/)}%
\label{fig:51prozent}%
\end{figure}

Satoshi Nakamoto f�hrt folgende Formel zur Berechnung der Wahrscheinlichkeit an, dass ein Angreifer die Blockchain einholen kann aus \emph{z} Bl�cken im R�ckstand. Der zeitliche Rahmen ist unbegrenzt.

$ p = $ Wahrscheinlichkeit, dass ein ehrlicher Node den n�chsten Block erstellt\\
$ q = $ Wahrscheinlichkeit, dass der Angreifer den n�chsten Block erstellt\\
$ q_{z} = $ Wahrscheinlichkeit, dass der Angreifer die ehrliche Blockchain �berhohlt\\

$ q_{z} = 1 \quad\quad\quad\: wenn\ p\leq q \\
q_{z} = $ ( $ \frac{p}{q} $ ) $^{z} \quad wenn\ p > q $

Es geht hervor, dass die Chancen des Angreifers die Hauptkette zu �berhohlen bei unter 50\% Hash-Rate exponentiell schwinden.

Diese Angriffsfl�che er�ffnet sich aufgrund der dezentralen Architektur von Blockchain und ist ein integraler Bestandteil eines, auf Konsens beruhenden, Systems.

\subsection{Anreiz zur Ehrlichkeit}
Durch die Entlohnung in Form von Bitcoins durch das Minen und durch die Transaktionsgeb�hren wird Anreiz geschaffen, das Bitcoin-Netzwerk nicht zu manipulieren. Eine Kette mit ung�ltigen Bl�cken w�rde auch nicht von ehrlichen Minern angenommen werden. Daher wird erhofft ~\cite{onlinebeispiel}, dass die Belohnung und das Eigeninteresse an der Integrit�t von Bitcoin ausreicht um Angreifer abzuwehren.

\subsection{Quantenprozessorresistenz}
Mit der Einf�hrung leistungsstarker Quantenprozessoren, die darauf ausgelegt sind, kryptographische Algorithmen zu entschl�sseln, k�nnten Algorithmen (unter anderem ECDSA) unsicher werden. Auch die starke Leistung von Quantencomputern er�ffnet eine neue Angriffsfl�che unter dem Aspekt, dass der Anteil von �ber der H�lfte der Gesamtleistung des Miner-Netzwerkes dem Nutzer langfristig eine l�ngere Blockchain gew�hren kann. Auf dieses Problem wird in Kapitel \ref{bitcoin:angriffsszenarios} eingegangen.

Die genutzten kryptographischen Algorithmen sind nicht bewiesen, dass sie Quantenprozessoren standhalten k�nnten(QUELLE). Indes ist das Bitcoin-Netzwerk durch ihre Architektur und Nutzung nicht gegen Quantencomputern gesch�tzt. Aber komplett ungesch�tzt bleibt die Technologie nicht.

Auch wenn der ECDSA mit einem entsprechenden Algorithmus gel�st werden sollte, oder durch Quantenprozessoren in anderer Hinsicht unsicher werden sollte, wird durch das hashen des �ffentlichen Schl�ssels genau dieser verschleiert. Er kommt nur zum Vorschein, wenn von dieser Adresse eine Zahlung ausgeht, aufgrund der Signatur. Wenn nun von da an ein neues Schl�sselpaar generiert und genutzt wird, sollten die Bitcoins an der neuen Adresse sicher sein.

\section{IOTA}
IOTA nutzt im Gegensatz zu Bitcoin keine Blockchain-Technologie, sondern setzt auf ein sogenanntes Tangle, welches wiederum auf einem mathematischem Konzept basiert, welches sich Directed-Acyclic-Graph (DAG) nennt (siehe zu Tangle \ref{grundlagen:tangle} auf Seite \pageref{grundlagen:tangle}). Dieses Konzept unterscheidet sich grunds�tzlich von Blockchain und weist unter diesem Aspekt andere Sicherheitsrisiken auf. Im folgenden sollen diese beschrieben werden.

\subsection{Konto Sicherheit}
Auch bei IOTA besteht das Konto aus einem digitalen Schl�ssel, dem Seed. Dieser Seed sollte nur dem Eigent�mer verf�gbar gemacht werden, da jede Person mit diesem Seed, �ber die zugeordneten IOTAs verf�gen kann. F�r einen optimalen Schutz vor digitalem Diebstahl sollten Seeds nur lokal (ohne Internetzugang) erstellt und gespeichert werden. Wenn diese allgemeinen Sicherheitsvorschl�ge eingehalten werden, besteht hier noch kein nennenswertes Risiko.

Ein Seed besteht aus 81 Trytes (siehe zu Trytes Grundlagen \ref{grundlagen:tritsundtrytes} auf Seite \pageref{grundlagen:tritsundtrytes}). Ein Tryte enth�lt drei Trits und kann folglich 27 ganze Zahlen darstellen. Daraus ergeben sich $ 27^{81} \approx 8,72 \times 10^{115} $ verschiedene Kombinationsm�glichkeiten. Das hei�t, die Wahrscheinlichkeit, dass eine andere Instanz denselben Seed generiert wie der Eigent�mer ist $ 1 : 8,72 \times 10^{115} $, damit ist die Anzahl der Kombinationsm�glichkeiten deutlich h�her als die Anzahl der Atome im bekannten Universum.(FOOTNOTE)

Durch das One-Time-Winternitz hashing (siehe zu One-Time-Winternitz Funktion \ref{OneTimeSignature} auf Seite \pageref{OneTimeSignature}) ist eine Adresse f�r ausgehende Zahlungen nur einmal zu benutzen, da ein Teil des privaten Schl�ssels bekannt wird. Sollte die Adresse mehr als einmal genutzt werden, so wird zu viel des privaten Schl�ssels �ffentlich gemacht und eine Entschl�sselung wird zunehmend einfacher. Ist der private Schl�ssel vollst�ndig enth�llt, kann der Angreifer auf die IOTAs zugreifen die auf der Adresse hinterlegt sind.

Hierdurch er�ffnet sich ein Risiko, welches der Nutzer von IOTA selbst entgegnen muss, wenn er nicht das offizielle IOTA-Wallet Programm nutzt um Transaktionen zu erstellen. Hierzu muss angemerkt werden, dass durch die Kompromittierung des privaten Schl�ssels nicht der Seed oder alle anderen Adressen gef�hrdet sind.

\subsection{Anonymit�t}
Es ist wichtig zu betrachten, wie Anonym die Identit�t im Tangle-Netzwerk ist. Zum einen was einige Nutzer �ber andere Nutzer wissen k�nnen und zum anderen was in einer Transaktion �ber die Identit�t preisgegeben wird.

Vorerst muss notiert werden, dass Anonymit�t keine Priorit�t in der Entwicklung von IOTA gewesen ist und auch nicht prim�r f�r diesen Zweck entwickelt worden ist. (QUELLE)

Um eine Transaktion zu veranlassen wird ein Paar aus privatem und �ffentlichem Schl�ssel ben�tigt, welche aus dem Seed und einem Adressindex generiert werden. Der �ffentliche Schl�ssel dient hierbei als Adresse die man an seinen Transaktionspartner senden kann. Der private Schl�ssel sollte verborgen bleiben, denn jener Schl�ssel wird genutzt um die Inputs eines Bundles zu signieren. Damit wird bewiesen, dass der Ersteller der Transaktion den privaten Schl�ssel zu der Adresse besitzt und folglich der Eigent�mer ist.

- Erkl�ren was das mit anonymit�t zu tun hat

\subsection{Validit�t einer Transaktion}
Eine Transaktion wird von anderen Nodes validiert. Typischerweise geschieht das wenn ein Nutzer eine Transaktion in das Tangle publizieren will. Daf�r muss der Nutzer zwei Transaktionen validieren. Bei diesen Transaktionen handelt es sich um "`Tips"', die noch nicht referenziert sind. Das Transaktionsobjekt muss die Tips als "`branch"'- und "`trunkTransaction"' referenzieren und �berpr�fen.

Das Objekt wird auf Syntax und G�ltigkeit gepr�ft. Daf�r wird der Transaction-Hash abgeglichen und die Signierung �berpr�ft. 

- wie wird eine einzelne transaktion verifiziert, wie wird bestimmt dass sie g�ltig ist\\
- wann ist eine transaktion verifiziert\\

\subsection{Angriffsszenarien}
Die Tangle erm�glicht durch ihre DAG spezifische Arten von Attacken die im weiteren Fortgang erl�utert werden sollen. Durch die Darstellung der m�glichen Angriffe, werden weitere Sicherheitsmechanismen veranschaulicht.

\subsubsection{34\%-Attacke}
Ein b�swilliger Nutzer k�nnte mit mehr als 34\% der Gesamtrechenleistung im Tangle-Netzwerk einen Angriff starten indem er falsche Transaktionen mit vielen anderen selbst initiierten Transaktionen verifiziert und damit im DAG einen Strang erzeugt der falsche Informationen �ber die verf�gbaren IOTAs enth�lt.

Dem entgegenwirken soll der, von der IOTA-Foundation bereitgestellte, Coordinator. Dieser ist ein Node, der eigenst�ndig Transaktionen erstellt, die als Meilenstein (Milestone) gelten. Der Milestone dient als Beweis, dass die vorherigen Transaktionen korrekt sind. Jede Transaktion muss diese Coordinators indirekt oder direkt referenzieren.(QUELLE)

Diese Coordinators bieten jedoch eine neue Angriffsfl�che. Dabei geht es um das derzeit nicht vollkommen dezentrale System der Tangle bei IOTA. IOTA l�st also noch nicht das Problem, ohne dritte Instanz Transaktionen vermitteln zu k�nnen. Laut IOTA-Foundation sollen die Coordinators ihren Dienst einstellen, sobald das Tangle Netzwerk ausreichend gro� ist, um 34\%-Attacken durch eine zu hohe Anforderung an Rechenleistung abwehren zu k�nnen.(QUELLE)

\subsubsection{Parasiten-Kette}

- wie sch�tzt sich das Bitcoin netzwerk vor Angriffen\\
- welche angriffsszenarios sind denkbar\\
- was kann dagegen getan werden\\

\subsection{Quantenprozessorresistenz}
Bei der Frage nach Zukunftssicherheit bei IOTA geht es auch um die Resistenz vor Quantenprozessoren. Genauer soll betrachtet werden, ob sich die Fertigstellung von diesen leistungsstarken Prozessoren auf die Sicherheit und Best�ndigkeit von IOTAs auswirkt.

Hierbei ist es n�tig, auf den Winternitz onetime hashing algorithmus zur�ckzugreifen um zu verstehen wie sich der Algorithmus auf Quantenresistenz auswirkt.
\chapter{Programmierbarkeit}
In diesem Abschnitt folgt eine Beschreibung der Programmierschnittstellen bei Bitcoin und IOTA. Dies ist relevant da gegebene Schnittstellen und Ressourcen f�r Programmierer die Einsetzbarkeit erh�hen. Wenn Entwickler auf den vorhandenen Code aufbauen und erweitern k�nnen, k�nnen die 
Eine aussage �ber die zuk�nftigen einsatzm�glichkeiten

\section{Bitcoin}
Der Code f�r Bitcoin ist Open-Source und damit frei zug�nglich
Ja was gibts bei bitcoin so keine ahnung

\section{IOTA}
Iota erlaubt 0 vlaue transactions f�r meta daten zb json datein (nutzen hiervon f�r alltag umstritten wegen stormverbrauch), das erlaubt also quasi vershcicken von daten in der tangle
Dadurch erschlie�t sich ein weiterer Anwendungsbereich. Diese Funktion erlaubt eine n�tzliche Anwendung f�r das "`Internet der Dinge"' (internet of things \ref{} auf seite \pageref{}
\chapter{Nutzen}
\section{IOTA}
\section{Bitcoin}

\chapter{Ergebnis}
Im Folgenden werden die Ergebnisse der Analyse gegen�bergestellt und auf die Fragestellung der Zukunftstauglichkeit projiziert. Da jedes System, sowohl Bitcoin als auch IOTA, verschiedene Ziele anstrebt, werden beide Ergebnisse vereinzelt f�r die Beantwortung herangezogen.

\section{Kryptographie}
Die beiden Kryptow�hrungen Bitcoin und IOTA realisieren ihr jeweiliges Distributed Ledger System durch verschiedene Arten der Kryptographie. Bitcoin verwendet die elliptische Kurven Kryptographie um einen validen privaten Schl�ssel zu generieren. Dieser dient als Bitcoin Wallet und verweist auf die hinterlegten BTC.

IOTA verwendet als Wallet einen 81 Trytes langen Seed. Der Seed wird genutzt um private Schl�ssel zu generieren, um aus dessen Hash eine Adresse f�r den Empfang oder Versand von IOTAs zu veranlassen. Um einen Geldausgang zu verifizieren wird durch die Winternitz One-Time Signature eine Signatur erstellt. Sobald eine Transaktion erstmals signiert wurde, ist ein Teil des privaten Schl�ssels der verwendeten Adresse ver�ffentlicht. Die erneute Verwendung der Adresse h�tte einen eventuellen Verlust der hinterlegten IOTAs zur Folge.

\section{Architektur}
Bitcoin verwendet eine Blockchain. Diese fungiert wie ein �ffentliches Kassenbuch in der alle Transaktionen gespeichert sind, die momentan get�tigt aber nicht verifiziert sind. Sollte ein Miner das Hash-R�tsel l�sen, werden alle gespeicherten nicht verifizierten Transaktion in einen Block geschrieben und der Blockchain angehangen. Die umliegenden Nachbarn werden �ber den neuen Block informiert und nehmen ihn in die eigene Blockchain auf. Der Miner erh�lt f�r das "`Finden"' des neuen Blockes eine kleine Menge an BTCs. Der Schweregrad des R�tsels l�sst eine durchschnittliche Generierung von einem Block alle 10 Minuten zu~\cite{onlinebeispiel}. Bevor eine Transaktion als sicher eingestuft wird, werden typischerweise auf 6 Bl�cke gewartet, die angehangen werden m�ssen. Die Trennung von Nutzern und Minern resultiert darin, dass Nutzer Transaktionsgeb�hren zahlen m�ssen, da die Miner Rechenaufwand betreiben. Diese beiden Gr�nde machen die Blockchain f�r kleine Zahlungen wie Micro-Payments untauglich.

IOTA verwendet die Tangle. Damit eine Transaktion in der Tangle aufgenommen werden darf, muss die Transaktion zuvor zwei ebenfalls nicht verifizierte Tips verifizieren. Da dies Voraussetzung f�r die Aufnahme in der Tangle ist, fallen keine Transaktionsgeb�hren an und die Rolle des Miners f�llt weg. Sobald die Transaktion durch ausreichend andere Tips verifiziert wurde, gilt sie als verifiziert. Dieser Vorgang ist vergleichsweise schnell mit Best�tigungszeiten von ca. 5 Minuten\footnote{Aufgerufen am 15.10.2018 https://tanglemonitor.com/}.

\section{Sicherheit}
Die Sicherheit der beiden Technologien unterscheidet sich grundlegend aufgrund der Architektur. W�hrend bei Bitcoin die Gefahr einzig von einer 51\%-Attacke ausgeht. Sind bei IOTA eine Vielzahl von Angriffsszenarien denkbar, darunter eine 34\%-Attacke. Gesch�tzt wird das Netzwerk derzeit von eigenen Coordinators, die von der IOTA-Foundation bereitgestellt werden. Dadurch ist IOTA noch nicht vollkommen dezentral. Es ist jedoch geplant, bei gen�gend Teilnehmern die Coordinators abschalten zu k�nnen~\cite{onlinebeispiel}.

\section{Zukunftstauglichkeit Bitcoin}
Bitcoin ist das erste Distributed Ledger Protokoll dass die Blockchain verwendet. Insofern war es bereits Vorreiter um ein Umdenken des klassischen Server - Client Protokoll zu verursachen. Daten m�ssen nicht zwangsl�ufig durch solch zentrale Systeme ihre Richtigkeit garantieren. Die Blockchain l�st dies durch den Konsens der Miner, die in der Richtigkeit �bereinstimmen. Aber auch durch die Kombination der Kryptographie mit der Blockchain lassen sich zuk�nftig die Richtigkeit sensibler Daten garantieren, indem die digitale Signatur eines Dokumentes in der Blockchain gespeichert wird. 

Problematisch ist die Zeit die vergeht, bis eine Transaktion als verifiziert gilt. Das Finden eines Blockes dauert circa 10 Minuten. Bei einem denkbar hohen Datenaufkommen w�re das System nur f�r langfristige Datenspeicherung und als Wertanlage sinnvoll.

\section{Zukunftstauglichkeit IOTA}
IOTA wurde f�r das Internet of Things entwickelt. Es implementiert Funktionen um einen Datenstrom in der Tangle zu sichern. Dabei ist es unwichtig ob die Daten durch einen Menschen in die Tangle gelangen oder ob eine Maschine daf�r verantwortlich ist. Zuk�nftig k�nnte eine unbegrenzt gro�e Datenmenge dem Netzwerk beigef�gt werden. Maschinen k�nnten Temperaturen in einem zeitlichen Intervall publizieren. Die Sichtbarkeit k�nnte eingeschr�nkt werden und die Sicherheit w�re durch die Signatur und des �Merkle tree based signature scheme� gew�hrleistet.
Ein weiterer Zukunftsaspekt w�re das Versenden von Informationen �ber eine Transaktion. Eine Transaktion enth�lt das Feld �signatureMessageFragment� in dem ein JSON Objekt abgespeichert werden kann. Dementsprechend k�nnten auch gr��ere Dateien versendet werden.

\chapter{Realisierung des Prototypen}
Um eine bessere Aussage �ber die Zukunftstauglichkeit von Distributed Ledger Technologien zu gewinnen, wird gemeinsam mit der IOTA Technologie ein Anwendungsfall prototypisch implementiert. Bei diesem Anwendungsfall handelt es sich um eine Tankstelle an der ein Auto m�glichst autonom den Tankvorgang startet und im Anschluss den anfallenden Betrag mit der Kryptow�hrung IOTA begleicht. Die kommende Ausf�hrung befasst sich mit der Implementation eines Prototypen.

Aus dem analytischen Teil folgt, dass zwischen Bitcoin und IOTA deutliche Unterschiede herrschen. F�r die Auswahl einer Distributed Ledger Technologie wurden Faktoren wie Best�tigungszeit und Transaktionsgeb�hren miteinbezogen. Ein weiterer Grund f�r die Auswahl einer Technologie war die Dokumentation ihrer Bibliotheken.

Aufgrund einer vorhandenen Dokumentation, schnellen Verifizierungen von Transaktionen und das fehlen einer Geb�hr wurde IOTA f�r die Implementierung ausgew�hlt.

\chapter{Tankstelle - Server}
Die Kommunikation zwischen der Tankstelle und einem Fahrzeug soll durch eine Restful API geschehen. Das Fahrzeug soll imstande sein, durch das Abfragen der implementierten API, einer vor�bergehenden ID zugeordnet zu werden, welche f�r den gesamten Tankvorgang notwendig ist. Auch der Tank - Prozess soll durch gezielte Abfragen an die API gestartet und beendet werden. In einem abschlie�enden Schritt soll die Tankstellen API die anfallenden Kosten, sowie die notwendige IOTA - Adresse, �bermitteln.

Dem Besitzer der Tankstelle muss es m�glich sein durch autorisierte Anfragen an die API die IOTA - Adressen zu wechseln oder diesen Prozess durch das hinterlegen eines IOTA - Seed zu automatisieren. Die Adressen, die f�r den Bezahlvorgang verwendet werden sollen, werden f�r den autorisierten Besitzer der Tankstelle frei verf�gbar sein.

Es ist wichtig anzumerken, dass diese Art der Kommunikation eine von vielen M�glichkeiten ist, dieses Projekt zu realisieren. Im Endeffekt sind folgende Grundvoraussetzung zu erf�llen:
\begin{itemize}
	\item Dem Kunden darf es nicht m�glich sein den IOTA Seed einzusehen.
  \item Hinterlegte IOTA - Adressen d�rfen f�r unbeschr�nkt viele Zahlungseing�nge, aber ausschlie�lich f�r einen Zahlungsausgang verwendet werden.
  \item Die Aktualisierung der IOTA - Adresse kann automatisiert werden, dies birgt jedoch das Risiko, einen validen IOTA Seed dem Netzwerk zu hinterlegen.
  \item Autorisierten Personen muss es gestattet sein, neue IOTA - Adressen der Datenbank hinzuzuf�gen.
  \item Das System muss Zugang zu dem IOTA - Tangle besitzen und imstande sein dieser Transaktionen hinzuf�gen zu k�nnen.
\end{itemize}

\section{Projektaufbau}
In dem folgenden Unterkapitel wird der grobe Projektaufbau skizziert sowie die Installation der notwendigen Software aufgef�hrt. Diese Schritte sind notwendig um den Prototypen den Anforderungen entsprechend ausf�hren zu lassen.
\subsection{Vorbedingungen}
\subsubsection{NodeJS}
NodeJS ist eine asynchrone, auf �Events� basierende, JavaScript Laufzeitumgebung, die es erm�glicht mehrere Anfragen zu derselben Zeit abzuarbeiten. Dadurch entstehen niedrige Latenzen zwischen dem Client und dem Server. Des Weiteren verf�gt NodeJS �ber eine �EventLoop� die der in Web - Browsern bekannten �EventLoops� sehr �hnelt. Anders als aus Frameworks bekannt, startet die �EventLoop� mit Beginn des ausf�hrenden Scripts und endet sobald keine "`Callbacks"' mehr vorhanden sind. Dank dieses Verhaltens ist NodeJS f�r das Entwickeln einer Restful API pr�destiniert. [27](QUELLE)

Die Installation von NodeJS gestaltet sich f�r das Betriebssystem Windows nicht besonders herausfordernd. Auf der Homepage von NodeJS wird eine ausf�hrbare Datei angeboten, welche alle n�tigen Programme beinhaltet.

\subsubsection{NPM}
Durch die Installation von NodeJS wird der hauseigene sogenannte �node package manager� (kurz npm) zus�tzlich installiert. Dieses Werkzeug bef�higt Nutzer vorhandene und publizierte Module f�r die Entwicklung in NodeJS herunterzuladen oder eigene Module der �ffentlichkeit zur Verf�gung zu stellen. Um npm nutzen zu k�nnen wird eine funktionierende �Bash� vorausgesetzt.

\subsubsection{PowerShell}
Um mit NodeJS effektiv arbeiten zu k�nnen ben�tigen alle Programmierer eine geeignete Bash (Shell). Viele Unix und Linux Distributionen besitzen eine geeignete vorinstallierte Bash die f�r die meisten Befehle ausreicht. Um mit Windows entsprechend zu arbeiten, empfiehlt sich die Installation von der PowerShell [28](QUELLE).

\subsubsection{Express.js}
Eine Restful API basiert auf Anfragen (requests) die an einen Server �bermittelt werden. Grunds�tzlich folgt auf einen request stets eine Antwort (response). Ein handels�blicher Webserver verk�rpert dieses Prinzip. Sobald in einem Webbrowser die Anfrage zu einer IP - Adresse verschickt wird, antwortet dieser mit den hinterlegten Ressourcen (Meistens vorhandene .html Dateien) (siehe Grundlagen \ref{RestfulAPI} auf Seite \pageref{RestfulAPI}). Um angefragte Routen wie bspw. �/getAddress� zu verarbeiten, welche keine HTML - Daten als Antwort verlangen, werden sogenannte �Middlewares� geschrieben. Diese nehmen den angefragten Request entgegen, verarbeiten ihn und senden entweder einen abschlie�enden Response oder geben den Request weiter an die n�chste Middleware - Instanz. Dieser Vorgang wird so lange durchgef�hrt bis eine beendende Antwort an den Client zur�ckgesendet wird.

Express.js ist ein Modul f�r NodeJS, dass dem Programmierer erm�glicht vollwertige Middlewares zu erstellen. Die Installation gestaltet sich sehr leicht. Voraussetzung sind eine funktionierende Bash, sowie der Paketmanager �npm�. In der Shell wird folgendes Kommando ausgef�hrt:

\begin{lstlisting}[language=bash]
npm install express
\end{lstlisting}

Falls das Paket in einem Projekt abgespeichert werden soll, empfiehlt sich das folgende Kommando:

\begin{lstlisting}[language=bash]
npm install --save express
\end{lstlisting}

\subsubsection{MySQL}
Eine MySQL - Datenbank wird f�r das permanente Speichern der IOTA - Adresse genutzt. Um die Datenbank aufzusetzen empfiehlt sich das Hilfsprogramm XAMPP [29](FOOTNOTE). Durch dieses Programm kann eine MySQL Datenbank gestartet und �ber eine lokale Webpage konfiguriert werden. Nach dem Download der ausf�hrbaren Installationsdatei wird diese mit einem Doppelklick gestartet. Im Anschluss gen�gt es den Installationsanweisungen zu folgen. Nicht alle Komponenten sind notwendig, es gen�gen der Apache Webserver und die MySQL Unterst�tzung. XAMPP bietet eine GUI (graphical user interface) mit der neben einer Datenbank auch ein Apache Server gestartet werden kann. Dieser Apache Server ist notwendig um die lokale Konfigurations - Webseite f�r die MySQL Datenbank zu hosten.

\subsubsection{Port Weiterleitung}
Damit die API auf zuk�nftige Anfragen au�erhalb des lokalen Systems reagieren kann, muss das Netzwerk entsprechend eingerichtet werden. Der Firewall des zentralen Routers (standard Gateway) werden entsprechende Regeln hinzugef�gt, die Anfragen auf den Port 1717 (Dieser Port ist frei w�hlbar und wurde f�r dieses Projekt festgelegt) zulassen. Anfragen, die den genannten Port adressieren, werden zu dem Endger�t weitergeleitet, auf welchem die API instanziiert ist.

\subsection{Projektabgrenzung}
Eine geeignete Kommunikation innerhalb eines Netzwerkes kann auf verschiedene Weisen geschehen. Das Verwenden einer Restful API bietet den Vorteil Informationen und Ressourcen auf vielf�ltige Art und Weise auszutauschen. Die Kommunikation kann dementsprechend mithilfe eines JSON - Objektes oder �ber andere Protokolle wie XML oder URL-Encoding gel�st werden. Anders als andere Netzwerk Kommunikationstechniken wie bspw. SOAP sind Serverressourcen direkt adressierbar, ohne das im Vorhinein der Inhalt der Anfrage ausgewertet werden muss [30](QUELLE).

\section{Umsetzung}
Mit dem Start der API �berpr�ft das System ob eine JSON-Datei hinterlegt ist, in der die notwendigen Datenbank-Anmeldeinformationen gespeichert sind. Falls diese Datei nicht vorhanden ist, wird eine neue erstellt, indem die Informationen abgefragt werden. Sobald der Zugang zur Datenbank gesichert ist, startet der HTTP-Server und wartet anschlie�end auf dem eingerichteten Port auf kommende Anfragen. An diesem Punkt sind verschiedene Routen eingerichtet, die sowohl Daten verf�gbar machen, mit Daten interagieren oder Daten abspeichern. Diese verschiedenen Routen und ihre Funktionen werden im Folgenden erl�utert.(PICTURE)

\subsection{Tank-Vorgang}
Der Tankvorgang w�rde in einer realen Implementierung erst starten k�nnen sobald das Fahrzeug vorgefahren ist, die Zapfs�ule ausgew�hlt und die Zapfpistole des notwendigen Kraftstoff in den Tankstutzen eingef�hrt wurde. W�hrend dessen verbindet das Fahrzeug sich mit dem lokalen Netzwerk mit gegebenen Mitteln. Dieser gesamte Vorgang kann entweder von dem Fahrer oder durch einen Automatismus erfolgen und ist Voraussetzung f�r den kommenden Tank-Prozesses. Der vorliegende Prototyp schreibt daf�r keine Regeln vor.

Die API h�rt auf den Port 1717 und erwartet dort folgende Anfragen:
\begin{itemize}
	\item \{IP-Adresse des Servers\}/fueling/getStationInfo
  \item \{IP-Adresse des Servers\}/fueling/initializeFueling?\{ID der Zapfs�ule\} \& \{Kraftstoffart\} 
  \item \{IP-Adresse des Servers\}/fueling/startFueling?\{Kundennummer\}
  \item \{IP-Adresse des Servers\}/fueling/pauseFueling?\{Kundennummer\}
	\item \{IP-Adresse des Servers\}/fueling/endFueling?\{Kundennummer\}
  \item \{IP-Adresse des Servers\}/fueling/getFueling?\{Kundennummer\}
\end{itemize}

\paragraph{getStationInfo}
Die Tankstelle liefert die aktuellen Tank-Preise der verschiedenen Kraftstofftypen an das Fahrzeug in Form eines JSON-Objektes.

\begin{lstlisting}[language=JavaScript]
router.get('/getStationInfo', (req, resp, next) => {

	resp.status(200).json(Petrolstation.stationInfoObject);

});
\end{lstlisting}

\paragraph{initializeFueling}
Die API empf�ngt von dem Fahrzeug die gew�hlte Kraftstoffart und die ID der Zapfs�ule, an der der Tank-Vorgang gestartet werden soll. Diese Daten werden gemeinsam mit einer neu erstellten Kundennummer in der Datenbank hinterlegt. Diese spezielle Zapfs�ule ist ab diesem Zeitpunkt f�r den Tank-Prozess vorbereitet und wird anschlie�end nur durch die Kundennummer ansprechbar sein. Der Kunde erh�lt als Antwort die erstellte Kundennummer und kann den Tankvorgang beginnen.

\begin{lstlisting}[language=JavaScript]
router.get('/initializeFueling', (req, resp, next) => {

	if(! (req.query.fuel_type && req.query.station)) return next('You missed the required parameter.');

	petrolstation.initialize_fueling(req.query.fuel_type, req.query.station, (err, result) => {

		if(err) return next(err);

		resp.status(200).json({
			id: result
		});

	});

});
\end{lstlisting}

\paragraph{startFueling}
Diese Anfrage startet den Tankvorgang des Fahrzeuges und ben�tigt die in der Initialisierung vergebene Kundennummer zur Freigabe. Das Tanken wird durch eine Schleife simuliert. Es wird davon ausgegangen, dass innerhalb einer Sekunde ein Liter bzw. Kilowattstunde in das Fahrzeug getankt wird. Die Kosten werden entsprechend aufgerechnet.

\begin{lstlisting}[language=JavaScript]
router.get('/startFueling', check_for_initialisation, (req, resp, next) => {	

	if(petrolstation.start_fueling(req.query.id)){	
		resp.status(200).json({
			message: 'Start fueling.'
		});		
	}else{
		next('Anything went wrong.');
	}

});
\end{lstlisting}

\paragraph{pauseFueling}
Auch diese Anfrage muss die in der Initialisierung vergebene Kundennummer �bergeben um das Fahrzeug f�r den aktuellen Tank-Prozess zu autorisieren. Diese Anfrage pausiert den Tankvorgang der durch �startFueling� begonnen wurde. Technisch wird das erstellte Intervall gel�scht und die momentanen Betr�ge gespeichert.

\begin{lstlisting}[language=JavaScript]
router.get('/pauseFueling', check_for_initialisation, (req, resp, next) => {

	petrolstation.pause_fueling(req.query.id, (err, success) => {

		if(err) return next(err);

		resp.status(200).json({
			message: 'Pause fueling'
		});

	});

});
\end{lstlisting}

\paragraph{endFueling}
Diese Anfrage muss die entsprechende Kundennummer enthalten und beendet den Tank-Prozess. Das Kundenfahrzeug erh�lt die Anzahl der getankten Einheiten, die zugeh�rigen Kosten und die aktuelle IOTA-Adresse als JSON-Objekt. Die Zapfs�ule wird f�r eine erneute Initialisierung freigegeben. Die Kosten sowie der Betrag der getankten Einheiten werden in der Datenbank zu der entsprechenden Kundennummer gespeichert.

\begin{lstlisting}[language=JavaScript]
router.get('/endFueling', check_for_initialisation, (req, resp, next) => {

	petrolstation.end_fueling(req.query.id, (err, result) => {

		if(err) return next(err);

		resp.status(200).json({
			data: result
		});

	});
	
});
\end{lstlisting}

\paragraph{getFueling}
Der Kunde kann �ber diese Anfrage die Menge des getankten Kraftstoffes und die Kosten erfahren. Daf�r muss die entsprechende Kundennummer �bergeben werden. Die Antwort geschieht in Form eines JSON-Objektes.

\begin{lstlisting}[language=JavaScript]
router.get('/getFueling', check_for_initialisation, (req, resp, next) => {	

	resp.status(200).json({

			data: petrolstation.get_fueling(req.query.id)

	});

});
\end{lstlisting}

\subsection{Neue Adresse generieren}
Ausschlie�lich ein autorisierter Benutzer darf die Anfrage an die API stellen, eine neue Adresse, f�r den Empfang von Zahlungen, zu generieren. Die Autorisierung geschieht im Prototypen durch die "`basic access authentication"' Methode, bei der ein Hash, erstellt aus einem Benutzernamen und Passwort, an den Header einer HTTP Anfrage an gehangen wird. Die API kann diese Daten auslesen und die Resource freigeben.

\begin{lstlisting}[language=JavaScript]
var authorizeUser = function(username, password, cb){

	db.connect();

	let db_call = new Promise((resolve, reject) => {

		db.get_user_password(username, (err, rows) => {
			db.end();
			if (!arr_is_empty(rows) && passwordHash.verify(password, rows[0].password)) resolve(true);
			else resolve(false);

		});

	});

	db_call.then(value => {
		cb(null, value);
	});

}
\end{lstlisting}

Die autorisierte Anfrage an den Server muss an diese Route gesendet werden:
\begin{itemize}
	\item \{IP-Adresse des Servers\}/user/addNewAddress"' 
\end{itemize}
Serverseitig werden nach und nach Adressen generiert, solange bis eine gefunden wurde die noch nicht in der IOTA Tangle "`attached"' ist. Dementsprechend wird eine verbundene "`IOTA Full-Node API"' angesprochen, die Zugriff zu der Tangle hat. Nachdem eine neue Adresse gefunden wurde, wird eine wertlose Transaktion erstellt. Diese ist daf�r zust�ndig, die neue Adresse in der Tangle zu hinterlegen.

\begin{lstlisting}[language=JavaScript]
PetrolstatioIotaInterface.prototype.addNewAddressToTangle = async function(callback) {

	//Get the newest unused address
	let get_address = new Promise((res, rej)=>{

		this.iota.api.getNewAddress(this.seed, {index: 0}, (err, address)=>{
			if(err){
				callback(err);
				rej(err);
			}else{
				callback(null, address);
				res(address);
			}
		});

	});

	//Put the data into the transfers Array
	try{
		let address = await get_address;
		let depth = 3;
		let minWeightMagnitude = 14;
		let transfers = [{
			value: 0,
			address: address
		}];

		//send the transfer to the tangle
		this.iota.api.sendTransfer(this.seed, depth, minWeightMagnitude, transfers, (err, object)=>{
			if(err){
				console.log(err);
			}else{
				console.log(object);
			}
		});

	}catch(err){
		callback(err);
	}

};
\end{lstlisting}

Die Adresse wird zudem in der lokalen MySQL Datenbank gespeichert. Als Antwort erh�lt der autorisierte Benutzer die neue Adresse zusammen mit den veralteten Adressen aus der Datenbank:

\begin{lstlisting}[language=JavaScript]
{
    "newAddress": "LOZMZOKJ...AWSQFFXX",
    "affectedRows": 1,
    "savedAddresses": [
        "A9XZMWXP...TRKKUHZD",
        "GHDGOJFL...TVOETV9Z",
        "YOUZCEEZ...AKNKQSRW",
        "LOZMZOKJ...AWSQFFXX"
    ],
    "message": "Address is successfully added to the database and will be used for later transactions."
}
\end{lstlisting}
\chapter{Tankstelle - Server}
Die Kommunikation zwischen der Tankstelle und einem Fahrzeug soll durch eine Restful API geschehen. Das Fahrzeug soll imstande sein, durch das Abfragen der implementierten API, einer vor�bergehenden ID zugeordnet zu werden, welche f�r den gesamten Tankvorgang notwendig ist. Auch der Tank - Prozess soll durch gezielte Abfragen an die API gestartet und beendet werden. In einem abschlie�enden Schritt soll die Tankstellen API die anfallenden Kosten, sowie die notwendige IOTA - Adresse, �bermitteln.

Dem Besitzer der Tankstelle muss es m�glich sein durch autorisierte Anfragen an die API die IOTA - Adressen zu wechseln oder diesen Prozess durch das hinterlegen eines IOTA - Seed zu automatisieren. Die Adressen, die f�r den Bezahlvorgang verwendet werden sollen, werden f�r den autorisierten Besitzer der Tankstelle frei verf�gbar sein.

Es ist wichtig anzumerken, dass diese Art der Kommunikation eine von vielen M�glichkeiten ist, dieses Projekt zu realisieren. Im Endeffekt sind folgende Grundvoraussetzung zu erf�llen:
\begin{itemize}
	\item Dem Kunden darf es nicht m�glich sein den IOTA Seed einzusehen.
  \item Hinterlegte IOTA - Adressen d�rfen f�r unbeschr�nkt viele Zahlungseing�nge, aber ausschlie�lich f�r einen Zahlungsausgang verwendet werden.
  \item Die Aktualisierung der IOTA - Adresse kann automatisiert werden, dies birgt jedoch das Risiko, einen validen IOTA Seed dem Netzwerk zu hinterlegen.
  \item Autorisierten Personen muss es gestattet sein, neue IOTA - Adressen der Datenbank hinzuzuf�gen.
  \item Das System muss Zugang zu dem IOTA - Tangle besitzen und imstande sein dieser Transaktionen hinzuf�gen zu k�nnen.
\end{itemize}

\section{Projektaufbau}
In dem folgenden Unterkapitel wird der grobe Projektaufbau skizziert sowie die Installation der notwendigen Software aufgef�hrt. Diese Schritte sind notwendig um den Prototypen den Anforderungen entsprechend ausf�hren zu lassen.
\subsection{Vorbedingungen}
\subsubsection{NodeJS}
NodeJS ist eine asynchrone, auf �Events� basierende, JavaScript Laufzeitumgebung, die es erm�glicht mehrere Anfragen zu derselben Zeit abzuarbeiten. Dadurch entstehen niedrige Latenzen zwischen dem Client und dem Server. Des Weiteren verf�gt NodeJS �ber eine �EventLoop� die der in Web - Browsern bekannten �EventLoops� sehr �hnelt. Anders als aus Frameworks bekannt, startet die �EventLoop� mit Beginn des ausf�hrenden Scripts und endet sobald keine "`Callbacks"' mehr vorhanden sind. Dank dieses Verhaltens ist NodeJS f�r das Entwickeln einer Restful API pr�destiniert. [27](QUELLE)

Die Installation von NodeJS gestaltet sich f�r das Betriebssystem Windows nicht besonders herausfordernd. Auf der Homepage von NodeJS wird eine ausf�hrbare Datei angeboten, welche alle n�tigen Programme beinhaltet.

\subsubsection{NPM}
Durch die Installation von NodeJS wird der hauseigene sogenannte �node package manager� (kurz npm) zus�tzlich installiert. Dieses Werkzeug bef�higt Nutzer vorhandene und publizierte Module f�r die Entwicklung in NodeJS herunterzuladen oder eigene Module der �ffentlichkeit zur Verf�gung zu stellen. Um npm nutzen zu k�nnen wird eine funktionierende �Bash� vorausgesetzt.

\subsubsection{PowerShell}
Um mit NodeJS effektiv arbeiten zu k�nnen ben�tigen alle Programmierer eine geeignete Bash (Shell). Viele Unix und Linux Distributionen besitzen eine geeignete vorinstallierte Bash die f�r die meisten Befehle ausreicht. Um mit Windows entsprechend zu arbeiten, empfiehlt sich die Installation von der PowerShell [28](QUELLE).

\subsubsection{Express.js}
Eine Restful API basiert auf Anfragen (requests) die an einen Server �bermittelt werden. Grunds�tzlich folgt auf einen request stets eine Antwort (response). Ein handels�blicher Webserver verk�rpert dieses Prinzip. Sobald in einem Webbrowser die Anfrage zu einer IP - Adresse verschickt wird, antwortet dieser mit den hinterlegten Ressourcen (Meistens vorhandene .html Dateien) (siehe Grundlagen \ref{RestfulAPI} auf Seite \pageref{RestfulAPI}). Um angefragte Routen wie bspw. �/getAddress� zu verarbeiten, welche keine HTML - Daten als Antwort verlangen, werden sogenannte �Middlewares� geschrieben. Diese nehmen den angefragten Request entgegen, verarbeiten ihn und senden entweder einen abschlie�enden Response oder geben den Request weiter an die n�chste Middleware - Instanz. Dieser Vorgang wird so lange durchgef�hrt bis eine beendende Antwort an den Client zur�ckgesendet wird.

Express.js ist ein Modul f�r NodeJS, dass dem Programmierer erm�glicht vollwertige Middlewares zu erstellen. Die Installation gestaltet sich sehr leicht. Voraussetzung sind eine funktionierende Bash, sowie der Paketmanager �npm�. In der Shell wird folgendes Kommando ausgef�hrt:

\begin{lstlisting}[language=bash]
npm install express
\end{lstlisting}

Falls das Paket in einem Projekt abgespeichert werden soll, empfiehlt sich das folgende Kommando:

\begin{lstlisting}[language=bash]
npm install --save express
\end{lstlisting}

\subsubsection{MySQL}
Eine MySQL - Datenbank wird f�r das permanente Speichern der IOTA - Adresse genutzt. Um die Datenbank aufzusetzen empfiehlt sich das Hilfsprogramm XAMPP [29](FOOTNOTE). Durch dieses Programm kann eine MySQL Datenbank gestartet und �ber eine lokale Webpage konfiguriert werden. Nach dem Download der ausf�hrbaren Installationsdatei wird diese mit einem Doppelklick gestartet. Im Anschluss gen�gt es den Installationsanweisungen zu folgen. Nicht alle Komponenten sind notwendig, es gen�gen der Apache Webserver und die MySQL Unterst�tzung. XAMPP bietet eine GUI (graphical user interface) mit der neben einer Datenbank auch ein Apache Server gestartet werden kann. Dieser Apache Server ist notwendig um die lokale Konfigurations - Webseite f�r die MySQL Datenbank zu hosten.

\subsubsection{Port Weiterleitung}
Damit die API auf zuk�nftige Anfragen au�erhalb des lokalen Systems reagieren kann, muss das Netzwerk entsprechend eingerichtet werden. Der Firewall des zentralen Routers (standard Gateway) werden entsprechende Regeln hinzugef�gt, die Anfragen auf den Port 1717 (Dieser Port ist frei w�hlbar und wurde f�r dieses Projekt festgelegt) zulassen. Anfragen, die den genannten Port adressieren, werden zu dem Endger�t weitergeleitet, auf welchem die API instanziiert ist.

\subsection{Projektabgrenzung}
Eine geeignete Kommunikation innerhalb eines Netzwerkes kann auf verschiedene Weisen geschehen. Das Verwenden einer Restful API bietet den Vorteil Informationen und Ressourcen auf vielf�ltige Art und Weise auszutauschen. Die Kommunikation kann dementsprechend mithilfe eines JSON - Objektes oder �ber andere Protokolle wie XML oder URL-Encoding gel�st werden. Anders als andere Netzwerk Kommunikationstechniken wie bspw. SOAP sind Serverressourcen direkt adressierbar, ohne das im Vorhinein der Inhalt der Anfrage ausgewertet werden muss [30](QUELLE).

\section{Umsetzung}
Mit dem Start der API �berpr�ft das System ob eine JSON-Datei hinterlegt ist, in der die notwendigen Datenbank-Anmeldeinformationen gespeichert sind. Falls diese Datei nicht vorhanden ist, wird eine neue erstellt, indem die Informationen abgefragt werden. Sobald der Zugang zur Datenbank gesichert ist, startet der HTTP-Server und wartet anschlie�end auf dem eingerichteten Port auf kommende Anfragen. An diesem Punkt sind verschiedene Routen eingerichtet, die sowohl Daten verf�gbar machen, mit Daten interagieren oder Daten abspeichern. Diese verschiedenen Routen und ihre Funktionen werden im Folgenden erl�utert.(PICTURE)

\subsection{Tank-Vorgang}
Der Tankvorgang w�rde in einer realen Implementierung erst starten k�nnen sobald das Fahrzeug vorgefahren ist, die Zapfs�ule ausgew�hlt und die Zapfpistole des notwendigen Kraftstoff in den Tankstutzen eingef�hrt wurde. W�hrend dessen verbindet das Fahrzeug sich mit dem lokalen Netzwerk mit gegebenen Mitteln. Dieser gesamte Vorgang kann entweder von dem Fahrer oder durch einen Automatismus erfolgen und ist Voraussetzung f�r den kommenden Tank-Prozesses. Der vorliegende Prototyp schreibt daf�r keine Regeln vor.

Die API h�rt auf den Port 1717 und erwartet dort folgende Anfragen:
\begin{itemize}
	\item \{IP-Adresse des Servers\}/fueling/getStationInfo
  \item \{IP-Adresse des Servers\}/fueling/initializeFueling?\{ID der Zapfs�ule\} \& \{Kraftstoffart\} 
  \item \{IP-Adresse des Servers\}/fueling/startFueling?\{Kundennummer\}
  \item \{IP-Adresse des Servers\}/fueling/pauseFueling?\{Kundennummer\}
	\item \{IP-Adresse des Servers\}/fueling/endFueling?\{Kundennummer\}
  \item \{IP-Adresse des Servers\}/fueling/getFueling?\{Kundennummer\}
\end{itemize}

\paragraph{getStationInfo}
Die Tankstelle liefert die aktuellen Tank-Preise der verschiedenen Kraftstofftypen an das Fahrzeug in Form eines JSON-Objektes.

\begin{lstlisting}[language=JavaScript]
router.get('/getStationInfo', (req, resp, next) => {

	resp.status(200).json(Petrolstation.stationInfoObject);

});
\end{lstlisting}

\paragraph{initializeFueling}
Die API empf�ngt von dem Fahrzeug die gew�hlte Kraftstoffart und die ID der Zapfs�ule, an der der Tank-Vorgang gestartet werden soll. Diese Daten werden gemeinsam mit einer neu erstellten Kundennummer in der Datenbank hinterlegt. Diese spezielle Zapfs�ule ist ab diesem Zeitpunkt f�r den Tank-Prozess vorbereitet und wird anschlie�end nur durch die Kundennummer ansprechbar sein. Der Kunde erh�lt als Antwort die erstellte Kundennummer und kann den Tankvorgang beginnen.

\begin{lstlisting}[language=JavaScript]
router.get('/initializeFueling', (req, resp, next) => {

	if(! (req.query.fuel_type && req.query.station)) return next('You missed the required parameter.');

	petrolstation.initialize_fueling(req.query.fuel_type, req.query.station, (err, result) => {

		if(err) return next(err);

		resp.status(200).json({
			id: result
		});

	});

});
\end{lstlisting}

\paragraph{startFueling}
Diese Anfrage startet den Tankvorgang des Fahrzeuges und ben�tigt die in der Initialisierung vergebene Kundennummer zur Freigabe. Das Tanken wird durch eine Schleife simuliert. Es wird davon ausgegangen, dass innerhalb einer Sekunde ein Liter bzw. Kilowattstunde in das Fahrzeug getankt wird. Die Kosten werden entsprechend aufgerechnet.

\begin{lstlisting}[language=JavaScript]
router.get('/startFueling', check_for_initialisation, (req, resp, next) => {	

	if(petrolstation.start_fueling(req.query.id)){	
		resp.status(200).json({
			message: 'Start fueling.'
		});		
	}else{
		next('Anything went wrong.');
	}

});
\end{lstlisting}

\paragraph{pauseFueling}
Auch diese Anfrage muss die in der Initialisierung vergebene Kundennummer �bergeben um das Fahrzeug f�r den aktuellen Tank-Prozess zu autorisieren. Diese Anfrage pausiert den Tankvorgang der durch �startFueling� begonnen wurde. Technisch wird das erstellte Intervall gel�scht und die momentanen Betr�ge gespeichert.

\begin{lstlisting}[language=JavaScript]
router.get('/pauseFueling', check_for_initialisation, (req, resp, next) => {

	petrolstation.pause_fueling(req.query.id, (err, success) => {

		if(err) return next(err);

		resp.status(200).json({
			message: 'Pause fueling'
		});

	});

});
\end{lstlisting}

\paragraph{endFueling}
Diese Anfrage muss die entsprechende Kundennummer enthalten und beendet den Tank-Prozess. Das Kundenfahrzeug erh�lt die Anzahl der getankten Einheiten, die zugeh�rigen Kosten und die aktuelle IOTA-Adresse als JSON-Objekt. Die Zapfs�ule wird f�r eine erneute Initialisierung freigegeben. Die Kosten sowie der Betrag der getankten Einheiten werden in der Datenbank zu der entsprechenden Kundennummer gespeichert.

\begin{lstlisting}[language=JavaScript]
router.get('/endFueling', check_for_initialisation, (req, resp, next) => {

	petrolstation.end_fueling(req.query.id, (err, result) => {

		if(err) return next(err);

		resp.status(200).json({
			data: result
		});

	});
	
});
\end{lstlisting}

\paragraph{getFueling}
Der Kunde kann �ber diese Anfrage die Menge des getankten Kraftstoffes und die Kosten erfahren. Daf�r muss die entsprechende Kundennummer �bergeben werden. Die Antwort geschieht in Form eines JSON-Objektes.

\begin{lstlisting}[language=JavaScript]
router.get('/getFueling', check_for_initialisation, (req, resp, next) => {	

	resp.status(200).json({

			data: petrolstation.get_fueling(req.query.id)

	});

});
\end{lstlisting}

\subsection{Neue Adresse generieren}
Ausschlie�lich ein autorisierter Benutzer darf die Anfrage an die API stellen, eine neue Adresse, f�r den Empfang von Zahlungen, zu generieren. Die Autorisierung geschieht im Prototypen durch die "`basic access authentication"' Methode, bei der ein Hash, erstellt aus einem Benutzernamen und Passwort, an den Header einer HTTP Anfrage an gehangen wird. Die API kann diese Daten auslesen und die Resource freigeben.

\begin{lstlisting}[language=JavaScript]
var authorizeUser = function(username, password, cb){

	db.connect();

	let db_call = new Promise((resolve, reject) => {

		db.get_user_password(username, (err, rows) => {
			db.end();
			if (!arr_is_empty(rows) && passwordHash.verify(password, rows[0].password)) resolve(true);
			else resolve(false);

		});

	});

	db_call.then(value => {
		cb(null, value);
	});

}
\end{lstlisting}

Die autorisierte Anfrage an den Server muss an diese Route gesendet werden:
\begin{itemize}
	\item \{IP-Adresse des Servers\}/user/addNewAddress"' 
\end{itemize}
Serverseitig werden nach und nach Adressen generiert, solange bis eine gefunden wurde die noch nicht in der IOTA Tangle "`attached"' ist. Dementsprechend wird eine verbundene "`IOTA Full-Node API"' angesprochen, die Zugriff zu der Tangle hat. Nachdem eine neue Adresse gefunden wurde, wird eine wertlose Transaktion erstellt. Diese ist daf�r zust�ndig, die neue Adresse in der Tangle zu hinterlegen.

\begin{lstlisting}[language=JavaScript]
PetrolstatioIotaInterface.prototype.addNewAddressToTangle = async function(callback) {

	//Get the newest unused address
	let get_address = new Promise((res, rej)=>{

		this.iota.api.getNewAddress(this.seed, {index: 0}, (err, address)=>{
			if(err){
				callback(err);
				rej(err);
			}else{
				callback(null, address);
				res(address);
			}
		});

	});

	//Put the data into the transfers Array
	try{
		let address = await get_address;
		let depth = 3;
		let minWeightMagnitude = 14;
		let transfers = [{
			value: 0,
			address: address
		}];

		//send the transfer to the tangle
		this.iota.api.sendTransfer(this.seed, depth, minWeightMagnitude, transfers, (err, object)=>{
			if(err){
				console.log(err);
			}else{
				console.log(object);
			}
		});

	}catch(err){
		callback(err);
	}

};
\end{lstlisting}

Die Adresse wird zudem in der lokalen MySQL Datenbank gespeichert. Als Antwort erh�lt der autorisierte Benutzer die neue Adresse zusammen mit den veralteten Adressen aus der Datenbank:

\begin{lstlisting}[language=JavaScript]
{
    "newAddress": "LOZMZOKJ...AWSQFFXX",
    "affectedRows": 1,
    "savedAddresses": [
        "A9XZMWXP...TRKKUHZD",
        "GHDGOJFL...TVOETV9Z",
        "YOUZCEEZ...AKNKQSRW",
        "LOZMZOKJ...AWSQFFXX"
    ],
    "message": "Address is successfully added to the database and will be used for later transactions."
}
\end{lstlisting}

\chapter{Ergebnis der Umsetzung}
Was aus den vorherigen Abschnitten folgt, ist ein Programm, welches mit einem Server �ber eine Rest-Schnittstelle kommunizieren kann und eigenst�ndig eine Transaktion ausf�hren kann. Mit dieser Applikation werden die Bezahlvorg�nge mit IOTAs abgewickelt.

Um das Ergebnis zu veranschaulichen wird das Programm vorgespielt.

...... Einmal probedurchlaufen mit dem Programm und erkl�ren wie das mit dem vorher erkl�rten zusammenh�ngt

Die API wird im ersten Schritt gestartet. Sollte die Verbindung zu der Datenbank gelingen erscheint die Nachricht, dass die API gestartet sei und auf den Port 1717 h�rt. (PICTURE start\_API)
\chapter{Fazit}
Abschlie�end werden die ausgearbeiteten Ergebnisse zusammengefasst und besprochen. Zudem wird ein Ausblick auf weitere Forschungsbereiche gegeben, die in dieser Arbeit nicht untersucht wurden.

\section{Zusammenfassung}
Der analytische Teil der Arbeit zeigt, dass eine Nutzung der Distributed Ledger Technologien in einer zuk�nftigen Umgebung durchaus denkbar ist. Wenn auch einige Bedenken seitens der Sicherheit gegeben sind. So ist es letztendlich schwer zu sagen ob die Einf�hrung der Quantencomputer nicht weitere Angriffsfl�chen auf die Kryptow�hrungen offenbaren als in dieser Ausarbeitung untersucht wurden sind. Denn die Sicherheit der kryptographischen Funktionen ist nur gegeben solange keine L�sungen f�r die zugrundeliegenden Probleme gefunden wurden.

Abseits der Verwendung als Kryptow�hrung k�nnen Distributed Ledger Technologien auch f�r die Datenspeicherung oder zum Versenden von Informationen genutzt werden. Diese experimentellen Implementierungen zeugen davon, dass das Potential noch nicht ausgesch�pft ist.

Zugleich best�tigt die prototypische Umsetzung der m�glichen Tankanwendung, dass eine Nutzung der Kryptow�hrungen f�r allt�gliche Prozesse schon gegenw�rtig m�glich ist. Da sich das Potential der Technologien nicht ausschlie�lich auf die Funktion als W�hrung beschr�nkt, ist mit weiteren Implementierungen im Bereich der Datenverarbeitung sowie Datensicherung zu rechnen.

Vorherzubestimmen welche Distributed Ledger Technologie sich behaupten k�nnte ist kaum Vorhersehbar. Jede Implementierung visiert unterschiedliche Ziele an und daher ist es wahrscheinlich, dass weiterhin eine Vielzahl an diversen Distributed Ledger Technologien existieren wird. Aufgrund der nicht ausgereiften Sicherheit ist jedoch �ber den momentanen Stand IOTAs keine endg�ltige Prognose aufzustellen.

\section{Ausblick}
Diese Arbeit hat sich mit der allgemeinen Technologie diverser Kryptow�hrungen besch�ftigt. Dies wurde in Aussicht auf die Beantwortung der Fragestellung durchgef�hrt, inwiefern Distributed Ledger Technologien zukunftstauglich sind. Dabei wurden einige weiterf�hrende Untersuchungen vernachl�ssigt, die im Folgenden kurz aufgef�hrt werden:

\subsection{Auswirkungen auf die Finanzwirtschaft}
Kryptow�hrungen sind Werteinheiten mit denen physische Gegenst�nde, sowohl im Internet als auch in einigen lokalen Gesch�ftstellen, erworben werden k�nnen. Bitcoin und IOTA verwenden pseudonymisierte Benutzerdaten, welche nicht den Besitzern direkt zugeordnet werden k�nnen. Aufgrund dieser Eigenschaft ist es deutlich erschwert den Geldfluss zu beobachten, zu regulieren oder zu unterbinden. Die Auswirkungen dieser Eigenschaften auf die zuk�nftige Finanzwirtschaft wurde in dieser Arbeit nicht behandelt.

\subsection{Die Funktion IOTAs neben der Kryptow�hrung}
IOTA bietet neben der Funktion als Kryptow�hrung noch weitere attraktive Funktionen. So ist es m�glich Dateien in Form eines JSON - Objektes an ein Ziel zu senden. Dabei ist es nicht von Belangen ob das Gegen�ber ein Mensch oder eine Maschine ist. Diese Eigenschaft, kombiniert mit der grenzenlosen Skalierbarkeit der Datenmenge k�nnte das momentane "`Client - Server"' Modell durch ein neues ersetzen. Auch die Art wie Daten im Internet publiziert und erreicht werden, k�nnte durch die Architektur des Tangle-Netzwerkes umgedacht werden.

\bibliographystyle{geralpha}
\bibliography{quellen}

\cleardoublepage
\pagestyle{plain}
\section*{Arbeitsteilung}
\pagenumbering{gobble}
Diese Seite stellt dar, welcher Autor ein bestimmtes Kapitel ausgearbeitet hat. In den wesentlichen Hauptkapiteln bestand eine klare Arbeitsteilung, in den anderen wurden beide Autoren f�r die Bearbeitung herangezogen.

Spatz, Janusz: \textbf{S}\\
Altrock, Jens: \textbf{A}\\

\begin{center}
\begin{tabular}{|l|c|}
\hline
Kapitel & Bearbeiter\\
\hline
1. Einleitung & A \& S\\
2. Grundlagen & \\
2.1 Hardware & S\\
2.2 Hash Tree (Merkle Tree) & A\\
2.3 Proof-of-Work (Hashcash) & A\\
2.4 Internet of Things (IoT) & S\\
2.5 Industrie 4.0 & S\\
2.6 Trits, Trytes \& Tryte-Alphabet & A\\
2.7 Distributed Ledger & A\\
2.8 Hash & A\\
2.9 Wallet & S\\
2.10 Restful API & A\\
2.11 Zusammenfassung & A \& S\\
3 Analyse der DLT & A \& S\\
4 Funktion & A\\
5 Sicherheit & S\\
6 Nutzen & A \& S\\
7 Ergebnis & A \& S\\
8 Realisierung des Prototypen & A \& S\\
9 Fahrzeug - Client & S\\
10 Tankstelle - Server & A\\
11 Ergebnis der Umsetzung & A \& S\\
12 Fazit & A \& S\\
\hline
\end{tabular}
\end{center}


\end{document}