\chapter{Programmierbarkeit}
In diesem Abschnitt folgt eine Beschreibung der Programmierschnittstellen bei Bitcoin und IOTA. Dies ist relevant da gegebene Schnittstellen und Ressourcen f�r Programmierer die Einsetzbarkeit erh�hen. Wenn Entwickler auf den vorhandenen Code aufbauen und erweitern k�nnen, k�nnen die 
Eine aussage �ber die zuk�nftigen einsatzm�glichkeiten

\section{Bitcoin}
Der Code f�r Bitcoin ist Open-Source und damit frei zug�nglich
Ja was gibts bei bitcoin so keine ahnung

\section{IOTA}
Iota erlaubt 0 vlaue transactions f�r meta daten zb json datein (nutzen hiervon f�r alltag umstritten wegen stormverbrauch), das erlaubt also quasi vershcicken von daten in der tangle
Dadurch erschlie�t sich ein weiterer Anwendungsbereich. Diese Funktion erlaubt eine n�tzliche Anwendung f�r das "`Internet der Dinge"' (internet of things \ref{} auf seite \pageref{}