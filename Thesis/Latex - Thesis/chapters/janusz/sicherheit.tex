\chapter{Sicherheit}
F�r die Beantwortung der zugrundeliegenden Thesis m�ssen die Sicherheitsmechanismen untersucht werden, durch die die Distributed Ledger Technologien vor Angriffen gesch�tzt sind. Die auserw�hlten Technologien werden dar�ber hinaus auch darauf gepr�ft, wie gut die Zahlungsmittel in einer Wallet vor Diebstahl gesch�tzt sind. Dies beinhaltet auch zu pr�fen, mit welchen Verschl�sselungsalgorythmen gearbeitet wird und wie hoch die Warscheinlichkeit ist, diesen Algorythmus l�sen zu k�nnen. Weiterhin wird gepr�ft, welche Ausma�e fehlerhafte Nutzung der Kryptow�hrung auf die Sicherheit hat.

\section{Bitcoin}
- wallet anlegen, wie viele verschiedene wallets k�nnte es geben � wie hoch warscheinlichkeit, dass 2 dieselben keys bekommen

- bezahlen mit bitcoin, was wird �ber die identit�t revealed

- wie wird eine einzelne transaktion verifiziert

- wann ist eine transaktion verifiziert

\section{Andere Krypto}
lorem ipsum

\section{IOTA}
- IOTA White paper, possible attack scenarios
- quantenresistenz durch one time hash
- wie wird eine einzelne transaktion verifiziert
- wann ist eine transaktion verifiziert

\section{Gegen�berstellung}
[Tabelle mit kriterien und vergleichen]