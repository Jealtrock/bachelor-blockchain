\chapter{Realisierung des Prototypen}
Um eine bessere Aussage �ber die Zukunftstauglichkeit von Distributed Ledger Technologien zu gewinnen, wird gemeinsam mit der IOTA Technologie ein Anwendungsfall prototypisch implementiert. Bei diesem Anwendungsfall handelt es sich um eine Tankstelle an der ein Auto m�glichst autonom den Tankvorgang startet und im Anschluss den anfallenden Betrag mit der Kryptow�hrung IOTA begleicht. Die kommende Ausf�hrung befasst sich mit der Implementation eines Prototypen.

Aus dem analytischen Teil folgt, dass zwischen Bitcoin und IOTA deutliche Unterschiede herrschen. F�r die Auswahl einer Distributed Ledger Technologie wurden Faktoren wie Best�tigungszeit und Transaktionsgeb�hren miteinbezogen. Ein weiterer Grund f�r die Auswahl einer Technologie war die Dokumentation ihrer Bibliotheken.

Aufgrund einer vorhandenen Dokumentation, schnellen Verifizierungen von Transaktionen und das fehlen einer Geb�hr wurde IOTA f�r die Implementierung ausgew�hlt.

\chapter{Tankstelle - Server}
Die Kommunikation zwischen der Tankstelle und einem Fahrzeug soll durch eine Restful API geschehen. Das Fahrzeug soll imstande sein, durch das Abfragen der implementierten API, einer vor�bergehenden ID zugeordnet zu werden, welche f�r den gesamten Tankvorgang notwendig ist. Auch der Tank - Prozess soll durch gezielte Abfragen an die API gestartet und beendet werden. In einem abschlie�enden Schritt soll die Tankstellen API die anfallenden Kosten, sowie die notwendige IOTA - Adresse, �bermitteln.

Dem Besitzer der Tankstelle muss es m�glich sein durch autorisierte Anfragen an die API die IOTA - Adressen zu wechseln oder diesen Prozess durch das hinterlegen eines IOTA - Seed zu automatisieren. Die Adressen, die f�r den Bezahlvorgang verwendet werden sollen, werden f�r den autorisierten Besitzer der Tankstelle frei verf�gbar sein.

Es ist wichtig anzumerken, dass diese Art der Kommunikation eine von vielen M�glichkeiten ist, dieses Projekt zu realisieren. Im Endeffekt sind folgende Grundvoraussetzung zu erf�llen:
\begin{itemize}
	\item Dem Kunden darf es nicht m�glich sein den IOTA Seed einzusehen.
  \item Hinterlegte IOTA - Adressen d�rfen f�r unbeschr�nkt viele Zahlungseing�nge, aber ausschlie�lich f�r einen Zahlungsausgang verwendet werden.
  \item Die Aktualisierung der IOTA - Adresse kann automatisiert werden, dies birgt jedoch das Risiko, einen validen IOTA Seed dem Netzwerk zu hinterlegen.
  \item Autorisierten Personen muss es gestattet sein, neue IOTA - Adressen der Datenbank hinzuzuf�gen.
  \item Das System muss Zugang zu dem IOTA - Tangle besitzen und imstande sein dieser Transaktionen hinzuf�gen zu k�nnen.
\end{itemize}

\section{Projektaufbau}
In dem folgenden Unterkapitel wird der grobe Projektaufbau skizziert sowie die Installation der notwendigen Software aufgef�hrt. Diese Schritte sind notwendig um den Prototypen den Anforderungen entsprechend ausf�hren zu lassen.
\subsection{Vorbedingungen}
\subsubsection{NodeJS}
NodeJS ist eine asynchrone, auf �Events� basierende, JavaScript Laufzeitumgebung, die es erm�glicht mehrere Anfragen zu derselben Zeit abzuarbeiten. Dadurch entstehen niedrige Latenzen zwischen dem Client und dem Server. Des Weiteren verf�gt NodeJS �ber eine �EventLoop� die der in Web - Browsern bekannten �EventLoops� sehr �hnelt. Anders als aus Frameworks bekannt, startet die �EventLoop� mit Beginn des ausf�hrenden Scripts und endet sobald keine "`Callbacks"' mehr vorhanden sind. Dank dieses Verhaltens ist NodeJS f�r das Entwickeln einer Restful API pr�destiniert. [27](QUELLE)

Die Installation von NodeJS gestaltet sich f�r das Betriebssystem Windows nicht besonders herausfordernd. Auf der Homepage von NodeJS wird eine ausf�hrbare Datei angeboten, welche alle n�tigen Programme beinhaltet.

\subsubsection{NPM}
Durch die Installation von NodeJS wird der hauseigene sogenannte �node package manager� (kurz npm) zus�tzlich installiert. Dieses Werkzeug bef�higt Nutzer vorhandene und publizierte Module f�r die Entwicklung in NodeJS herunterzuladen oder eigene Module der �ffentlichkeit zur Verf�gung zu stellen. Um npm nutzen zu k�nnen wird eine funktionierende �Bash� vorausgesetzt.

\subsubsection{PowerShell}
Um mit NodeJS effektiv arbeiten zu k�nnen ben�tigen alle Programmierer eine geeignete Bash (Shell). Viele Unix und Linux Distributionen besitzen eine geeignete vorinstallierte Bash die f�r die meisten Befehle ausreicht. Um mit Windows entsprechend zu arbeiten, empfiehlt sich die Installation von der PowerShell [28](QUELLE).

\subsubsection{Express.js}
Eine Restful API basiert auf Anfragen (requests) die an einen Server �bermittelt werden. Grunds�tzlich folgt auf einen request stets eine Antwort (response). Ein handels�blicher Webserver verk�rpert dieses Prinzip. Sobald in einem Webbrowser die Anfrage zu einer IP - Adresse verschickt wird, antwortet dieser mit den hinterlegten Ressourcen (Meistens vorhandene .html Dateien) (siehe Grundlagen \ref{RestfulAPI} auf Seite \pageref{RestfulAPI}). Um angefragte Routen wie bspw. �/getAddress� zu verarbeiten, welche keine HTML - Daten als Antwort verlangen, werden sogenannte �Middlewares� geschrieben. Diese nehmen den angefragten Request entgegen, verarbeiten ihn und senden entweder einen abschlie�enden Response oder geben den Request weiter an die n�chste Middleware - Instanz. Dieser Vorgang wird so lange durchgef�hrt bis eine beendende Antwort an den Client zur�ckgesendet wird.

Express.js ist ein Modul f�r NodeJS, dass dem Programmierer erm�glicht vollwertige Middlewares zu erstellen. Die Installation gestaltet sich sehr leicht. Voraussetzung sind eine funktionierende Bash, sowie der Paketmanager �npm�. In der Shell wird folgendes Kommando ausgef�hrt:

\begin{lstlisting}[language=bash]
npm install express
\end{lstlisting}

Falls das Paket in einem Projekt abgespeichert werden soll, empfiehlt sich das folgende Kommando:

\begin{lstlisting}[language=bash]
npm install --save express
\end{lstlisting}

\subsubsection{MySQL}
Eine MySQL - Datenbank wird f�r das permanente Speichern der IOTA - Adresse genutzt. Um die Datenbank aufzusetzen empfiehlt sich das Hilfsprogramm XAMPP [29](FOOTNOTE). Durch dieses Programm kann eine MySQL Datenbank gestartet und �ber eine lokale Webpage konfiguriert werden. Nach dem Download der ausf�hrbaren Installationsdatei wird diese mit einem Doppelklick gestartet. Im Anschluss gen�gt es den Installationsanweisungen zu folgen. Nicht alle Komponenten sind notwendig, es gen�gen der Apache Webserver und die MySQL Unterst�tzung. XAMPP bietet eine GUI (graphical user interface) mit der neben einer Datenbank auch ein Apache Server gestartet werden kann. Dieser Apache Server ist notwendig um die lokale Konfigurations - Webseite f�r die MySQL Datenbank zu hosten.

\subsubsection{Port Weiterleitung}
Damit die API auf zuk�nftige Anfragen au�erhalb des lokalen Systems reagieren kann, muss das Netzwerk entsprechend eingerichtet werden. Der Firewall des zentralen Routers (standard Gateway) werden entsprechende Regeln hinzugef�gt, die Anfragen auf den Port 1717 (Dieser Port ist frei w�hlbar und wurde f�r dieses Projekt festgelegt) zulassen. Anfragen, die den genannten Port adressieren, werden zu dem Endger�t weitergeleitet, auf welchem die API instanziiert ist.

\subsection{Projektabgrenzung}
Eine geeignete Kommunikation innerhalb eines Netzwerkes kann auf verschiedene Weisen geschehen. Das Verwenden einer Restful API bietet den Vorteil Informationen und Ressourcen auf vielf�ltige Art und Weise auszutauschen. Die Kommunikation kann dementsprechend mithilfe eines JSON - Objektes oder �ber andere Protokolle wie XML oder URL-Encoding gel�st werden. Anders als andere Netzwerk Kommunikationstechniken wie bspw. SOAP sind Serverressourcen direkt adressierbar, ohne das im Vorhinein der Inhalt der Anfrage ausgewertet werden muss [30](QUELLE).

\section{Umsetzung}
Mit dem Start der API �berpr�ft das System ob eine JSON-Datei hinterlegt ist, in der die notwendigen Datenbank-Anmeldeinformationen gespeichert sind. Falls diese Datei nicht vorhanden ist, wird eine neue erstellt, indem die Informationen abgefragt werden. Sobald der Zugang zur Datenbank gesichert ist, startet der HTTP-Server und wartet anschlie�end auf dem eingerichteten Port auf kommende Anfragen. An diesem Punkt sind verschiedene Routen eingerichtet, die sowohl Daten verf�gbar machen, mit Daten interagieren oder Daten abspeichern. Diese verschiedenen Routen und ihre Funktionen werden im Folgenden erl�utert.(PICTURE)

\subsection{Tank-Vorgang}
Der Tankvorgang w�rde in einer realen Implementierung erst starten k�nnen sobald das Fahrzeug vorgefahren ist, die Zapfs�ule ausgew�hlt und die Zapfpistole des notwendigen Kraftstoff in den Tankstutzen eingef�hrt wurde. W�hrend dessen verbindet das Fahrzeug sich mit dem lokalen Netzwerk mit gegebenen Mitteln. Dieser gesamte Vorgang kann entweder von dem Fahrer oder durch einen Automatismus erfolgen und ist Voraussetzung f�r den kommenden Tank-Prozesses. Der vorliegende Prototyp schreibt daf�r keine Regeln vor.

Die API h�rt auf den Port 1717 und erwartet dort folgende Anfragen:
\begin{itemize}
	\item \{IP-Adresse des Servers\}/fueling/getStationInfo
  \item \{IP-Adresse des Servers\}/fueling/initializeFueling?\{ID der Zapfs�ule\} \& \{Kraftstoffart\} 
  \item \{IP-Adresse des Servers\}/fueling/startFueling?\{Kundennummer\}
  \item \{IP-Adresse des Servers\}/fueling/pauseFueling?\{Kundennummer\}
	\item \{IP-Adresse des Servers\}/fueling/endFueling?\{Kundennummer\}
  \item \{IP-Adresse des Servers\}/fueling/getFueling?\{Kundennummer\}
\end{itemize}

\paragraph{getStationInfo}
Die Tankstelle liefert die aktuellen Tank-Preise der verschiedenen Kraftstofftypen an das Fahrzeug in Form eines JSON-Objektes.

\begin{lstlisting}[language=JavaScript]
router.get('/getStationInfo', (req, resp, next) => {

	resp.status(200).json(Petrolstation.stationInfoObject);

});
\end{lstlisting}

\paragraph{initializeFueling}
Die API empf�ngt von dem Fahrzeug die gew�hlte Kraftstoffart und die ID der Zapfs�ule, an der der Tank-Vorgang gestartet werden soll. Diese Daten werden gemeinsam mit einer neu erstellten Kundennummer in der Datenbank hinterlegt. Diese spezielle Zapfs�ule ist ab diesem Zeitpunkt f�r den Tank-Prozess vorbereitet und wird anschlie�end nur durch die Kundennummer ansprechbar sein. Der Kunde erh�lt als Antwort die erstellte Kundennummer und kann den Tankvorgang beginnen.

\begin{lstlisting}[language=JavaScript]
router.get('/initializeFueling', (req, resp, next) => {

	if(! (req.query.fuel_type && req.query.station)) return next('You missed the required parameter.');

	petrolstation.initialize_fueling(req.query.fuel_type, req.query.station, (err, result) => {

		if(err) return next(err);

		resp.status(200).json({
			id: result
		});

	});

});
\end{lstlisting}

\paragraph{startFueling}
Diese Anfrage startet den Tankvorgang des Fahrzeuges und ben�tigt die in der Initialisierung vergebene Kundennummer zur Freigabe. Das Tanken wird durch eine Schleife simuliert. Es wird davon ausgegangen, dass innerhalb einer Sekunde ein Liter bzw. Kilowattstunde in das Fahrzeug getankt wird. Die Kosten werden entsprechend aufgerechnet.

\begin{lstlisting}[language=JavaScript]
router.get('/startFueling', check_for_initialisation, (req, resp, next) => {	

	if(petrolstation.start_fueling(req.query.id)){	
		resp.status(200).json({
			message: 'Start fueling.'
		});		
	}else{
		next('Anything went wrong.');
	}

});
\end{lstlisting}

\paragraph{pauseFueling}
Auch diese Anfrage muss die in der Initialisierung vergebene Kundennummer �bergeben um das Fahrzeug f�r den aktuellen Tank-Prozess zu autorisieren. Diese Anfrage pausiert den Tankvorgang der durch �startFueling� begonnen wurde. Technisch wird das erstellte Intervall gel�scht und die momentanen Betr�ge gespeichert.

\begin{lstlisting}[language=JavaScript]
router.get('/pauseFueling', check_for_initialisation, (req, resp, next) => {

	petrolstation.pause_fueling(req.query.id, (err, success) => {

		if(err) return next(err);

		resp.status(200).json({
			message: 'Pause fueling'
		});

	});

});
\end{lstlisting}

\paragraph{endFueling}
Diese Anfrage muss die entsprechende Kundennummer enthalten und beendet den Tank-Prozess. Das Kundenfahrzeug erh�lt die Anzahl der getankten Einheiten, die zugeh�rigen Kosten und die aktuelle IOTA-Adresse als JSON-Objekt. Die Zapfs�ule wird f�r eine erneute Initialisierung freigegeben. Die Kosten sowie der Betrag der getankten Einheiten werden in der Datenbank zu der entsprechenden Kundennummer gespeichert.

\begin{lstlisting}[language=JavaScript]
router.get('/endFueling', check_for_initialisation, (req, resp, next) => {

	petrolstation.end_fueling(req.query.id, (err, result) => {

		if(err) return next(err);

		resp.status(200).json({
			data: result
		});

	});
	
});
\end{lstlisting}

\paragraph{getFueling}
Der Kunde kann �ber diese Anfrage die Menge des getankten Kraftstoffes und die Kosten erfahren. Daf�r muss die entsprechende Kundennummer �bergeben werden. Die Antwort geschieht in Form eines JSON-Objektes.

\begin{lstlisting}[language=JavaScript]
router.get('/getFueling', check_for_initialisation, (req, resp, next) => {	

	resp.status(200).json({

			data: petrolstation.get_fueling(req.query.id)

	});

});
\end{lstlisting}

\subsection{Neue Adresse generieren}
Ausschlie�lich ein autorisierter Benutzer darf die Anfrage an die API stellen, eine neue Adresse, f�r den Empfang von Zahlungen, zu generieren. Die Autorisierung geschieht im Prototypen durch die "`basic access authentication"' Methode, bei der ein Hash, erstellt aus einem Benutzernamen und Passwort, an den Header einer HTTP Anfrage an gehangen wird. Die API kann diese Daten auslesen und die Resource freigeben.

\begin{lstlisting}[language=JavaScript]
var authorizeUser = function(username, password, cb){

	db.connect();

	let db_call = new Promise((resolve, reject) => {

		db.get_user_password(username, (err, rows) => {
			db.end();
			if (!arr_is_empty(rows) && passwordHash.verify(password, rows[0].password)) resolve(true);
			else resolve(false);

		});

	});

	db_call.then(value => {
		cb(null, value);
	});

}
\end{lstlisting}

Die autorisierte Anfrage an den Server muss an diese Route gesendet werden:
\begin{itemize}
	\item \{IP-Adresse des Servers\}/user/addNewAddress"' 
\end{itemize}
Serverseitig werden nach und nach Adressen generiert, solange bis eine gefunden wurde die noch nicht in der IOTA Tangle "`attached"' ist. Dementsprechend wird eine verbundene "`IOTA Full-Node API"' angesprochen, die Zugriff zu der Tangle hat. Nachdem eine neue Adresse gefunden wurde, wird eine wertlose Transaktion erstellt. Diese ist daf�r zust�ndig, die neue Adresse in der Tangle zu hinterlegen.

\begin{lstlisting}[language=JavaScript]
PetrolstatioIotaInterface.prototype.addNewAddressToTangle = async function(callback) {

	//Get the newest unused address
	let get_address = new Promise((res, rej)=>{

		this.iota.api.getNewAddress(this.seed, {index: 0}, (err, address)=>{
			if(err){
				callback(err);
				rej(err);
			}else{
				callback(null, address);
				res(address);
			}
		});

	});

	//Put the data into the transfers Array
	try{
		let address = await get_address;
		let depth = 3;
		let minWeightMagnitude = 14;
		let transfers = [{
			value: 0,
			address: address
		}];

		//send the transfer to the tangle
		this.iota.api.sendTransfer(this.seed, depth, minWeightMagnitude, transfers, (err, object)=>{
			if(err){
				console.log(err);
			}else{
				console.log(object);
			}
		});

	}catch(err){
		callback(err);
	}

};
\end{lstlisting}

Die Adresse wird zudem in der lokalen MySQL Datenbank gespeichert. Als Antwort erh�lt der autorisierte Benutzer die neue Adresse zusammen mit den veralteten Adressen aus der Datenbank:

\begin{lstlisting}[language=JavaScript]
{
    "newAddress": "LOZMZOKJ...AWSQFFXX",
    "affectedRows": 1,
    "savedAddresses": [
        "A9XZMWXP...TRKKUHZD",
        "GHDGOJFL...TVOETV9Z",
        "YOUZCEEZ...AKNKQSRW",
        "LOZMZOKJ...AWSQFFXX"
    ],
    "message": "Address is successfully added to the database and will be used for later transactions."
}
\end{lstlisting}
\chapter{Tankstelle - Server}
Die Kommunikation zwischen der Tankstelle und einem Fahrzeug soll durch eine Restful API geschehen. Das Fahrzeug soll imstande sein, durch das Abfragen der implementierten API, einer vor�bergehenden ID zugeordnet zu werden, welche f�r den gesamten Tankvorgang notwendig ist. Auch der Tank - Prozess soll durch gezielte Abfragen an die API gestartet und beendet werden. In einem abschlie�enden Schritt soll die Tankstellen API die anfallenden Kosten, sowie die notwendige IOTA - Adresse, �bermitteln.

Dem Besitzer der Tankstelle muss es m�glich sein durch autorisierte Anfragen an die API die IOTA - Adressen zu wechseln oder diesen Prozess durch das hinterlegen eines IOTA - Seed zu automatisieren. Die Adressen, die f�r den Bezahlvorgang verwendet werden sollen, werden f�r den autorisierten Besitzer der Tankstelle frei verf�gbar sein.

Es ist wichtig anzumerken, dass diese Art der Kommunikation eine von vielen M�glichkeiten ist, dieses Projekt zu realisieren. Im Endeffekt sind folgende Grundvoraussetzung zu erf�llen:
\begin{itemize}
	\item Dem Kunden darf es nicht m�glich sein den IOTA Seed einzusehen.
  \item Hinterlegte IOTA - Adressen d�rfen f�r unbeschr�nkt viele Zahlungseing�nge, aber ausschlie�lich f�r einen Zahlungsausgang verwendet werden.
  \item Die Aktualisierung der IOTA - Adresse kann automatisiert werden, dies birgt jedoch das Risiko, einen validen IOTA Seed dem Netzwerk zu hinterlegen.
  \item Autorisierten Personen muss es gestattet sein, neue IOTA - Adressen der Datenbank hinzuzuf�gen.
  \item Das System muss Zugang zu dem IOTA - Tangle besitzen und imstande sein dieser Transaktionen hinzuf�gen zu k�nnen.
\end{itemize}

\section{Projektaufbau}
In dem folgenden Unterkapitel wird der grobe Projektaufbau skizziert sowie die Installation der notwendigen Software aufgef�hrt. Diese Schritte sind notwendig um den Prototypen den Anforderungen entsprechend ausf�hren zu lassen.
\subsection{Vorbedingungen}
\subsubsection{NodeJS}
NodeJS ist eine asynchrone, auf �Events� basierende, JavaScript Laufzeitumgebung, die es erm�glicht mehrere Anfragen zu derselben Zeit abzuarbeiten. Dadurch entstehen niedrige Latenzen zwischen dem Client und dem Server. Des Weiteren verf�gt NodeJS �ber eine �EventLoop� die der in Web - Browsern bekannten �EventLoops� sehr �hnelt. Anders als aus Frameworks bekannt, startet die �EventLoop� mit Beginn des ausf�hrenden Scripts und endet sobald keine "`Callbacks"' mehr vorhanden sind. Dank dieses Verhaltens ist NodeJS f�r das Entwickeln einer Restful API pr�destiniert. [27](QUELLE)

Die Installation von NodeJS gestaltet sich f�r das Betriebssystem Windows nicht besonders herausfordernd. Auf der Homepage von NodeJS wird eine ausf�hrbare Datei angeboten, welche alle n�tigen Programme beinhaltet.

\subsubsection{NPM}
Durch die Installation von NodeJS wird der hauseigene sogenannte �node package manager� (kurz npm) zus�tzlich installiert. Dieses Werkzeug bef�higt Nutzer vorhandene und publizierte Module f�r die Entwicklung in NodeJS herunterzuladen oder eigene Module der �ffentlichkeit zur Verf�gung zu stellen. Um npm nutzen zu k�nnen wird eine funktionierende �Bash� vorausgesetzt.

\subsubsection{PowerShell}
Um mit NodeJS effektiv arbeiten zu k�nnen ben�tigen alle Programmierer eine geeignete Bash (Shell). Viele Unix und Linux Distributionen besitzen eine geeignete vorinstallierte Bash die f�r die meisten Befehle ausreicht. Um mit Windows entsprechend zu arbeiten, empfiehlt sich die Installation von der PowerShell [28](QUELLE).

\subsubsection{Express.js}
Eine Restful API basiert auf Anfragen (requests) die an einen Server �bermittelt werden. Grunds�tzlich folgt auf einen request stets eine Antwort (response). Ein handels�blicher Webserver verk�rpert dieses Prinzip. Sobald in einem Webbrowser die Anfrage zu einer IP - Adresse verschickt wird, antwortet dieser mit den hinterlegten Ressourcen (Meistens vorhandene .html Dateien) (siehe Grundlagen \ref{RestfulAPI} auf Seite \pageref{RestfulAPI}). Um angefragte Routen wie bspw. �/getAddress� zu verarbeiten, welche keine HTML - Daten als Antwort verlangen, werden sogenannte �Middlewares� geschrieben. Diese nehmen den angefragten Request entgegen, verarbeiten ihn und senden entweder einen abschlie�enden Response oder geben den Request weiter an die n�chste Middleware - Instanz. Dieser Vorgang wird so lange durchgef�hrt bis eine beendende Antwort an den Client zur�ckgesendet wird.

Express.js ist ein Modul f�r NodeJS, dass dem Programmierer erm�glicht vollwertige Middlewares zu erstellen. Die Installation gestaltet sich sehr leicht. Voraussetzung sind eine funktionierende Bash, sowie der Paketmanager �npm�. In der Shell wird folgendes Kommando ausgef�hrt:

\begin{lstlisting}[language=bash]
npm install express
\end{lstlisting}

Falls das Paket in einem Projekt abgespeichert werden soll, empfiehlt sich das folgende Kommando:

\begin{lstlisting}[language=bash]
npm install --save express
\end{lstlisting}

\subsubsection{MySQL}
Eine MySQL - Datenbank wird f�r das permanente Speichern der IOTA - Adresse genutzt. Um die Datenbank aufzusetzen empfiehlt sich das Hilfsprogramm XAMPP [29](FOOTNOTE). Durch dieses Programm kann eine MySQL Datenbank gestartet und �ber eine lokale Webpage konfiguriert werden. Nach dem Download der ausf�hrbaren Installationsdatei wird diese mit einem Doppelklick gestartet. Im Anschluss gen�gt es den Installationsanweisungen zu folgen. Nicht alle Komponenten sind notwendig, es gen�gen der Apache Webserver und die MySQL Unterst�tzung. XAMPP bietet eine GUI (graphical user interface) mit der neben einer Datenbank auch ein Apache Server gestartet werden kann. Dieser Apache Server ist notwendig um die lokale Konfigurations - Webseite f�r die MySQL Datenbank zu hosten.

\subsubsection{Port Weiterleitung}
Damit die API auf zuk�nftige Anfragen au�erhalb des lokalen Systems reagieren kann, muss das Netzwerk entsprechend eingerichtet werden. Der Firewall des zentralen Routers (standard Gateway) werden entsprechende Regeln hinzugef�gt, die Anfragen auf den Port 1717 (Dieser Port ist frei w�hlbar und wurde f�r dieses Projekt festgelegt) zulassen. Anfragen, die den genannten Port adressieren, werden zu dem Endger�t weitergeleitet, auf welchem die API instanziiert ist.

\subsection{Projektabgrenzung}
Eine geeignete Kommunikation innerhalb eines Netzwerkes kann auf verschiedene Weisen geschehen. Das Verwenden einer Restful API bietet den Vorteil Informationen und Ressourcen auf vielf�ltige Art und Weise auszutauschen. Die Kommunikation kann dementsprechend mithilfe eines JSON - Objektes oder �ber andere Protokolle wie XML oder URL-Encoding gel�st werden. Anders als andere Netzwerk Kommunikationstechniken wie bspw. SOAP sind Serverressourcen direkt adressierbar, ohne das im Vorhinein der Inhalt der Anfrage ausgewertet werden muss [30](QUELLE).

\section{Umsetzung}
Mit dem Start der API �berpr�ft das System ob eine JSON-Datei hinterlegt ist, in der die notwendigen Datenbank-Anmeldeinformationen gespeichert sind. Falls diese Datei nicht vorhanden ist, wird eine neue erstellt, indem die Informationen abgefragt werden. Sobald der Zugang zur Datenbank gesichert ist, startet der HTTP-Server und wartet anschlie�end auf dem eingerichteten Port auf kommende Anfragen. An diesem Punkt sind verschiedene Routen eingerichtet, die sowohl Daten verf�gbar machen, mit Daten interagieren oder Daten abspeichern. Diese verschiedenen Routen und ihre Funktionen werden im Folgenden erl�utert.(PICTURE)

\subsection{Tank-Vorgang}
Der Tankvorgang w�rde in einer realen Implementierung erst starten k�nnen sobald das Fahrzeug vorgefahren ist, die Zapfs�ule ausgew�hlt und die Zapfpistole des notwendigen Kraftstoff in den Tankstutzen eingef�hrt wurde. W�hrend dessen verbindet das Fahrzeug sich mit dem lokalen Netzwerk mit gegebenen Mitteln. Dieser gesamte Vorgang kann entweder von dem Fahrer oder durch einen Automatismus erfolgen und ist Voraussetzung f�r den kommenden Tank-Prozesses. Der vorliegende Prototyp schreibt daf�r keine Regeln vor.

Die API h�rt auf den Port 1717 und erwartet dort folgende Anfragen:
\begin{itemize}
	\item \{IP-Adresse des Servers\}/fueling/getStationInfo
  \item \{IP-Adresse des Servers\}/fueling/initializeFueling?\{ID der Zapfs�ule\} \& \{Kraftstoffart\} 
  \item \{IP-Adresse des Servers\}/fueling/startFueling?\{Kundennummer\}
  \item \{IP-Adresse des Servers\}/fueling/pauseFueling?\{Kundennummer\}
	\item \{IP-Adresse des Servers\}/fueling/endFueling?\{Kundennummer\}
  \item \{IP-Adresse des Servers\}/fueling/getFueling?\{Kundennummer\}
\end{itemize}

\paragraph{getStationInfo}
Die Tankstelle liefert die aktuellen Tank-Preise der verschiedenen Kraftstofftypen an das Fahrzeug in Form eines JSON-Objektes.

\begin{lstlisting}[language=JavaScript]
router.get('/getStationInfo', (req, resp, next) => {

	resp.status(200).json(Petrolstation.stationInfoObject);

});
\end{lstlisting}

\paragraph{initializeFueling}
Die API empf�ngt von dem Fahrzeug die gew�hlte Kraftstoffart und die ID der Zapfs�ule, an der der Tank-Vorgang gestartet werden soll. Diese Daten werden gemeinsam mit einer neu erstellten Kundennummer in der Datenbank hinterlegt. Diese spezielle Zapfs�ule ist ab diesem Zeitpunkt f�r den Tank-Prozess vorbereitet und wird anschlie�end nur durch die Kundennummer ansprechbar sein. Der Kunde erh�lt als Antwort die erstellte Kundennummer und kann den Tankvorgang beginnen.

\begin{lstlisting}[language=JavaScript]
router.get('/initializeFueling', (req, resp, next) => {

	if(! (req.query.fuel_type && req.query.station)) return next('You missed the required parameter.');

	petrolstation.initialize_fueling(req.query.fuel_type, req.query.station, (err, result) => {

		if(err) return next(err);

		resp.status(200).json({
			id: result
		});

	});

});
\end{lstlisting}

\paragraph{startFueling}
Diese Anfrage startet den Tankvorgang des Fahrzeuges und ben�tigt die in der Initialisierung vergebene Kundennummer zur Freigabe. Das Tanken wird durch eine Schleife simuliert. Es wird davon ausgegangen, dass innerhalb einer Sekunde ein Liter bzw. Kilowattstunde in das Fahrzeug getankt wird. Die Kosten werden entsprechend aufgerechnet.

\begin{lstlisting}[language=JavaScript]
router.get('/startFueling', check_for_initialisation, (req, resp, next) => {	

	if(petrolstation.start_fueling(req.query.id)){	
		resp.status(200).json({
			message: 'Start fueling.'
		});		
	}else{
		next('Anything went wrong.');
	}

});
\end{lstlisting}

\paragraph{pauseFueling}
Auch diese Anfrage muss die in der Initialisierung vergebene Kundennummer �bergeben um das Fahrzeug f�r den aktuellen Tank-Prozess zu autorisieren. Diese Anfrage pausiert den Tankvorgang der durch �startFueling� begonnen wurde. Technisch wird das erstellte Intervall gel�scht und die momentanen Betr�ge gespeichert.

\begin{lstlisting}[language=JavaScript]
router.get('/pauseFueling', check_for_initialisation, (req, resp, next) => {

	petrolstation.pause_fueling(req.query.id, (err, success) => {

		if(err) return next(err);

		resp.status(200).json({
			message: 'Pause fueling'
		});

	});

});
\end{lstlisting}

\paragraph{endFueling}
Diese Anfrage muss die entsprechende Kundennummer enthalten und beendet den Tank-Prozess. Das Kundenfahrzeug erh�lt die Anzahl der getankten Einheiten, die zugeh�rigen Kosten und die aktuelle IOTA-Adresse als JSON-Objekt. Die Zapfs�ule wird f�r eine erneute Initialisierung freigegeben. Die Kosten sowie der Betrag der getankten Einheiten werden in der Datenbank zu der entsprechenden Kundennummer gespeichert.

\begin{lstlisting}[language=JavaScript]
router.get('/endFueling', check_for_initialisation, (req, resp, next) => {

	petrolstation.end_fueling(req.query.id, (err, result) => {

		if(err) return next(err);

		resp.status(200).json({
			data: result
		});

	});
	
});
\end{lstlisting}

\paragraph{getFueling}
Der Kunde kann �ber diese Anfrage die Menge des getankten Kraftstoffes und die Kosten erfahren. Daf�r muss die entsprechende Kundennummer �bergeben werden. Die Antwort geschieht in Form eines JSON-Objektes.

\begin{lstlisting}[language=JavaScript]
router.get('/getFueling', check_for_initialisation, (req, resp, next) => {	

	resp.status(200).json({

			data: petrolstation.get_fueling(req.query.id)

	});

});
\end{lstlisting}

\subsection{Neue Adresse generieren}
Ausschlie�lich ein autorisierter Benutzer darf die Anfrage an die API stellen, eine neue Adresse, f�r den Empfang von Zahlungen, zu generieren. Die Autorisierung geschieht im Prototypen durch die "`basic access authentication"' Methode, bei der ein Hash, erstellt aus einem Benutzernamen und Passwort, an den Header einer HTTP Anfrage an gehangen wird. Die API kann diese Daten auslesen und die Resource freigeben.

\begin{lstlisting}[language=JavaScript]
var authorizeUser = function(username, password, cb){

	db.connect();

	let db_call = new Promise((resolve, reject) => {

		db.get_user_password(username, (err, rows) => {
			db.end();
			if (!arr_is_empty(rows) && passwordHash.verify(password, rows[0].password)) resolve(true);
			else resolve(false);

		});

	});

	db_call.then(value => {
		cb(null, value);
	});

}
\end{lstlisting}

Die autorisierte Anfrage an den Server muss an diese Route gesendet werden:
\begin{itemize}
	\item \{IP-Adresse des Servers\}/user/addNewAddress"' 
\end{itemize}
Serverseitig werden nach und nach Adressen generiert, solange bis eine gefunden wurde die noch nicht in der IOTA Tangle "`attached"' ist. Dementsprechend wird eine verbundene "`IOTA Full-Node API"' angesprochen, die Zugriff zu der Tangle hat. Nachdem eine neue Adresse gefunden wurde, wird eine wertlose Transaktion erstellt. Diese ist daf�r zust�ndig, die neue Adresse in der Tangle zu hinterlegen.

\begin{lstlisting}[language=JavaScript]
PetrolstatioIotaInterface.prototype.addNewAddressToTangle = async function(callback) {

	//Get the newest unused address
	let get_address = new Promise((res, rej)=>{

		this.iota.api.getNewAddress(this.seed, {index: 0}, (err, address)=>{
			if(err){
				callback(err);
				rej(err);
			}else{
				callback(null, address);
				res(address);
			}
		});

	});

	//Put the data into the transfers Array
	try{
		let address = await get_address;
		let depth = 3;
		let minWeightMagnitude = 14;
		let transfers = [{
			value: 0,
			address: address
		}];

		//send the transfer to the tangle
		this.iota.api.sendTransfer(this.seed, depth, minWeightMagnitude, transfers, (err, object)=>{
			if(err){
				console.log(err);
			}else{
				console.log(object);
			}
		});

	}catch(err){
		callback(err);
	}

};
\end{lstlisting}

Die Adresse wird zudem in der lokalen MySQL Datenbank gespeichert. Als Antwort erh�lt der autorisierte Benutzer die neue Adresse zusammen mit den veralteten Adressen aus der Datenbank:

\begin{lstlisting}[language=JavaScript]
{
    "newAddress": "LOZMZOKJ...AWSQFFXX",
    "affectedRows": 1,
    "savedAddresses": [
        "A9XZMWXP...TRKKUHZD",
        "GHDGOJFL...TVOETV9Z",
        "YOUZCEEZ...AKNKQSRW",
        "LOZMZOKJ...AWSQFFXX"
    ],
    "message": "Address is successfully added to the database and will be used for later transactions."
}
\end{lstlisting}

\chapter{Ergebnis der Umsetzung}
Was aus den vorherigen Abschnitten folgt, ist ein Programm, welches mit einem Server �ber eine Rest-Schnittstelle kommunizieren kann und eigenst�ndig eine Transaktion ausf�hren kann. Mit dieser Applikation werden die Bezahlvorg�nge mit IOTAs abgewickelt.

Um das Ergebnis zu veranschaulichen wird das Programm vorgespielt.

...... Einmal probedurchlaufen mit dem Programm und erkl�ren wie das mit dem vorher erkl�rten zusammenh�ngt

Die API wird im ersten Schritt gestartet. Sollte die Verbindung zu der Datenbank gelingen erscheint die Nachricht, dass die API gestartet sei und auf den Port 1717 h�rt. (PICTURE start\_API)