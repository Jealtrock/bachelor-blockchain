\chapter{Einleitung}
Seit 2017 besteht ein starker Wachstum des Interesses an Blockchain, wobei Bitcoin als prominenteste Implementation angesehen wird. Am Sonntag, den 17. Dezember 2017 durchbricht Bitcoin erstmals fast die \$20.000 Marke (\$19,783.21)~\cite{ja:bitcoin20000}. Doch nicht nur Bitcoin hat von dem starken Interesse profitiert, sondern auch viele weitere Kryptow�hrungen sowie andere Blockchain-Implementierungen. Damit war die Technologie in das Bewusstsein der breiten Masse gerutscht und zugleich h�ufen sich Fragen im Bezug auf Zukunftssicherheit und Tauglichkeit der jeweiligen Technologien. Jene Fragen sollen in dieser Bachelorarbeit er�rtert werden.

\section{Kooperationen mit Blockchain/Tangle}
Immer mehr kommerzielle Firmen entdecken das Potenzial von Kryptow�hrungen und weiteren Blockchain-technologien. Die Kooperation zwischen dem Automobilhersteller VW und diversen Kryptow�hrungen ist ein aktuelles Beispiel in dem eine Firma potential in dieser Technologie sieht.
In einer am 8. August, 2018 ver�ffentlichten Kurznachricht auf der Social-Media-Plattform Twitter schreibt die Volkswagen Group:

\begin{quote}
"`Bringing \#blockchain systems to the road: We�re working full steam ahead on making super-safe \#cryptosystems available to our customers. For filling the tank, unlocking your car � and all kinds of other possibilities: \#bitcoin \#ethereum \#iota"'

- Volkswagen Group auf Twitter\footnote{https://twitter.com/VWGroup/status/1027205629436407810}
\end{quote}

In einem Blockeintrag der Volkswagen Group geht hervor, dass sie anhand von Blockchains, die Zuverl�ssigkeit und Sicherheit von Autos erh�hen wollen. Zum Beispiel indem Kilometerz�hler vor Manipulation gesch�tzt sind, und Hackerangriffe auf selbstfahrende Autos vorzubeugen.[2]

Auch weitere Firmen wie Bosch geben in einer Pressemitteilung ende 2017 bekannt, dass auch sie in IOTA investieren und eng mit ihnen an verschiedenen Fronten zusammenarbeiten werden. [3] Sowie auch Microsoft hat sich die Kryptow�hrungen zu Nutzen gemacht, die Nutzer k�nnen anhand von Bitcoin k�ufe im Microsoft Store t�tigen [4].

Es ist davon auszugehen, dass viele weitere namhafte Firmen dem Beispiel folgen werden und auch Blockchain/Tangle Technologien in ihren jeweiligen Anwendungsbereichen einbinden werden. So kristallisiert sich die M�glichkeit heraus, dass Blockchains oder Tangles im zuk�nftigen Alltag eine erhebliche Rolle spielen werden, was zum Beispiel Transaktionen und Sicherheit betrifft.

Aber die Reichweite von Kryptow�hrungen ist nicht nur auf Unternehmen limitiert, sondern erstreckt sich international, sodass diverse Nationen sich umfassend mit der neuen Art von W�hrung besch�ftigen.

\section{Ziel der Thesis}
Grundlegend soll die Frage gekl�rt werden, wie sich ein, auf Blockchain basierendes, Konzept f�r die Zukunft durchsetzen kann und mit welchen realistischen Aussichten zu rechnen ist. Hierbei ist es notwendig, vorhandene Implementierungen nach Einsatzm�glichkeit und Art der Technologie zu kategorisieren und zu beschreiben.

Um einen genauen Ausblick f�r die Zukunft geben zu k�nnen, muss die Entwicklung, Sicherheit und Usability von den zuvor kategorisierten Technologien ausf�hrlich und kritisch untersucht werden. Unterst�tzt wird das Ergebnis der Thesis von einem selbst erstellten Programm, welches mit einer Kryptow�hrung arbeitet. Dieses Programm soll eine m�gliche zuk�nftige Nutzung von Blockchain aufzeigen. Sie soll eine Dienstleistung annehmen k�nnen, und anschlie�end autonom den Dienst mittels einer Kryptow�hrung bezahlen k�nnen. Dies ist ein Anwendungsfall, welcher im Rahmen der zuvor erw�hnten Kooperation zwischen VW und IOTA auch ausgearbeitet wird.
