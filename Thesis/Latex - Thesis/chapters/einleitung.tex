\chapter{Einleitung}
Seit 2017 besteht ein starker Wachstum des Interesses an Blockchain, wobei Bitcoin als erste Implementation angesehen wird~\cite{onlinebeispiel}. Am Sonntag, den 17. Dezember 2017 durchbricht Bitcoin erstmals fast die \$20.000 Marke (\$19,783.21)\footnote{Nach https://www.coindesk.com/900-20000-bitcoins-historic-2017-price-run-revisited/}. Doch nicht nur Bitcoin hat von dem starken Interesse profitiert, sondern auch viele weitere Kryptow�hrungen sowie andere Blockchain-Implementierungen. Damit war die Technologie in das Bewusstsein der breiten Masse gerutscht und zugleich h�ufen sich Fragen im Bezug auf Zukunftssicherheit und Tauglichkeit der jeweiligen Technologien. Jene Fragen sollen in dieser Bachelorarbeit er�rtert werden.

\section{Kooperationen mit Blockchain/Tangle}
Die Kooperation zwischen dem Automobilhersteller VW und diversen Kryptow�hrungen ist ein aktuelles Beispiel in dem eine Firma Potential in einer Distributed Ledger Technologie sieht. In einer am 8. August, 2018 ver�ffentlichten Kurznachricht auf der Social-Media-Plattform Twitter schreibt die Volkswagen Group:

\begin{quote}
"`Bringing \#blockchain systems to the road: We�re working full steam ahead on making super-safe \#cryptosystems available to our customers. For filling the tank, unlocking your car � and all kinds of other possibilities: \#bitcoin \#ethereum \#iota"'

- Volkswagen Group auf Twitter\footnote{https://twitter.com/VWGroup/status/1027205629436407810}
\end{quote}

Und auch in einem Blockeintrag der Volkswagen Group geht hervor, dass sie anhand von Blockchains, die Zuverl�ssigkeit und Sicherheit von Autos erh�hen wollen. Zum Beispiel indem Kilometerz�hler vor Manipulationen gesch�tzt und Hackerangriffe auf selbstfahrende Autos vorgebeugt werden\footnote{Quelle: https://www.volkswagenag.com/en/news/stories/2018/08/putting-blockchains-on-the-road.html}.

Auch weitere Firmen wie Bosch geben in einer Pressemitteilung Ende 2017 bekannt, dass auch sie in IOTA investieren und eng mit ihnen an verschiedenen Fronten zusammenarbeiten werden\footnote{Quelle: https://www.bosch-presse.de/pressportal/de/en/robert-bosch-venture-capital-makes-first-investment-in-distributed-ledger-technology-137411.html}. Sowie auch Microsoft hat sich die Kryptow�hrungen zu Nutzen gemacht. Die Nutzer k�nnen anhand von Bitcoin k�ufe im Microsoft Store t�tigen\footnote{Quelle: https://support.microsoft.com/en-us/help/13942/microsoft-account-add-money-with-bitcoin}.

Es ist davon auszugehen, dass viele weitere namhafte Firmen dem Beispiel folgen werden und auch Blockchain/Tangle Technologien in ihren jeweiligen Anwendungsbereichen einbinden werden. So kristallisiert sich die M�glichkeit heraus, dass Blockchains oder Tangles im zuk�nftigen Alltag eine erhebliche Rolle spielen werden, was zum Beispiel Transaktionen und Sicherheit betrifft.

\section{Zielsetzung}
Das Ziel dieser Bachelor Thesis umfasst die umfangreiche Untersuchung der Distributed Ledger Technologie (kurz DLT). Dabei wird der Fokus auf die verschiedenen M�glichkeiten gelegt, die durch diese jeweiligen Technologien verfolgt werden. DLTs haben verschiedene Schwerpunkte die durch die Bearbeitung der Bachelorthesis auf Zukunftstauglichkeit untersucht werden.

Um die Ausarbeitung zu unterst�tzen wird ein Prototyp implementiert. Dieser soll exemplarisch darstellen, wie mit Kryptow�hrungen und DLTs in Zukunft gearbeitet werden kann. Bei dem Anwendungsfall handelt es sich um ein Fahrzeug, das an einer simulierten futuristischen Tankstelle autonom tankt. Um die entstehenden (Tank-)kosten zu begleichen wird eine Kryptow�hrung verwendet. Umgesetzt wird sowohl die Kundenperspektive aus Sicht eines Autofahrers, als auch die Perspektive der Tankstelle, die sich um die Abwicklung des gesamten Tank-Prozesses k�mmert. Der Zweck dieser Umsetzung ist dem Leser die einzelnen Schritte der Benutzung einer Kryptow�hrung zu demonstrieren. Von der Nutzung des entsprechenden Wallets bis hin zu der Einsicht in die Transaktionshistorie der verwendeten DLT. Der Verlauf und Ablauf einer Transaktion soll deutlich werden. Der Prototyp unterst�tzt bei abgeschlossener Implementierung auch eine Aussage �ber die Zukunftstauglichkeit.

In einer analytischen Ausarbeitung werden die Funktionen und Eigenschaften diverser DLTs detailliert ausgearbeitet und anschlie�end gegen�bergestellt. Dabei steht besonders zum einen die Funktion der einzelnen kryptographischen Methoden und das Verst�ndnis der Transaktionsabwicklung im Vordergrund. Des Weiteren wird eine ausf�hrliche Darlegung der Sicherheitsfunktionen und m�gliche b�swillige Attacken gegen das Netzwerk aufgef�hrt. Bei der abschlie�enden Gegen�berstellung werden die Technologien auf die Fragestellung der Zukunftstauglichkeit einer solchen Technologie projiziert. Daf�r werden zwei grundverschiedene Kryptow�hrungen f�r eine umfangreiche Analyse ausgesucht.

Die Bearbeitung der Thesis soll eine konkrete Vorstellung dar�ber vermitteln, inwiefern eine Zukunft mit Distributed Ledger Technologien vorzustellen ist. Dementsprechend werden kryptographische Grundlagen vermittelt und jeweilige existierende Technologien analysiert.

\section{Abgrenzung}
Diese Arbeit befasst sich mit der kritischen Auseinandersetzung verschiedener Blockchain-Iterationen, was beinhaltet auf welche Weise sie entwickelt wurden, was der verfolgte Kerngedanke ist und mit welchem Ziel die bekannten Sicherheitsprobleme angegangen und durch welche Techniken sie getilgt wurden. Das h�ufigste Vorkommen von Blockchain-Systemen ist in Form von kryptographischen W�hrungen. Es wird sich intensiv mit der Funktion und Sicherheit dieser Systeme auseinandergesetzt jedoch dabei vernachl�ssigt, welchen Einfluss diese W�hrungen auf die Wirtschaft oder den Wertpapierhandel haben. Auch die Untersuchung sogenannter �Mining-Farmen�, in denen ein Gro�teil der Rechenleistung f�r die Berechnung der Blockchain gew�hrleistet wird, findet in dieser Arbeit keine detaillierte Erl�uterung.
