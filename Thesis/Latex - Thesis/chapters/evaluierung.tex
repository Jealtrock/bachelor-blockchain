\chapter{Fazit}
Abschlie�end werden die ausgearbeiteten Ergebnisse zusammengefasst und besprochen. Zudem wird ein Ausblick auf weitere Forschungsbereiche gegeben, die in dieser Arbeit nicht untersucht wurden.

\section{Zusammenfassung}
Der analytische Teil der Arbeit zeigt, dass eine Nutzung der Distributed Ledger Technologien in einer zuk�nftigen Umgebung durchaus denkbar ist. Wenn auch einige Bedenken seitens der Sicherheit gegeben sind. So ist es letztendlich schwer zu sagen ob die Einf�hrung der Quantencomputer nicht weitere Angriffsfl�chen auf die Kryptow�hrungen offenbaren als in dieser Ausarbeitung untersucht. Denn die Sicherheit der kryptographischen Funktionen ist nur gegeben solange keine L�sungen f�r die zugrundeliegenden Probleme gefunden wurden.

Zugleich best�tigt die prototypische Umsetzung der m�glichen Tankanwendung, dass eine Nutzung der Kryptow�hrung schon gegenw�rtig m�glich w�re. Dadurch ist eine positive Prognose �ber den weiteren Verlauf der Nutzung IOTAs als Zahlungsmittel gegeben. Ungeachtet der derzeit nicht vollst�ndig dezentralisierten Architektur von IOTA durch seine Coordinator.

Es ist unm�glich vorherzubestimmen welche Distributed Ledger Technologie sich behaupten k�nnte, jedoch ist eine Durchsetzung einer Kryptow�hrung eine hohe Wahrscheinlichkeit gegeben der Einfachheit des Konzepts. Eine endg�ltige Antwort ist allerdings schwer zu stellen.

\section{Ausblick}
Diese Arbeit hat sich mit der allgemeinen Technologie diverser Kryptow�hrungen besch�ftigt. Dies wurde in Aussicht auf die Beantwortung der Fragestellung durchgef�hrt, inwiefern Distributed Ledger Technologien zukunftstauglich sind. Dabei wurden einige weiterf�hrende Untersuchungen vernachl�ssigt, die im Folgenden kurz aufgef�hrt werden:

\subsection{Auswirkungen auf die Finanzwirtschaft}
Kryptow�hrungen sind Werteinheiten mit denen physische Gegenst�nde, sowohl im Internet als auch in einigen lokalen Gesch�ftstellen, erworben werden k�nnen. Bitcoin und IOTA verwenden pseudonymisierte Benutzerdaten, welche nicht den Besitzern direkt zugeordnet werden k�nnen. Aufgrund dieser Eigenschaft ist es deutlich erschwert den Geldfluss zu beobachten, zu regulieren oder zu unterbinden. Die Auswirkungen dieser Eigenschaften auf die zuk�nftige Finanzwirtschaft wurde in dieser Arbeit nicht behandelt.

\subsection{Die Funktion IOTAs neben der Kryptow�hrung}
IOTA bietet neben der Funktion als Kryptow�hrung noch weitere attraktive Funktionen. So ist es m�glich Dateien in Form eines JSON - Objektes an ein Ziel zu senden. Dabei ist es nicht von Belangen ob das Gegen�ber ein Mensch oder eine Maschine ist. Diese Eigenschaft, kombiniert mit der grenzenlosen Skalierbarkeit der Datenmenge k�nnte das momentane "`Client - Server"' Modell durch ein neues ersetzen. Auch die Art wie Daten im Internet publiziert und erreicht werden, k�nnte durch die Architektur des Tangle-Netzwerkes umgedacht werden.