\chapter{Fazit}
Abschlie�end werden die ausgearbeiteten Ergebnisse zusammengefasst und besprochen. Zudem wird ein Ausblick auf weitere Forschungsbereiche gegeben, die in dieser Arbeit nicht untersucht wurden.

\section{Zusammenfassung}
Der analytische Teil der Arbeit zeigt, dass eine Nutzung der Distributed Ledger Technologien in einer zuk�nftigen Umgebung durchaus denkbar ist. Wenn auch einige Bedenken seitens der Sicherheit gegeben sind. So ist es letztendlich schwer zu sagen ob die Einf�hrung der Quantencomputer nicht weitere Angriffsfl�chen auf die Kryptow�hrungen offenbaren als in dieser Ausarbeitung untersucht. Denn die Sicherheit der kryptographischen Funktionen ist nur gegeben solange keine L�sungen f�r die zugrundeliegenden Probleme gefunden wurden.

Zugleich best�tigt die prototypische Umsetzung der m�glichen Tankanwendung, dass eine Nutzung der Kryptow�hrung schon gegenw�rtig m�glich w�re. Dadurch ist eine positive Prognose �ber den weiteren Verlauf der Nutzung IOTAs als Zahlungsmittel gegeben. Ungeachtet der derzeit nicht vollst�ndig dezentralisierten Architektur von IOTA durch seine Coordinator.

Es ist unm�glich vorherzubestimmen welche Distributed Ledger Technologie sich behaupten k�nnte, jedoch ist eine Durchsetzung einer Kryptow�hrung eine hohe Wahrscheinlichkeit gegeben der Einfachheit des Konzepts. Eine endg�ltige Antwort ist allerdings schwer zu stellen.

\section{Ausblick}
Lorem ipsum