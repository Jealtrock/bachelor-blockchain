\chapter{Analyse der DLT}
In der folgenden Analyse werden zwei Distributed Ledger Architekturen auf ihre Funktion, Sicherheit und Nutzen untersucht. Die gewonnen Informationen werden zum Ende dieses Kapitels gegen�bergestellt, um eine fundierte Aussage �ber die Zukunftstauglichkeit zu treffen. Bei den zwei Technologien handelt es sich zum einen um die popul�re Kryptow�hrung Bitcoin, welche erstmalig das Prinzip der Blockchain umsetzt und neue M�glichkeiten der dezentralen Datenverarbeitung realisiert. Als weitere Distributed Ledger Technologie wird IOTA herangezogen. IOTA leitet sich von "`Internet of Things (IoT)"' ab und stellt eine L�sung f�r das Skalierungsproblem von gro�en Datenmengen bereit. Beide Technologien basieren auf dem Konzept dezentral die Richtigkeit von Daten zu garantieren, indem ein Konsens gefunden wird. Die jeweilige Realisierung dieser Konzepte unterscheidet sich jedoch fundamental. Dank ihrer Unterschiede sind die beiden Technologien besonders geeignet um eine begr�ndete Aussage �ber die allgemeine Zukunftstauglichkeit der Distributed Ledger Architektur zu treffen.

\chapter{Funktion}
Die Funktion von Distributed Ledger Architekturen wird �ber verschiedene Kryptographien und Techniken erm�glicht. Das kommende Kapitel stellt die Funktion dieser Konzepte anhand der jeweiligen Umsetzung in Bitcoin und IOTA detailliert dar.

\section{Bitcoin}
Bitcoin ist ein Distributed Ledger Protokoll das erstmalig Geldtransfer �ber eine dezentrale, anonyme �bereinstimmung der Richtigkeit verf�gt. Diesen Konsens erreicht es �ber die implementierte Blockchain, die wie ein global einsichtiges Kassenbuch funktioniert. Die Blockchain kann auf verschiedene Art und Weise genutzt und entwickelt werden. Sie erm�glicht es �ber das Internet Werte auszutauschen ohne von einem Mittelsmann abh�ngig zu sein~\cite{je:whatIsBitcoin}. Durch kryptographische Techniken sind Daten die in die Blockchain gelangen nur durch sehr gro�en Aufwand manipulierbar, was sie f�r das Speichern von sensiblen Daten attraktiv macht. Im Folgenden wird die grundlegende Funktion der Blockchain sowie dessen Einsatz bei Bitcoin sachlich untersucht.

\subsection{Allgemeine Funktion}
F�r die gegenw�rtige Funktion von Bitcoin und dessen Blockchain sind zwei elementare Techniken notwendig. Zum Einen ist das die Public-Key-Kryptographie zum anderen die kryptographischen Hashfunktionen.

Bei der Public-Key-Kryptographie resp. digitalen Signatur erstellt der Sender einer Nachricht ein Schl�sselpaar bestehend aus einem privaten sowie einem �ffentlichen Schl�ssel. Die beiden Schl�ssel haben zwei erg�nzende Funktionen. Der Autor der Nachricht verwendet den privaten Schl�ssel um diesen mit seinen Informationen zu verbinden und dadurch zu signieren. Die signierten Daten werden gemeinsam mit dem �ffentlichen Schl�ssel an den Empf�nger weitergegeben. Dank des �ffentlichen Schl�ssel ist es dem Empf�nger m�glich die Daten zu authentifizieren. Des Weiteren ist durch die Verkn�pfung der Daten mit dem privaten Schl�ssel die inhaltliche Integrit�t vor Manipulation gesch�tzt.

Durch kryptographische Hashfunktionen entstehen aus Zeichenketten mit variabler L�nge Zeichenketten fester L�nge. Diese Zeichenketten sind �deterministisch�. Die gleichen Eingangsdaten werden nach der Hash-Berechnung immer den gleichen Hash verursachen. Ver�nderte Eingangsdaten f�hren zu einem stark abweichenden Hash.

Au�erdem gibt es drei weitere nennenswerte Eigenschaften von Hashfunktionen. Zum Einen ist es nahezu unm�glich durch den Hash die anf�nglichen Daten wiederherzustellen. Des Weiteren ist es nahezu unm�glich mit abweichenden Eingangsdaten denselben Hash zu generieren. Zuletzt ist es nahezu unm�glich zwei verschiedene Eingangsdaten zu finden aus denen sich derselbe Hashwert ergibt.~\cite{je:whatIsHash}

\subsection{Privater \& �ffentlicher Schl�ssel}\label{ECC}
Der private Schl�ssel in Bitcoin kennzeichnet den Besitz der �bertragenden BTCs (Geldeinheit). Sollte der Schl�ssel verloren gehen, ist es nicht m�glich die BTCs, die dem privaten Schl�ssel zugeordnet sind, wiederherzustellen. Der �ffentliche Schl�ssel wird f�r die Verifikation einer Transaktion ben�tigt, da er durch den privaten Schl�ssel generiert wurde. Man kann durch den �ffentlichen Schl�ssel, nach heutigem Stand, nicht den privaten Schl�ssel erraten. Dieses Ziel wird durch die folgende Kryptographie erreicht:

\paragraph{Endliche Felder}
Ein endliches Feld ist eine Gruppe aus Zahlen, das, wie der Name suggeriert, endlich ist. Ein endliches Feld, das mit Zahlen des Modulo P gebildet wird, bei dem P eine Primzahl ist, f�hrt zu n�tzlichen Funktionen in der Kryptographie.

\paragraph{Elliptische Kurven}
In der Mathematik sind elliptische Kurven aufgrund ihrer Eigenschaft, mathematische Gruppen zu sein, interessant. Eine elliptische Kurve wird nach der folgenden Formel gebildet:

\begin{equation*}
	\{(x,y) \in \mathbb{R}^2 \mid y^2 = x^3 + ax + b,\, 4a^3 + 27b^2 \neq 0 \} \cup \{ 0 \}
\end{equation*}

(PICTURE)
F�r jeden Punkt der elliptischen Kurve gelten die Gesetze der Kommutativit�t, Assoziativit�t und Distributivit�t. Dar�ber hinaus gelten weitere, f�r elliptische Kurven spezifische, Regeln:

\begin{description}
	\item[-] Die Inverse -P eines Punktes P kann durch das Spiegeln des Punktes P an der x-Achse bestimmt werden.
	\item[-] Wenn zwei Punkte (P, Q) einer elliptischen Kurve bekannt sind, kann immer auch ein dritter Punkt (R) bestimmt werden. Die Addition der drei Punkte ergibt in jedem Fall $ P + Q + R = 0 $. Insofern w�re die Rechnung der Punkte zur Bestimmung von Punkt R: $ P + Q = -R $. Punkt -R ist die Inverse von Punk R gespiegelt an der x-Achse.
\end{description}

Eine Formel die diese beiden Konzepte, der elliptischen Kurven und der endlichen Felder, vereint sieht wie folgt aus:

\begin{equation*}
	\{(x,y) \in \mathbb{R}^2 \mid y^2 \equiv x^3 + ax + b\, (mod\, p),\, 4a^3 + 27b^2 \not\equiv 0\, (mod\, p)\} \cup \{ 0 \}
\end{equation*}
Gruppe aus den Reellen Zahlen der elliptischen Kurven des modulo P \cite{je:ellipticCurve}

Dank des Moduls p befindet sich die Gleichung in einem endlichen Feld. Eine elliptische Kurve die sich in einem endlichen Feld befindet ist weiterhin in eine mathematische Gruppe. Alle Formeln k�nnen wie zuvor beschrieben angewandt werden.

Tritt der Fall ein, dass Punkt P sich wie eine Tangente zu der elliptischen Kurve verh�lt, wird das sogenannte Prinzip der Punktdopplung genutzt: 
\begin{equation*}
	P + Q = -R;\: Q = P \rightarrow 2P = -R
\end{equation*}

Es ist m�glich den Punkt P um eine beliebige Anzahl x zu skalieren um einen Punkt R zu erreichen.

\begin{equation*}
	xP = R
\end{equation*}

Da sich Punkt P in einem endlichen Feld befindet, wird durch ein, abh�ngig des Punktes P, gew�hlter bestimmter Skalar x daf�r Sorgen, dass das Produkt auf die Ausgangsposition P verweist. Dieses Prinzip l�sst sich gut durch den Zeiger einer Uhr verdeutlichen. Die Ausgangsuhrzeit ist 3 Uhr. Der Zeiger wandert in drei Stunden Schritten voran. Nach vier Schritten zeigt der Stunden Zeiger erneut auf die Ausgangsuhrzeit 3 Uhr.

Es entsteht eine Untergruppe die durch den Punkt P definiert ist. Die Ordnung n dieser Gruppe P wird so gew�hlt dass nP = 0 ist. In dem Beispiel der Uhrzeit entspricht die Ordnung n = 4 denn $4(3)\: mod(12) = 0$.

Bitcoin nutzt folgende Werte f�r die elliptische Kurven Kryptographie:

\begin{description}
	\item[1.] Der prim�re Wert des Moduls zum Bestimmen des endlichen Raumes entspricht: $2^{256} - 2^{32} - 2^9 - 2^8 - 2^7 - 2^6 - 2^4 - 1 \rightarrow$ FFFFFFFF FFFFFFFF FFFFFFFF FFFFFFFF FFFFFFFF FFFFFFFF FFFFFFFE FFFFFC2F (hex).
	\item[2.] Die elliptische Kurve wird mit den Parametern $a=0$ und $b=7$ gebildet.
	\item[3.] Die Basis der Untergruppe P betr�gt in hexadezimal: 04 79BE667E F9DCBBAC 55A06295 CE870B07 029BFCDB 2DCE28D9 59F2815B 16F81798 483ADA77 26A3C465 5DA4FBFC 0E1108A8 FD17B448 A6855419 9C47D08F FB10D4B8 (hex).
	\item[4.] Die Ordnung n der Untergruppe P betr�gt in hexadezimal: FFFFFFFF FFFFFFFF FFFFFFFF FFFFFFFE BAAEDCE6 AF48A03B BFD25E8C D0364141 (hex).
\end{description}

Der private Schl�ssel ist eine Zahl zwischen 1 und der angegebenen Ordnung. Um den �ffentlichen Schl�ssel zu errechnen, wird der private Schl�ssel mit dem Basispunkt der Untergruppe P multipliziert. Das Ergebnis $mod$ des prim�ren Wertes des Moduls ergibt den �ffentlichen Schl�ssel. Dieser entspricht einem Wert der durch nahezu unendlich M�glichkeiten erreicht werden kann. Nur der Besitzer des privaten Schl�ssels kann diesen Wert gezielt reproduzieren.~\cite{je:ellipticCurve}

\subsection{Transaktionen}
Um eine Transaktion im Bitcoin Netzwerk zu veranschaulichen wird ein fiktives Beispiel herangezogen. Alice m�chte an Bob zwei Bitcoins (BTC) versenden. Es existieren jedoch keine Konten oder Kontost�nde, sondern ausschlie�lich die �ffentliche Liste aller jemals get�tigten Transaktionen, also die Bitcoin - Blockchain selbst, sowie die Hashwerte der einzelnen �ffentlichen Schl�ssel denen die existierenden Transaktionen zugeordnet werden. Aufgrund der vergangenen Transaktionen kann der Wertbestand an BTCs des zugeh�rigen privaten Schl�ssels ermittelt werden. Mithilfe einer digitalen Software (Wallet) kann dieser Wertbestand des privaten Schl�ssels eingesehen und neue Transaktionen eingereicht werden. Um eine Transaktion zu veranlassen werden mithilfe des privaten Schl�ssels die zur�ckliegenden eingegangenen Transaktionen signiert und zusammen mit der ausgehenden Transaktion an das Bitcoin Netzwerk versandt. Dies entspricht in dem anf�nglichen Beispiel einer Auflistung aller Transaktionen in denen Alice involviert war, sowie Alice� Intention zwei BTCs an den �ffentlichen Schl�ssel von Bob zu senden. 

Diese Nachricht gelangt an einen Netzknoten der �ber verschiedene Ressourcen verf�gt. Neben einer aktuellen und vollst�ndigen Kopie der Blockchain, einen Cachespeicher ("`Unspent Transaction Output"' UTXO), der Transaktionswerte der Blockchain enth�lt die noch nicht f�r neue Transaktionen verwendet wurden, existiert noch eine Datenbank mit unbest�tigten Transaktionen. In diesem Netzknoten wird die Legitimit�t von Alice� Transaktion �berpr�ft, indem zum einen ihre Liquidit�t aufgrund ihrer bisherigen Transaktionen, sowie gleichzeitig ihre Signatur anhand des �ffentlichen Schl�ssels, best�tigt wird. Mithilfe der UTXO wird au�erdem gepr�ft, ob die genutzten BTCs nicht bereits anderweitig reserviert wurden. Wenn dies fehlerfrei geschieht, wird die unbest�tigte Transaktion in die daf�r eingerichtete Datenbank aufgenommen. Der Netzknoten meldet diese eingereichte Transaktion an m�glichst viele weitere Netzknoten weiter, die wiederum die Daten nach dem erkl�rten Schema �berpr�fen und anschlie�end in die Datenbank der unbest�tigten Transaktionen aufnehmen.~\cite{je:blockchainBasics}

\subsection{Mining}
Das Bitcoin-Netzwerk verf�gt �ber zwei Arten von Netzwerkknoten. Einer ist ausschlie�lich f�r das zuvor genannte Verifizieren und Speichern der eingehenden Transaktionen sowie dessen Weiterleitung an weitere Netzknoten zust�ndig. Der zweite ist dar�ber hinaus bef�higt neue Bl�cke in der Blockchain zu speichern. Dieser sogenannte �Mining-Netzknoten� speichert zun�chst unbest�tigte Transaktionen in einem Block zusammen. Dieser Block besitzt zus�tzlich einen Blockheader, der f�r die kommende Abspeicherung in der Blockchain essentiell ist.

Es ergeben sich ab diesem Zeitpunkt einige Problematiken. Aufgrund von Verz�gerungen im Netzwerk oder m�glichen Ausf�llen sind einige Netzknoten aktueller. Dies f�hrt dazu, dass verschiedene Transaktionen, welche sich voneinander unterscheiden, in den einzelnen Netzknoten existieren. Au�erdem k�nnen b�sartige Absender gef�lschte Transaktionen weiterleiten, welche dieselben Wert-Ressourcen aufweisen. Bitcoin verwendet f�r diese Problematiken die Proof-Of-Work (PoW) Technologie. In der Regel wird diese Technik f�r das Verhindern von missbr�uchlicher Benutzung von Diensten eingesetzt. Damit ein Dienst genutzt werden kann muss bei dieser Technik ein gewisser Aufwand erbracht werden. Im Falle der Bitcoin-Blockchain muss gem�� dieses Schemas eine rechenintensive Aufgabe gel�st werden. Der Block-Header ergibt einen gewissen Hashwert. Dieser muss solange manipuliert werden bis das Ergebnis unterhalb eines gewissen Zielwertes liegt. Diese partielle Hashinversion beruht auf dem Hashcash-Prinzip von Adam Back (siehe Grundlagen Kapitel \ref{basic_HashCash} auf Seite \pageref{basic_HashCash} ). Der Hash entspringt dem Block-Header und wird aus einer Referenz zum vorherigen Block, einem Zeitstempel, dem Zielwert des Hash-R�tsels, dem Top Hash eines Merkle-Trees, welcher alle Transaktionen durch aufeinander aufbauende Hashes zusammenfasst, sowie einer variablen Zeichenfolge (Nonce) errechnet. Die Nonce wird so oft ge�ndert bis der Hashwert unterhalb des Zielwertes liegt. In anderen Worten muss dieser mit einer gewissen Anzahl von Nullen beginnen. Der erste Mining-Netzknoten der dieses R�tsel l�st versendet seinen Block an das Netzwerk. Dort wird der Block ebenfalls auf Validit�t gepr�ft und bei Erfolg in die eigene Blockchain integriert.

In 1,69\% der F�lle, finden zwei Netzknoten zu nahezu der gleichen Zeit eine L�sung des Hash-R�tsels und versenden ihre jeweiligen Bl�cke. In diesem Fall erhalten unabh�ngige Netzknoten verschieden Versionen der Blockchain. Infolgedessen teilen sich die Netzknoten auf. Dieses Ph�nomen ist als Gabelung bekannt. Die jeweiligen Netzknoten arbeiten weiter auf Grundlage ihrer bekannten Blockchain bis zu dem Moment bei dem sie �ber eine l�ngere Blockchain Version informiert werden. Tritt dieser Fall ein, wird die bekannte Blockchain durch die l�ngere ausgetauscht. Sollte die l�ngere Version einige Transaktionen der zuvor bekannten Blockchain nicht enthalten, werden diese wieder in die Datenbank der nicht best�tigten Transaktionen aufgenommen. Momentan wird ca. alle zehn Minuten ein neuer Block in die Blockchain aufgenommen. Damit sich diese Zeitspanne weiter in einem angemessenen Rahmen befindet, wird stetig die Komplexit�t des Hash-R�tsels angepasst. F�r das Finden der L�sung eines Hash-R�tsels und das erfolgreiche Integrieren eines neuen Blockes in der Blockchain, erh�lt der Mining-Netzknoten eine gewisse Menge an BTCs, wodurch neue BTCs erzeugt werden.~\cite{je:blockchainBasics}

\section{IOTA}
IOTA ist ein Distributed Ledger Protokoll, welches eine grenzenlose Skalierbarkeit des eigenen Netzwerkes zul�sst. Dies wird durch die Eigenschaft erm�glicht, das der Benutzer und der Miner nicht l�nger voneinander getrennt sind. In IOTA wird das Prinzip der durch Bitcoin bekannten Blockchain umgedacht. Transaktionen werden nicht l�nger in einem Block gespeichert, sondern �ber ein Netz aus Transaktionen verteilt. In diesem Netz best�tigen sich die jeweiligen Transaktionen gegenseitig ihre Richtigkeit. Dieses Netz wird Tangle genannt und kann sich wie ein "`Directed Acyclic Graph"' vorgestellt werden. Die folgende Untersuchung liefert einen sachlichen Einblick in die Funktion IOTAs und der implementierten Tangle.

\subsection{IOTA Seed}
Der IOTA Seed ist essentiell wichtig f�r die Funktion von IOTA. Er fungiert als "`Kontonummer"' und verweist auf die Menge an IOTAs die dem Konto zugeordnet werden (das IOTA-Wallet). Wenn der Besitzer des Seeds eine Transaktion durch die IOTA-Tangle veranlassen m�chte, wird mit Hilfe von kryptographischen Hashfunktionen eine Adresse aus dem Seed erstellt. Diese Adresse kann nun genutzt werden um IOTAs zu empfangen oder, sollte sie bereits einen Wert besitzen, zu versenden. Adressen k�nnen nur durch den Seed generiert werden und es ist (technisch) unm�glich von der Adresse den Seed wiederherzustellen. Ein Seed besteht aus 81 Trytes die dank des von der IOTA-Foundation zur Verf�gung gestellten "`Tryte-Alphabet"' durch die 26 Gro�buchstaben A-Z des Alphabets und der Zahl 9 dargestellt werden. \footnote{Diese Vorgaben sind von der IOTA Foundation ver�ffentlicht: https://iota.readme.io/docs/seeds-private-keys-and-accounts}

\subsection{Tangle}\label{grundlagen:tangle}
Bei Tangle handelt es sich um ein Distributed Ledger Software Protokoll, welches sich grundlegend von Blockchain unterscheidet. Die Tangle-Technologie wird von den Entwicklern als n�chste evolution�re Weiterentwicklung der Blockchain beschrieben ~\cite{je:theTangle}. Die Technologie nimmt sich den Schw�chen der Blockchain an und integriert L�sungen f�r diese. Konzeptionell bedeutet dies, dass die heterogene Unterscheidung von Minern und Usern in Blockchains homogenisiert werden soll, indem Transaktionen von Nutzern verifiziert werden sollen, die selber eine Transaktion veranlassen m�chten. So l�st sich das Problem der Skalierung und zugleich fallen die Transaktionsgeb�hren weg. Letzteres ist m�glich, da es zwingend notwendig ist, andere Transaktionen zu verifizieren, falls eine eigene Transaktion aufgegeben werden soll. So "`zahlt"' der Nutzer die Geb�hren mit Rechenleistung.

Tangle wurde von der IOTA-Foundation entwickelt und kommt in der gleichnamigen Kryptow�hrung IOTA zum Einsatz.

Anders als bei Blockchain ist es nicht m�glich, neue Tokens zu "`minen"' (bedeutet: herzustellen). Es besteht seit der Entwicklung ein fester Betrag an verf�gbaren Tokens von genau 2.779.530.283 MIOTA (1 MIOTA = 1.000.000 IOTA) \footnote{Daten bezogen von: https://coinmarketcap.com/currencies/iota/}. Die Tangle richtet sich nach einem Mathematischen Konzept, dem Directed Acyclic Graph. Mit diesem Konzept speichert IOTA die Transaktionen.
Eine Transaktion muss von einem Node aufgegeben werden. Sobald eine Transaktion nicht verifiziert werden kann, kann die urspr�ngliche Transaktion nicht ausgehen.~\cite{je:theTangle}

\subsection{Directed Acyclic Graph}
Die IOTA-Tangle ist aufgebaut in Form eines "`Directed Acyclic Graph"' (DAG) was in diesem Kontext bedeutet, dass sie sich in eine Richtung bewegt und niemals kreisf�rmig ist. Damit eine neue nicht verifizierte Transaktion in die Tangle aufgenommen werden kann, ist neben dem Proof-of-Work eine Verifikation von zwei ebenfalls nicht verifizierten Transaktionen notwendig. Diese beiden Transaktionen werden in dem Transaktions-Bundle abgespeichert und sind dementsprechend referenziert. Da nur nicht verifizierte Transaktionen in diesem Prinzip verwendet werden, entspricht der entstehende Graph eines DAG. \footnote{Daten bezogen von: https://www.forbes.com/sites/shermanlee/2018/01/22/explaining-directed-acylic-graph-dag-the-real-blockchain-3-0}

\subsection{Transaktion}
Eine Transaktion in IOTA besteht aus mehreren Hashes und Werten, die jeweils untereinander eine verschiedene Aufgabe erf�llen. In den folgenden Abschnitten werden diese detailliert aufgef�hrt.

\paragraph{signatureMessageFragment}
Sollte die Transaktion eine ausgehende Zahlung sein, wird dieses Feld ben�tigt um die Signatur des privaten Schl�ssels zu enthalten. Die L�nge der Signatur ist davon abh�ngig, mit welcher Sicherheitsstufe diese Transaktion gehasht wird. Sollte die Sicherheitsstufe 2 oder 3 sein, wird eine weitere wertlose Transaktion ben�tigt. Diese speichert in dem noch freien Signature Message Fragment den Rest der Signatur der vorangehenden ausgehenden Zahlung.

Sollte jedoch die Signatur des privaten Schl�ssels nicht erforderlich sein, verbleibt dieses Feld leer und kann f�r das �bertragen einer Nachricht genutzt werden.
Diesem Feld werden 2187 Trytes reserviert.

\paragraph{hash}
In diesem Feld wird der "`transaction Hash"' nach dem Finden der "`nonce"' und dem Proof-of-Work abgespeichert. Die L�nge betr�gt 81 Trytes.

\paragraph{address}
Die Adresse der Transaktion wird erneut f�r verschiedene Aufgaben verwendet, die sich aus der Art der Transaktionen definieren. Sollte eine Zahlung durchgef�hrt werden, wird in diesem Feld die Adresse des Empf�ngers angegeben. Handelt es sich jedoch um eine urspr�ngliche Geldeingangs-Adresse so ist hier eine Adresse aufgef�hrt die aus einem private Key des Besitzer's Seed erstellt wurde.
F�r dieses Feld sind 81 Trytes reserviert.

\paragraph{value}
Der Wert einer Transaktion definiert dessen Art. Sollte dieser Wert positiv sein, handelt es sich um eine zahlende Transaktion (Output). In dem Feld "`Address"' wird sich in diesem Fall die Adresse des Empf�ngers befinden und das Feld "`Signature Message Fragment"' ist entweder leer oder beinhaltet eine benutzerspezifische Nachricht.

Sollte der Wert negativ sein handelt es sich bei der Transaktion um eine empfangene Transaktion (Input). Das Feld "`Address"' enth�lt entsprechend eine Adresse die dank des Seeds und einem daraus entstandenen private Key generiert wurde. Eine eingehende Transaktion enth�lt einen negativen Wert, um den positiven Wert einer ausgehende Zahlung im Bundle zu rechtfertigen. Zwangsl�ufig wurde diese Adresse zuvor von einer fremden IOTA-Transaktion als Empf�nger Adresse genutzt.

Um den Besitz des entsprechenden privaten Schl�ssels zu beweisen, wird in dem "`Signature Message Fragment"' die Signatur durch den privaten Schl�ssel abgespeichert. Sollte die Sicherheitsstufe gr��er als 1 sein, werden weitere Transaktionen ben�tigt, um die gesamte Signatur zu speichern.

Tr�gt dieses Feld keinen Wert und entspricht 0, handelt es sich bei der Transaktion entweder um das Versenden einer einfachen Nachricht an eine Empf�ngeradresse oder wird genutzt um die Signatur einer vorangehenden Transaktion zu speichern.

F�r dieses Feld wurden 27 Trytes reserviert.

\paragraph{obsoleteTag}
Der obsolete Tag enth�lt einen vom Benutzer definierten Tag. Dieser k�nnte in zuk�nftigen IOTA - Iterationen entfernt werden.
Die reservierte L�nge dieses Feldes betr�gt 27 Trytes.

\paragraph{timestamp}
Der Zeitpunkt ist nicht verpflichtend und kann ausgelassen werden.
Ihm werden 9 Trytes reserviert.

\paragraph{currentIndex}
Dieses Feld zeigt die momentane Position im Bundle.
Ihm werden ebenfalls 9 Trytes reserviert.

\paragraph{lastIndex}
Dieses Feld f�hrt den Index der letzten Transaktion des umfassenden Bundles auf und liefert dementsprechend die L�nge dessen.
Reserviert werden erneut 9 Trytes.

\paragraph{bundle}
Dieses Feld enth�lt den Hash des umfassenden Bundles. Er wird genutzt um alle Transaktionen dieses Bundles zu gruppieren. Jede Transaktion in einem Bundle enth�lt in diesem Feld denselben entsprechenden Bundle Hash. Das Feld umfasst 81 Trytes L�nge.

\paragraph{branch- \& trunkTransaction}
Diese beiden Felder sind jeweils 81 Trytes lang und enthalten zwei zuf�llig gew�hlte, nicht verifizierte Transaktionen aus der Tangle. Nicht verifizierte Transaktionen im Tangle werden "'Tips"' genannt. Jede Transaktion ist verpflichtet, zwei zuf�llig gew�hlte Tips in der Tangle zu verifizieren, um selbst als Tip aufgenommen zu werden.

\paragraph{tag}
Dieser Tag wird vom Benutzer gew�hlt und kann frei vergeben werden. Er kann die Suche einer Transaktion im Tangle unterst�tzen. Seine L�nge in der Transaktion betr�gt 27 Trytes.

\paragraph{attachmentTimestamp}
Dieses neun Trytes gro�e Feld beinhaltet den Zeitpunkt direkt nachdem der Proof-of-Work durchgef�hrt wurde. 

\paragraph{attachmentTimestampLowerBound \& attachmentTimestampUpperBound}
Diese beiden Felder zeigen ein Intervall auf in der die Aufnahme in die Tangle geschah. Sie sind jeweils 9 Trytes gro�.

\paragraph{nonce}
Die nonce ist ein besonders wichtiger Teil einer Transaktion, denn sie wird ben�tigt um den Proof-of-Work durchzuf�hren. Die L�nge dieses Feldes betr�gt 27 Trytes.

Rechnet man die L�ngen aller Felder zusammen, so erh�lt man die gesamte L�nge von 2673 Trytes. Dies ist die von der IOTA-Foundation vorgegebene L�nge die eine Transaktion haben soll. Die Art einer Transaktion bestimmt sich durch den Wert. Sollte eine Transaktion einen negativen Wert haben, so handelt es sich dabei um eine "`Input"' Transaktion. Bei einem positiven Wert, wird Balance an eine "`Output"' Adresse geschickt. Viele Felder tragen eine wichtige Aufgabe f�r die Aufnahme in der Tangle. So wird das Feld "`signatureMessageFragment"` entweder f�r benutzerspezifische Nachrichten genutzt oder, falls ben�tigt, f�r die Signatur.\footnote{Diese Vorgaben sind von der IOTA Foundation ver�ffentlicht: https://iota.readme.io/docs/the-anatomy-of-a-transaction}

\subsection{Proof-of-Work}
Um Transaktionen in die Tangle zu platzieren, sind keine Geb�hren notwendig. Das ist aufgrund zwei essentieller Eigenschaften m�glich. Zum Einen muss jede Transaktion zwei weitere nicht verifizierte Transaktionen best�tigen, des Weiteren wird mit Rechenleistung in Form eines Proof-of-Work die Transaktion beglaubigt. Im Detail l�uft dieses Verfahren wie folgt ab:
Nachdem alle Felder der Transaktion, bis auf die "`nonce"' und den Hash f�r  "`bundle"', einen Wert erhalten haben, wird mit Hilfe eines Algorithmus die nonce gesucht. Diese ist 27 Trytes lang. Sobald die nonce gefunden wurde, wird sie f�r die letzten 27 Trytes der insgesamt 2673 Trytes der Transaktion verwendet. Diese Trytes werden durch den Curl Hash Algorithmus zu einem 81 Trytes langen Hash umgewandelt. Der Hash wird im Anschluss in Trits umgerechnet. Am Ende der resultierenden 243 Trits soll eine Folge aus Nullen stehen. Die l�nge der Folge wird von der IOTA-Foundation vorgegeben. So m�ssen die letzten Trits der Trytes, eines Transaktions Hashes, momentan mindestens 14 mal Null in Folge f�r die Aufnahme in der Tangle und Neun mal Null in Folge f�r die Aufnahme in dem Testnetzwerk betragen. Diese Vorgabe wird "`Minimum Weight Magnitude (MWM)"' genannt. Der Vorgang wird sooft wiederholt bis das vorgegebene MWM erreicht ist. Grunds�tzlich kann die Aussage getroffen werden, dass mit einer steigenden MWM auch die Zeit, die f�r das Finden der passenden nonce ben�tigt wird, exponentiell ansteigt.

\subsection{Bundle}
IOTA ist nach einem Account - Schema aufgebaut, was genauer bedeutet, dass Adressen ben�tigt werden auf die Balance (der Fachausdruck des Geldwertes) eingegangen ist, um Balance weiter zu senden. Ein Bundle in IOTA umfasst in der Regel vier Transaktionen. Die erste Transaktion enth�lt die Adresse des Empf�ngers. Diese Transaktion wird mit einer positiven Balance versehen. In dem Bundle ist dies eine sogenannte "`Output"' - Transaktion. Daraufhin folgen in der Regel, abh�ngig von der gew�hlten Sicherheitsstufe, zwei Transaktionen. Jede dieser Transaktionen basiert auf der gleichen Adresse, auf der zuvor Balance eingegangen ist. In anderen Worten muss diese Adresse zuvor Teil einer "`Output"' - Transaktion gewesen sein. Es werden mehrere Transaktionen ben�tigt, da die Signatur des privaten Schl�ssels mit versandt wird. Diese Signatur vergr��ert sich, abh�ngig von der gew�hlten Sicherheitsstufe. Eine Transaktion, die sich einer Adresse bedient, die zuvor Balance empfing, nennt sich "`Input"' - Transaktion. Sollte die Balance der Input Transaktion nicht ausreichen um den Betrag der Output Transaktion zu decken, m�ssen weitere "`Input"' - Transaktionen angef�gt werden, die jeweils durch ihren privaten Schl�ssel signiert werden. Sollte die Balance der "`Input"' - Adressen den Wert der "`Output"' - Transaktion �berschreiten, wird der Restwert an eine "`Output"' - Adresse des Besitzers �bertragen. Solange der summierte Wert der "`Input"'-Adressen nicht �berschritten wird, k�nnen beliebig viele "`Output"'-Adressen einem Bundle angehangen werden.

Da Bundles atomar sind, werden entweder alle oder keine Transaktion verifiziert.

\subsection{Signatur}\label{OneTimeSignature}
Bei IOTA werden ausgehende Zahlungen (Output) mit zuvor eingegangenen Zahlungen (Input) durchgef�hrt. Um den Besitz des notwendigen Inputs nachzuweisen, verwendet IOTA die "`Winternitz One Time Signature"'. Diese Art der Signatur ver�ffentlicht einen Teil des privaten Schl�ssels und wird aus diesem Grund f�r den einmaligen Gebrauch von Signaturen verwendet. Um die Funktionsweise einer "`One Time Signature"' (kurz OTS) zu verdeutlichen, wird die �hnliche Lamport OTS in einem fiktiven Beispiel herangezogen.

Ein privater Schl�ssel besteht aus zwei gleichlangen Zahlenfolgen "`k1"' \& "`k2"'. Die L�nge des jeweiligen Schl�ssel Teils stimmt mit der L�nge der zu signierenden Nachricht �berein. In diesem Beispiel betr�gt die Schl�ssell�nge 512 (2x 256). Die Nachricht kann eine beliebige Zeichenkette sein, die bspw. durch den SHA-256 Hash Algorithmus zu einem Hash "`H"' mit der L�nge von insgesamt 256 Bits umgewandelt wurde. Der Wert eines Bits kann ausschlie�lich Null oder Eins betragen. In einer Schleife wird jeder Wert des Hashes durchlaufen. Betr�gt der Wert H(n) des Hashes an der Stelle n => H(n) = 0, so wird der Wert des ersten Schl�ssel Teils an selbiger Stelle f�r die Signatur Sig(n) = k1(n)  verwendet. Sollte jedoch der Wert H(n) des Hashes an der Stelle n => H(n) = 1 betragen, wird der Wert des zweiten Schl�ssel Teils in der Signatur Sig(n) = k2(n) verwendet. Nach dem erfolgreichen Durchlaufen des Hashes "`H"' wurde eine Signatur "`Sig"' mit einer L�nge von 256 angefertigt. Die Signatur enth�lt 50\% der Werte des privaten Schl�ssels.

Damit der Empf�nger die Signatur �berpr�fen kann, ben�tigt er den �ffentlichen Schl�ssel. Dieser ist bspw. ein, durch den SHA-256 Hash Algorithmus angefertigter, Hash "`pub"' des privaten Schl�ssels.
Der Empf�nger erh�lt die Nachricht, die Signatur �Sig� und den �ffentlichen Schl�ssel "`pub"'. Er wiederholt den Vorgang, den der Versender der Nachricht f�r die Signatur angewandt hat, mit dem �ffentlichen, mitgelieferten Schl�sselpaar "`pub1"' \& "`pub2"' anstatt des privaten Schl�sselpaars "`k1"' \& "`k2"'. Als Resultat erh�lt er eine gehashte Variante "`HSig"'. Sollte jeder Wert der "`HSig(n)"' identisch mit jedem Hash-Wert der Signatur "`SHA-256( Sig(n) )"' sein, ist die Integrit�t der Nachricht gew�hrleistet.~\cite{je:lamportSignature}

Sollte bei den kommenden Transaktionen erneut der gleiche private Schl�ssel zum Signieren herangezogen werden, wird bei jeder Signatur ein weiterer Teil des privaten Schl�ssels aufgedeckt. Ein b�swilliger Beobachter der Tangle k�nnte die Transaktionen abfangen und mittels Bruteforce-Attacken den gesamten privaten Schl�ssel aufdecken. Der Seed w�re weiterhin unbekannt, sollte der Adresse jedoch ein Wert hinterlegt sein, w�re der Angreifer in der Lage dar�ber zu verf�gen.

\section{Zusammenfassung}
Das Kapitel Funktion umfasst die detaillierte Funktionalit�t von den Kryptow�hrungen Bitcoin und IOTA. Die Kryptographie der privaten und �ffentlichen Schl�ssel der jeweiligen Technologien unterscheidet sich insofern, dass Bitcoin ausschlie�lich einen privaten Schl�ssel und den entsprechenden �ffentlichen Schl�ssel f�r das Erstellen und Bearbeiten einer Transaktion nutzt. IOTA hingegen generiert aus dem Seed mehrere private Schl�ssel dessen �ffentliche Schl�ssel als Adressen gelten. Bitcoin speichert Transaktionen in Bl�cken in der Blockchain, die durch "`minen"' circa alle 10 Minuten um einen weiteren Block verl�ngert wird und �ffentlich einsehbar ist. IOTA verwendet eine Tangle die sich nach dem DAG (Directed Acyclic Graph) in eine Richtung fortbewegt. Transaktionen referenzieren und validieren sich untereinander, wodurch kein 'Mining' notwendig ist. Weiterhin ist keine Grenze hinsichtlich der Skalierbarkeit der Datenmenge gesetzt.
\chapter{Sicherheit}
F�r die Beantwortung der zugrundeliegenden Fragestellung, ob die ausgew�hlten Distributed Ledger Technologien zukunftsf�hig sind, m�ssen die Sicherheitsmechanismen untersucht werden, durch die diese Technologien vor Angriffen gesch�tzt sind. Die gew�hlten Technologien werden dar�ber hinaus auch darauf gepr�ft, wie gut die Zahlungsmittel in der jeweiligen virtuellen Geldb�rse (Wallet) vor Diebstahl gesch�tzt sind. Dies beinhaltet auch zum einen, mit welchen Verschl�sselungsalgorithmen gearbeitet wird und zum anderen wie sicher sie sind. Weiterhin soll ergr�ndet werden, welche Ausma�e fehlerhafte Nutzung der Technologie auf die Sicherheit hat.

\section{Bitcoin}
Bitcoin setzt bei seiner Implementierung auf die Blockchain-Technologie, diese enth�lt intrinsische Sicherheitsrisiken, die bei der Umsetzung dieser Kryptow�hrung adressiert werden mussten. Im folgenden werden diese Risiken und das entsprechende Sicherheitsmerkmal auf Zukunftssicherheit gepr�ft.

\subsection{Konto Sicherheit}
Bitcoins geh�ren zu einer digitalen Geldb�rse. Diese Geldb�rse ist vereinfacht betrachtet ein digitaler Schl�ssel zu dem nur der Eigent�mer Zugang haben sollte. Es ist nur mit diesem Schl�ssel m�glich, zu beweisen welche und wie viele Bitcoins einer bestimmten Person zugeordnet sind. Die Generierung der Schl�ssel basiert auf ein ECC (Elliptic Curve Cryptography) Verschl�sselungsverfahren (siehe zu ECC \ref{ECC} auf Seite \pageref{ECC}). 

Aus diesem privatem Schl�ssel kann der �ffentliche Schl�ssel nachvollzogen werden, die eine Adresse darstellt, an die Bitcoins gesendet oder, von ihr ausgehend, Bitcoins versendet werden. Der �ffentliche Schl�ssel wird nochmals mit SHA256 und dann mit RIPEMD-160 gehasht. Die resultierende Zeichenfolge nennt sich eine P2PKH-Adresse (Pay To Public Key Hash). Sie besteht aus 27-34 alphanumerischer Zeichen mit Ausnahme von einigen Buchstaben/Zahlen f�r eine bessere Lesbarkeit. Dies geschieht als Absicherung f�r den Fall, dass es durch einen neuentdeckten Algorithmus  m�glich sein sollte, das ECDSA-Verschl�sselungsverfahren zu invertieren und den privaten Schl�ssel aus dem �ffentlichen Schl�ssel zu generieren.

Ein privater Schl�ssel bei Bitcoin besteht aus 256 Bits. Daraus erschlie�en sich $ 2^{256} \approx 1,16 \times 10^{77} $ Kombinationsm�glichkeiten f�r einen privaten Schl�ssel. Die Wahrscheinlichkeit, dass eine andere Instanz denselben Schl�ssel generiert, stellt somit kein Sicherheitsrisiko dar.

\subsection{Anonymit�t}
Durch die vollst�ndige, �ffentlich zug�ngliche Historie \emph{aller} Bitcointransaktionen ist das Netzwerk transparent. Es ist m�glich Bitcoins zur�ckzuverfolgen, um nachvollziehen zu k�nnen, durch welche Adressen diese gelaufen sind. 

Aufgrund dessen liegt der Aspekt der Anonymit�t nicht in der Verh�llung der Transaktionen, sondern in der Verschleierung der Identit�t. Private und �ffentliche Schl�ssel-Paare sind von der Identit�t des Nutzers entkoppelt, was wiederum bedeutet, dass der Schutz der Identit�t bei der Verantwortung des Nutzers liegt. Der Nutzer kann durch neue Schl�sselpaare versuchen seine Person weiter zu verschleiern und eine Zur�ckverfolgung zu erschweren. Aber auch ein sparsames Herausgeben von Informationen sollte angef�hrt werden als ein "`Best-Practise"'-Umgang mit Kryptow�hrungen allgemein.

\subsection{Validit�t einer Transaktion}
Eine Transaktion wird von allen Nodes verifiziert an die die Transaktion geschickt wird. Diese �berpr�fen die Transaktion auf diverse Kriterien wie unter anderem auf Syntax und ob die Inputs mit den Outputs �bereinstimmen. Nachdem die Transaktion als valide eingestuft wurde, wird sie in die Liste hinzugef�gt, die sich in dem - zu minenden - Block befindet.

Wenn der passende Hash zu dem Block gefunden wurde, wird dieser Block von der Node in das Netzwerk ausgesendet. Andere Nodes verifizieren diesen Block auf Korrektheit der Syntax und der enthaltenen Transaktionen. Nur wenn diese korrekt sind, dann wird an der Erzeugung des n�chsten Blocks auf Basis des vorherigen Blocks gearbeitet.

Es ist also f�r einen Angreifer nicht m�glich ung�ltige Transaktionen in das Netzwerk zu ver�ffentlichen, da sie nicht von ehrlichen Nodes angenommen werden w�rden. Dasselbe gilt auch f�r ung�ltige Bl�cke. Das wird garantiert durch die vollst�ndige Historie aller Bitcoins seit dem ersten Block, die jede Full-Node gespeichert hat. Es w�re realistisch nicht wahrscheinlich, dass ein Angreifer langfristig eine l�ngere Kette als die Hauptblockchain erzeugen kann, wenn er nicht 50\% der Gesamtleistung des Netzwerkes besitzt (mehr hierzu bei \ref{bitcoin:angriffsszenarios} auf Seite \pageref{bitcoin:angriffsszenarios}). Das ist n�mlich n�tig um als korrekte Kette angesehen zu werden, es gilt:\\Die Kette, die am meisten Rechenleistung b�rgt (l�ngste Kette) ist die Hauptkette.

Jedoch sollte angemerkt werden, dass durch die Wahrscheinlichkeit, dass ein Angreifer kurzfristig mehr Bl�cke erzeugen k�nnte als die wahre Hauptkette, eine Transaktion erst nach ca. sechs weiteren Bl�cken als best�tigt angesehen wird.(QUELLE)

\subsection{Angriffsszenarien}\label{bitcoin:angriffsszenarios}
Im Folgenden werden diverse Angriffsszenarien auf das Bitcoin-Netzwerk beschrieben. Es wird erl�utert mit welchen Mechanismen sich die Blockchain-Technologie vor diesen Angriffen sch�tzt.

\subsubsection{51\%-Attacke}
In Satoshi Nakamotos Bitcoin-Whitepaper wird erkl�rt, wie ein m�glicher Angriff durchzuf�hren w�re. Er beschreibt ein Szenario, welches als 51\%-Attacke bekannt ist. Grundlegend hiermit gemeint, das ein Angreifer mehr als 50\% der Hash-Rate (Gesamteistung) des Netzwerkes besitzt und diese Leistung nutzt um schneller eigene Bl�cke in das Netzwerk zu publizieren.

Dies wird n�tzlich, um Transaktionen die man get�tigt hat ung�ltig zu machen indem man eine weitere Transaktion von der Adresse an eine andere, eigene Adresse erstellt. Zum Beispiel kann damit ein Verk�ufer get�uscht werden, sodass die Person denkt, man h�tte die Bitcoins gezahlt. Im Hintergrund jedoch erstellt der Angreifer seine eigene Blockchain mit der, zu diesem Zweck, erstellten Transaktion an die eigene Adresse, womit zugleich die anf�ngliche Transaktion nicht mehr in einen Block aufgenommen werden kann, da die Bitcoins nicht mehr an der hinterlegten Adresse sind. Es m�ssen solange Blocks erzeugt werden, bis die falsche Blockchain l�nger ist als die ehrliche.

Bei einem erfolgreichen Angriff w�rde die eigene Wallet nicht mit der Zahlung an den Verk�ufer belastet werden. Ein weiterer Nebeneffekt ist, dass wom�glich viele Transaktionen in der Hauptkette nicht mehr validiert sind und erneut aufgenommen werden m�ssen.

Der Angriff hat jedoch Grenzen. Es k�nnen keine ung�ltigen Bl�cke vom Angreifer erstellt werden, da diese von anderen ehrlichen Nodes nicht angenommen werden w�rden. Der Konsens liegt weiterhin bei der Hauptkette.

Dieses Szenario ist auch denkbar bei einer Hash-Rate von unter 50\%, wird jedoch immer unwahrscheinlicher je mehr Bl�cke der Angreifer zur�ckliegt.
\begin{figure}[h]%
\includegraphics[scale=0.75]{26Prozent.PNG}%
\caption{Veranschaulichung zeigt wie unwahrscheinlich es wird, je weiter man zur�ckliegt (Quelle https://www.btc-echo.de/tutorial/bitcoin-51-attacke/)}%
\label{fig:26prozent}%
\end{figure}

Langfristig kann die Hauptkette nur ersetzt werden, wenn �ber 50\% der Hash-Rate dem Angreifer zur Verf�gung stehen.
\begin{figure}[h]%
\includegraphics[scale=0.75]{51Prozent.PNG}%
\caption{Zeigt wie hoch die Wahrscheinlichkeit ist, sechs Bl�cke hintereinander zu generieren (Quelle https://www.btc-echo.de/tutorial/bitcoin-51-attacke/)}%
\label{fig:51prozent}%
\end{figure}

Satoshi Nakamoto f�hrt folgende Formel zur Berechnung der Wahrscheinlichkeit an, dass ein Angreifer die Blockchain einholen kann aus \emph{z} Bl�cken im R�ckstand. Der zeitliche Rahmen ist unbegrenzt.

$ p = $ Wahrscheinlichkeit, dass ein ehrlicher Node den n�chsten Block erstellt\\
$ q = $ Wahrscheinlichkeit, dass der Angreifer den n�chsten Block erstellt\\
$ q_{z} = $ Wahrscheinlichkeit, dass der Angreifer die ehrliche Blockchain �berhohlt\\

$ q_{z} = 1 \quad\quad\quad\: wenn\ p\leq q \\
q_{z} = $ ( $ \frac{p}{q} $ ) $^{z} \quad wenn\ p > q $

Es geht hervor, dass die Chancen des Angreifers die Hauptkette zu �berhohlen bei unter 50\% Hash-Rate exponentiell schwinden.

Diese Angriffsfl�che er�ffnet sich aufgrund der dezentralen Architektur von Blockchain und ist ein integraler Bestandteil eines, auf Konsens beruhenden, Systems.

\subsection{Anreiz zur Ehrlichkeit}
Durch die Entlohnung in Form von Bitcoins durch das Minen und durch die Transaktionsgeb�hren wird Anreiz geschaffen, das Bitcoin-Netzwerk nicht zu manipulieren. Eine Kette mit ung�ltigen Bl�cken w�rde auch nicht von ehrlichen Minern angenommen werden. Daher wird erhofft ~\cite{onlinebeispiel}, dass die Belohnung und das Eigeninteresse an der Integrit�t von Bitcoin ausreicht um Angreifer abzuwehren.

\subsection{Quantenprozessorresistenz}
Mit der Einf�hrung leistungsstarker Quantenprozessoren, die darauf ausgelegt sind, kryptographische Algorithmen zu entschl�sseln, k�nnten Algorithmen (unter anderem ECDSA) unsicher werden. Auch die starke Leistung von Quantencomputern er�ffnet eine neue Angriffsfl�che unter dem Aspekt, dass der Anteil von �ber der H�lfte der Gesamtleistung des Miner-Netzwerkes dem Nutzer langfristig eine l�ngere Blockchain gew�hren kann. Auf dieses Problem wird in Kapitel \ref{bitcoin:angriffsszenarios} eingegangen.

Die genutzten kryptographischen Algorithmen sind nicht bewiesen, dass sie Quantenprozessoren standhalten k�nnten(QUELLE). Indes ist das Bitcoin-Netzwerk durch ihre Architektur und Nutzung nicht gegen Quantencomputern gesch�tzt. Aber komplett ungesch�tzt bleibt die Technologie nicht.

Auch wenn der ECDSA mit einem entsprechenden Algorithmus gel�st werden sollte, oder durch Quantenprozessoren in anderer Hinsicht unsicher werden sollte, wird durch das hashen des �ffentlichen Schl�ssels genau dieser verschleiert. Er kommt nur zum Vorschein, wenn von dieser Adresse eine Zahlung ausgeht, aufgrund der Signatur. Wenn nun von da an ein neues Schl�sselpaar generiert und genutzt wird, sollten die Bitcoins an der neuen Adresse sicher sein.

\section{IOTA}
IOTA nutzt im Gegensatz zu Bitcoin keine Blockchain-Technologie, sondern setzt auf ein sogenanntes Tangle, welches wiederum auf einem mathematischem Konzept basiert, welches sich Directed-Acyclic-Graph (DAG) nennt (siehe zu Tangle \ref{grundlagen:tangle} auf Seite \pageref{grundlagen:tangle}). Dieses Konzept unterscheidet sich grunds�tzlich von Blockchain und weist unter diesem Aspekt andere Sicherheitsrisiken auf. Im folgenden sollen diese beschrieben werden.

\subsection{Konto Sicherheit}
Auch bei IOTA besteht das Konto aus einem digitalen Schl�ssel, dem Seed. Dieser Seed sollte nur dem Eigent�mer verf�gbar gemacht werden, da jede Person mit diesem Seed, �ber die zugeordneten IOTAs verf�gen kann. F�r einen optimalen Schutz vor digitalem Diebstahl sollten Seeds nur lokal (ohne Internetzugang) erstellt und gespeichert werden. Wenn diese allgemeinen Sicherheitsvorschl�ge eingehalten werden, besteht hier noch kein nennenswertes Risiko.

Ein Seed besteht aus 81 Trytes (siehe zu Trytes Grundlagen \ref{grundlagen:tritsundtrytes} auf Seite \pageref{grundlagen:tritsundtrytes}). Ein Tryte enth�lt drei Trits und kann folglich 27 ganze Zahlen darstellen. Daraus ergeben sich $ 27^{81} \approx 8,72 \times 10^{115} $ verschiedene Kombinationsm�glichkeiten. Das hei�t, die Wahrscheinlichkeit, dass eine andere Instanz denselben Seed generiert wie der Eigent�mer ist $ 1 : 8,72 \times 10^{115} $, damit ist die Anzahl der Kombinationsm�glichkeiten deutlich h�her als die Anzahl der Atome im bekannten Universum.(FOOTNOTE)

Durch das One-Time-Winternitz hashing (siehe zu One-Time-Winternitz Funktion \ref{OneTimeSignature} auf Seite \pageref{OneTimeSignature}) ist eine Adresse f�r ausgehende Zahlungen nur einmal zu benutzen, da ein Teil des privaten Schl�ssels bekannt wird. Sollte die Adresse mehr als einmal genutzt werden, so wird zu viel des privaten Schl�ssels �ffentlich gemacht und eine Entschl�sselung wird zunehmend einfacher. Ist der private Schl�ssel vollst�ndig enth�llt, kann der Angreifer auf die IOTAs zugreifen die auf der Adresse hinterlegt sind.

Hierdurch er�ffnet sich ein Risiko, welches der Nutzer von IOTA selbst entgegnen muss, wenn er nicht das offizielle IOTA-Wallet Programm nutzt um Transaktionen zu erstellen. Hierzu muss angemerkt werden, dass durch die Kompromittierung des privaten Schl�ssels nicht der Seed oder alle anderen Adressen gef�hrdet sind.

\subsection{Anonymit�t}
Es ist wichtig zu betrachten, wie Anonym die Identit�t im Tangle-Netzwerk ist. Zum einen was einige Nutzer �ber andere Nutzer wissen k�nnen und zum anderen was in einer Transaktion �ber die Identit�t preisgegeben wird.

Vorerst muss notiert werden, dass Anonymit�t keine Priorit�t in der Entwicklung von IOTA gewesen ist und auch nicht prim�r f�r diesen Zweck entwickelt worden ist. (QUELLE)

Um eine Transaktion zu veranlassen wird ein Paar aus privatem und �ffentlichem Schl�ssel ben�tigt, welche aus dem Seed und einem Adressindex generiert werden. Der �ffentliche Schl�ssel dient hierbei als Adresse die man an seinen Transaktionspartner senden kann. Der private Schl�ssel sollte verborgen bleiben, denn jener Schl�ssel wird genutzt um die Inputs eines Bundles zu signieren. Damit wird bewiesen, dass der Ersteller der Transaktion den privaten Schl�ssel zu der Adresse besitzt und folglich der Eigent�mer ist.

- Erkl�ren was das mit anonymit�t zu tun hat

\subsection{Validit�t einer Transaktion}
Eine Transaktion wird von anderen Nodes validiert. Typischerweise geschieht das wenn ein Nutzer eine Transaktion in das Tangle publizieren will. Daf�r muss der Nutzer zwei Transaktionen validieren. Bei diesen Transaktionen handelt es sich um "`Tips"', die noch nicht referenziert sind. Das Transaktionsobjekt muss die Tips als "`branch"'- und "`trunkTransaction"' referenzieren und �berpr�fen.

Das Objekt wird auf Syntax und G�ltigkeit gepr�ft. Daf�r wird der Transaction-Hash abgeglichen und die Signierung �berpr�ft. 

- wie wird eine einzelne transaktion verifiziert, wie wird bestimmt dass sie g�ltig ist\\
- wann ist eine transaktion verifiziert\\

\subsection{Angriffsszenarien}
Die Tangle erm�glicht durch ihre DAG spezifische Arten von Attacken die im weiteren Fortgang erl�utert werden sollen. Durch die Darstellung der m�glichen Angriffe, werden weitere Sicherheitsmechanismen veranschaulicht.

\subsubsection{34\%-Attacke}
Ein b�swilliger Nutzer k�nnte mit mehr als 34\% der Gesamtrechenleistung im Tangle-Netzwerk einen Angriff starten indem er falsche Transaktionen mit vielen anderen selbst initiierten Transaktionen verifiziert und damit im DAG einen Strang erzeugt der falsche Informationen �ber die verf�gbaren IOTAs enth�lt.

Dem entgegenwirken soll der, von der IOTA-Foundation bereitgestellte, Coordinator. Dieser ist ein Node, der eigenst�ndig Transaktionen erstellt, die als Meilenstein (Milestone) gelten. Der Milestone dient als Beweis, dass die vorherigen Transaktionen korrekt sind. Jede Transaktion muss diese Coordinators indirekt oder direkt referenzieren.(QUELLE)

Diese Coordinators bieten jedoch eine neue Angriffsfl�che. Dabei geht es um das derzeit nicht vollkommen dezentrale System der Tangle bei IOTA. IOTA l�st also noch nicht das Problem, ohne dritte Instanz Transaktionen vermitteln zu k�nnen. Laut IOTA-Foundation sollen die Coordinators ihren Dienst einstellen, sobald das Tangle Netzwerk ausreichend gro� ist, um 34\%-Attacken durch eine zu hohe Anforderung an Rechenleistung abwehren zu k�nnen.(QUELLE)

\subsubsection{Parasiten-Kette}

- wie sch�tzt sich das Bitcoin netzwerk vor Angriffen\\
- welche angriffsszenarios sind denkbar\\
- was kann dagegen getan werden\\

\subsection{Quantenprozessorresistenz}
Bei der Frage nach Zukunftssicherheit bei IOTA geht es auch um die Resistenz vor Quantenprozessoren. Genauer soll betrachtet werden, ob sich die Fertigstellung von diesen leistungsstarken Prozessoren auf die Sicherheit und Best�ndigkeit von IOTAs auswirkt.

Hierbei ist es n�tig, auf den Winternitz onetime hashing algorithmus zur�ckzugreifen um zu verstehen wie sich der Algorithmus auf Quantenresistenz auswirkt.
\chapter{Programmierbarkeit}
In diesem Abschnitt folgt eine Beschreibung der Programmierschnittstellen bei Bitcoin und IOTA. Dies ist relevant da gegebene Schnittstellen und Ressourcen f�r Programmierer die Einsetzbarkeit erh�hen. Wenn Entwickler auf den vorhandenen Code aufbauen und erweitern k�nnen, k�nnen die 
Eine aussage �ber die zuk�nftigen einsatzm�glichkeiten

\section{Bitcoin}
Der Code f�r Bitcoin ist Open-Source und damit frei zug�nglich
Ja was gibts bei bitcoin so keine ahnung

\section{IOTA}
Iota erlaubt 0 vlaue transactions f�r meta daten zb json datein (nutzen hiervon f�r alltag umstritten wegen stormverbrauch), das erlaubt also quasi vershcicken von daten in der tangle
Dadurch erschlie�t sich ein weiterer Anwendungsbereich. Diese Funktion erlaubt eine n�tzliche Anwendung f�r das "`Internet der Dinge"' (internet of things \ref{} auf seite \pageref{}
\chapter{Nutzen}
\section{IOTA}
\section{Bitcoin}

\chapter{Ergebnis}
\chapter{Zusammenfassung}